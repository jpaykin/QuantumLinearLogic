So far we have been talking about quantum computation in the physicist's language
of quantum mechanics---superposition, measurement, and wires. 
But if we want quantum computation to be an area of study for the sake of
computation itself, then computer scientists need to find a way to align
the language of quantum mechanics with the language of programming.
The approach we describe in this section is the work of \cite{abramsky2009categorical},
which axiomatizes the major properties of quantum mechanics and logic as a categorical model.
In this section we will provide an overview of their theory and use it to describe the
teleportation algorithm given in \cref{sec:where}.

Category theory is a powerful tool for describing the algebraic structure of
various mathematical constructions. It allows for the axiomatic description
of mathematical structures and relationships between structures. These axiomatizations
can then be instantiated with concrete mathematical models, like sets,
vector spaces, algebras, et cetera. 

In connection to the theory of computation, category theory can be used as a semantic
domain. Given a programming language, a categorical semantics is a categorical axiomatization
that is sound with respect to the language's operational semantics. 
Thus the language might have many different operational semantics that fit within the same
categorical framework. 
The benefit of axiomatization is that it illuminates exactly the mathematical structure
required by the language, and no more.

There is still much debate over the correct \emph{langauge} to describe quantum computation,
but the semantic domain is very well understood---Hilbert spaces.
Is there any reason to leave this familiar space and move to the abstract domain of category theory?
The hope is that by examining the fundamental structure of Hilbert spaces that makes it possible
to encode quantum algorithms, it will be easier to understand what a language for
quantum computation could look like. It is a step for working backwards, not forwards,
on the categorical semantic chain.

In this section a certain amount of familiarity with category theory is expected,
including objects, morphisms, functors and natural transformations. 
For the unfamiliar reader, \cite{pierce1991basic} provides a good introduction for
computer scientists.

When developing our categorical notions, we will use the category $\Hilb$ of finite-dimensional
Hilbert spaces as a running example. The objects are finite-dimensional Hilbert spaces,
and the morphisms are linear transformations. 

\subsection{Symmetric Monoidal Categories and the Tensor Product}

\begin{definition}
A symmetric monoidal category $C$ consists of a bifunctor $\tensor$
and a unit $I$, along with natural isormorphisms:
\begin{align*}
    \text{Associativity:}&\qquad 
    \alpha_{A,B,C} : A \tensor (B \tensor C) \rightarrow (A \tensor B) \tensor C \\
    \text{Left and right units:}&\qquad 
    \lambda_A : I \tensor A \rightarrow A \text{ and } \rho_A : A \tensor I \rightarrow A \\
    \text{Symmetry:}&\qquad 
    \sigma_{A,B} : A \tensor B \rightarrow B \tensor A \\
\end{align*}
In addition, the monoidal structure must commute with the isomorphisms in 
various ways. For full details, see \cite{bierman1995categorical}.
\end{definition}

\subsubsection*{Example}
The category $\Hilb$ is symmetric monoidal, where the tensor product
is the regular tensor product on vector spaces. Notice that $\tensor$ is
in fact a functor---it is defined both on objects ($H_1 \tensor H_2$ is a Hilbert space)
and on morphisms (matrices).
The unit of the tensor is the underlying field $\C$, with the basis $B = \{ 1 \}$.

The tensor can be thought of as a new kind of composition. Traditional
sequential composition is given by the categorical composition of two morphisms:
\tikzstyle{boxmode} = [draw,minimum height=2em, minimum width=2em]
\begin{center} \begin{tikzpicture}
    \node[boxmode] (f) at (3,0) {$f$};
    \node[boxmode] (g) at (6,0) {$g$};

    \draw (0,0) -- (f);
    \draw (f) -- (g);
    \draw (g) -- (9,0);
\end{tikzpicture} \end{center}
whereas parallel composition is the result of tensoring two morphisms together:
\begin{center} \begin{tikzpicture}
    \node[boxmode] (f) at (3,1) {$f$};
    \node[boxmode] (g) at (3,-0.5) {$g$};

    \draw (0,1) -- (f);
    \draw (f) -- (6,1);
    \draw (0,-0.5) -- (g);
    \draw (g) -- (6,-0.5);
\end{tikzpicture} \end{center}

Along with monoidal categories comes a range of categorical structures tailored
to the monoidal structure. Relevent here are functors and natural transformations.

\begin{definition}
A symmetric monoidal functor is a functor $F : C \Rightarrow D$ between two symmetric
monoidal categories, along with a morphism
\[ m_I^F : I^D \rightarrow F I^C \]
in D and a natural transformation
\[ m_{A,B}^F : F A \tensor F B \rightarrow F(A \tensor B). \]
Again, these morphisms must commute with the monoidal structures in $C$ and $D$
in certain ways not outlined here \citep{bierman1995categorical}.

A monoidal natural transformation between two symmetric monoidal functors
is a natural transformation that commutes with the monoidal components $m_I$ and $m_{A,B}$.
\end{definition}

\subsubsection*{States and Scalars}
There is no default way to talk about the internal structure of objects in category theory,
but the unit of monoidal categories give a kind of work-around. Consider a morphism 
$f : I \rightarrow A$. In $\Hilb$ where $I=\C$, $f$ is uniquely defined by where it sends
the basis element $1$ of $I$. Thus $f(1)$ uniquely identifies a vector in $A$. We call such morphisms
\emph{states} and refer to them as $\Ket{\varphi}$.\footnotemark

\footnotetext{A costate $\Bra{\psi}$ is $\varphi^\dagger : A \rightarrow I$; the $(-)^\dagger$ operator 
is defined later in the section.}

When $f : I \rightarrow I$, $f(1)$ picks out an element in $\C$---a complex number.
Therefore we treat such a morphism as a \emph{scalar}.
Scalars defined in this way give rise to a notion of abstract scalar multiplication: for 
$\alpha : I \rightarrow I$ and $f : A \rightarrow B$, define
\[ \alpha \bullet f : A \xrightarrow{\rho_A^{-1}} A \tensor I
    \xrightarrow{f \tensor \alpha} B \tensor I
    \xrightarrow{\rho_B} B.
\]
It is easy to demonstrate that abstract scalar multiplication is in fact linear in $\alpha$
and, in the case of $\Hilb$, corresponds to regular scalar multiplication.


\subsection{Compact Closed Categories: Names and Conames}

\begin{definition}
A symmetric monoidal category is \emph{compact closed} if every object $A$ is
equipped with a dual object $A^*$, along with morphisms
\[ \eta_A : I \rightarrow A^* \tensor A \qquad\text{and}\qquad
   \varepsilon_A : A \tensor A^* \rightarrow I \]
provided
\[
    A \xrightarrow{\rho_A^{-1}} A \tensor I
    \xrightarrow{\id_A \tensor \eta_A} A\tensor(A^* \tensor A)
    \xrightarrow{\alpha_{A,A^*,A}} (A \tensor A^*)\tensor A
    \xrightarrow{\varepsilon_A \tensor \id_A} I \tensor A
    \xrightarrow{\lambda_A} A
\]
is the identity morphism on $A$, and
\[
    A \xrightarrow{\lambda_{A^*}^{-1}} I \tensor A^*
    \xrightarrow{\eta_A \tensor \id_{A^*}} (A^* \tensor A) \tensor A^*
    \xrightarrow{\alpha_{A^*,A,A^*}^{-1}} A^* \tensor (A \tensor A^*) 
    \xrightarrow{\id_{A^*} \tensor \varepsilon_A} A^* \tensor I
    \xrightarrow{\rho_{A^*}} A^*
\]
is the identity morphism on $A^*$.
\end{definition}

\subsubsection*{Example} In $\Hilb$, the dual
of a Hilbert space $H$ is the space of linear maps from $H$
to the underlying field. For a basis element $b$, let $\delta_b$
be the linear map defined on basis elements as follows:
\[ \delta_b(b') = \begin{cases}
    1 &b'=b \\
    0 &b' \neq b
\end{cases}\]
For a basis $B$ of $H$, the set $\overline{B} = \{\delta_b \mid b \in B\}$
is a basis for $H^*$. Then we can define $\eta$ and $\varepsilon$ as follows:
\begin{align*}
    \eta_H(\unit) &= \sum_{b \in \basis(H)} \delta_b \tensor b \\
    \varepsilon_H(b \tensor \delta_{b'}) &= 
        \begin{cases}   
            \unit &b = b' \\
            \zero &b \neq b'
        \end{cases}
\end{align*}

From the unit $\eta$ and counit $\varepsilon$ we can induce a notion of currying
on arbitrary morphisms.

\begin{definition}
    Let $f : A \rightarrow B$ be a morphism in a compact closed category. The name of $f$,
    denoted $\lift f$, is
    \begin{align*}
        \lift f = 1 
         \xrightarrow{\eta_A} A^* \tensor A
         \xrightarrow{\id_{A^*} \tensor f} A^* \tensor B
    \end{align*}
    The coname of $f$, denoted $\llower f$, is
    \begin{align*}
        \llower f = A \tensor B^*
        \xrightarrow{f \tensor \id_{B^*}} B \tensor B^*
        \xrightarrow{\varepsilon_B} 1
    \end{align*}
\end{definition}
Graphically, we denote the name and coname of $f$ as
\tikzstyle{namenode} = [draw,shape border rotate=180, 
                        isosceles triangle,
                        node distance=2cm, minimum width=3em]
\tikzstyle{conamenode} = [draw,shape border rotate=0,
                        isosceles triangle,
                        node distance=2cm, minimum width=3em]

\begin{center} \begin{tikzpicture}
    \tikzset{invisible/.style={rectangle, minimum height=1.6em}}

    \node[namenode] (name) at (0,1) {$f$};
    \node[invisible] (destination) at (2,1) {};
    \path (name.north east) edge [above] node {$A^*$} (destination.north west);
    \path (name.south east) edge [above] node {$B$}   (destination.south west);
\end{tikzpicture}
\qquad\text{and}\qquad
\begin{tikzpicture}
    \tikzset{invisible/.style={rectangle, minimum height=1.6em}}

    \node[conamenode] (coname) at (0,-1) {$f$};
    \node[invisible] (source) at (-2,-1) {};
    \path (source.north east) edge [above] node {$A$} (coname.north west);
    \path (source.south east) edge [above] node {$B^*$} (coname.south west);
\end{tikzpicture} \end{center}
respectively.
In fact, it is also the case that every morphism $g : I \rightarrow A^* \tensor B$ can
be ``uncurried'', meaning that it is equivalent to some $\lift{f}$ for $f:A \rightarrow B$.
Similarly, all morphisms $A \tensor B^* \rightarrow I$ are equivalent to some coname.
In the case of $\eta$ and $\varepsilon$, we have
\[ \eta_A = \lift{\id_A} \qquad\text{and}\qquad \varepsilon_A = \llower{\id_A}. \]

From the commutativity of the unit and counit, we can then induce certain properties of the
$\lift{-}$ and $\llower{-}$ operations, for example those in \cref{fig:names}.

\begin{figure}
\begin{tabular}{ccc}
&Absorbtion
\\
\begin{tikzpicture}
    \tikzset{invisible/.style={rectangle, minimum height=1.6em}}
    \node[namenode] (f) at (1,1) {$f$};
    \node[boxmode]  (g) at (3,0.66) {$g$};
    \node[invisible] (target) at (5,1) {};
    \path (f.north east) edge [above] node {$A^*$}  (target.north west);
    \path (f.south east) edge [above] node {$B$}    (g);
    \path (g)            edge [above] node {$C$}    (target.south west);
\end{tikzpicture}
&
$=$
&
\begin{tikzpicture}
    \tikzset{invisible/.style={rectangle, minimum height=2.8em}}
    \node[namenode] (gf) at (1,1) {$g \circ f$};
    \node[invisible] (target) at (5,1) {};
    \path (gf.north east) edge [above] node {$A^*$}  (target.north west);
    \path (gf.south east) edge [above] node {$C$}    (target.south west);
\end{tikzpicture}
\\ \\
&Compositionality 
\\
\begin{tikzpicture}
    \tikzset{invisible/.style={rectangle, minimum height=1.6em}}
    \node[invisible] (source)   at (0,.67)    {};
    \node[namenode]  (g)        at (2,0)    {$g$};
    \node[conamenode](f)        at (4,.67)    {$f$};
    \node[invisible] (target)   at (6,0)    {};

    \path (source.north east) edge [above] node {$A$} (f.north west);
    \path (g.north east) edge [above] node {$B^*$} (f.south west);
    \path (g.south east) edge [above] node {$C$} (target.south west);
\end{tikzpicture}
&
$=$
&
\begin{tikzpicture}
    \node[boxmode] (f) at (2,0) {$f$};
    \node[boxmode] (g) at (4,0) {$g$};
    \path (0,0) edge [above] node {$A$} (f);
    \path (f)   edge [above] node {$B$} (g);
    \path (g)   edge [above] node {$C$} (6,0);
\end{tikzpicture}

\end{tabular}
\caption{Commutativity of names and conames}
\label{fig:names}
\end{figure}


\subsubsection*{Example} In $\Hilb$, consider that
\[ \lift{\id_I}(1) = \ket{00} + \ket{11}, \]
which is an entangled state. By scalar multiplication with the constant
$s(1)=\frac 1 {\sqrt{2}}$, we can obtain the normalized state 
$s \bullet \lift{id_I}$ corresponding to 
\[ \frac {\ket{00} + \ket{11}} {\sqrt{2}}. \]
In general, names correspond to preparations of entangled states.

\subsection{Strong Compact Closure and the Inner Product}

So far we have not touched on the notion of inner product in terms of the category theory.
The inner product in the category of finite-dimensional hilbert spaces is sequilinear,
which means it is linear in the second argument and conjugate-linear in the first argument.
In other words,
\[ \Braket{\alpha \varphi_1 + \beta \varphi_2 \mid \psi} 
   = \overline{\alpha} \Braket{\varphi_1 \mid \psi} + \overline{\beta} \Braket{\varphi_2 \mid \psi}
\]
where $\overline{\alpha}$ is the complex conjugate of $\alpha$.


\begin{definition}
    A dagger category is a category $C$ equipped with a contravariant functor $(-)^\dagger$ 
    which is the identity on objects, and such that 
    \[ (g \circ f)^\dagger = f^\dagger \circ g^\dagger \qquad\text{and}\qquad f^{\dagger\dagger} = f. \]
\end{definition}

A morphism $f$ is called \emph{unitary} if $f$ is an isomorphism and $f^{-1}=f^\dagger$.

\subsubsection*{Example} In $\Hilb$, $(-)^\dagger$ is the conjugate-transpose.

\begin{definition}
    A strongly compact closed category is a dagger category which is also compact closed,
    where $\alpha$, $\lambda$, $\rho$ and $\sigma$ are all componentwise unitary transformations,
    where the dagger is a strong monoidal functor, meaning:
    \[ (f \tensor g)^\dagger = f^\dagger \tensor g^\dagger; \]
    and where
    \[ \varepsilon_A = A \tensor A^*
        \xrightarrow{\sigma_{A,A^*}} A^* \tensor A
        \xrightarrow{\eta_A^\dagger} I.
    \]
\end{definition}


Now, the inner product of two states $\Ket{\varphi}$ and $\Ket{\psi}$
is just the scalar
\[ \Braket{\varphi \mid \psi} = \varphi^\dagger \circ \psi : I \rightarrow I. \]

As expected from quantum theory, unitary morphisms preserve the inner product:
\begin{align*}
    \Braket{U\circ \varphi \mid U \circ \psi}
    &= (U \circ \varphi)^\dagger \circ (U \circ \psi) \\
    &= \varphi^\dagger \circ U^\dagger \circ U \circ \psi \\
    &= \varphi^\dagger \circ \psi = \Braket{\varphi \mid \psi}
\end{align*}

In $\Hilb$, $(-)^*$ is the transpose operation and $(-)^\dagger$ is the conjugate-transpose.
We would like to formalize the relationship between these two kinds of duality. 
In particular, if $(-)_*$ is a covariant complex conjugation operation, then 
$f^\dagger = (f_*)^* = (f^*)_*$. Categorically, define $(-)_*$ to act as $(-)^*$ on objects, 
and on morphisms $f: A \rightarrow B$, define 
\[ f_* = (f^\dagger)^* : A^* \rightarrow B^*. \]

\subsection{Measurement via Biproducts}

Abramksy and Coecke's categorical model subscribes to the many-worlds interpretation of quantum
theory. That is, the act of measuring a qubit results in two possible branches---one in 
which the qubit was measured $\Ket{0}$ and one in which the qubit was measured $\Ket{1}$.
As a morphism this measurement will be modeled as
\[ \meas : \Qubit \rightarrow A \oplus A \]
where $\Qubit$ is the space of qubits and $A$ is some resulting state space.

\begin{definition}
    A category is said to have biproducts if for every $A_1,\ldots,A_n$
    there is an object $A_1 \oplus \cdots \oplus A_n$ along with
    projections $\pi_i$ and injections $\iota_i$ forming both a product and a coproduct.
    
    For $f : A \rightarrow C$ and $g : B \rightarrow C$, let $[f,g]$
    denote the morphism from $A \oplus B$ to $C$.

    For $f : A \rightarrow B$ and $g : A \rightarrow C$, let $\langle f,g\rangle$ denote
    the morphism from $A$ to $B \oplus C$.
\end{definition}

We can turn $\oplus$ into a bifunctor as follows: for $f_1:A_1 \rightarrow B_1$
and $f_2 : A_2 \rightarrow B_2$, define 
\begin{align*}
    f_1 \oplus f_2 &: A_1 \oplus A_2 \rightarrow B_1 \oplus B_2 \\
    f_1 \oplus f_2 &= [\iota_1 \circ f_1, \iota_2 \circ f_2].
\end{align*}

\subsubsection*{Example} In the category $\Hilb$, the 
biproduct is the direct sum.\footnote{Also the direct product; the two coincide
for finite-dimensional vector spaces.} The basis of $H_1 \oplus H_2$ is 
\[ \{\inl b_1 \mid b_1 \in B_1 \} \cup \{\inr b_2 \mid b_2 \in B_2 \} \]
where $B_1$ is a basis for $H_1$ and $B_2$ is a basis for $H_2$. In this case,
\begin{align*}
    [f,g](\inl a) &= f(a) \\
    [f,g](\inr a) &= g(a) \\
    \langle f,g \rangle (a) &= \inl f(a) + \inr g(a) \\
    f_1 \oplus f_2(\inl a_1) &= \inl f_1(a_1) \\
    f_1 \oplus f_2(\inr a_2) &= \inr f_2(a_2)
\end{align*}

\subsubsection*{Linearity of morphisms}

In any category with biproducts, we can define addition of morphisms as follows:
For $f, g : A \rightarrow B$, define $f + g : A \rightarrow B$ as
    \[ f + g : A
    \xrightarrow{\langle \id_A, \id_A \rangle}
    A \oplus A
    \xrightarrow{f \oplus g}
    B \oplus B  
    \xrightarrow{[\id_A,\id_A]}
    B
    \]
 \subsubsection*{Example} In $\Hilb$, addition of morphisms corresponds to addition of linear
transformations, as expected. That is,
\begin{align*}
    f+g(a)
    &= [\id_A,\id_A] \circ (f \oplus g) \circ \langle \id_A, \id_A \rangle (a) \\
    &= [\id_A,\id_A] \circ (f \oplus g) (\inl a + \inr a) \\
    &= [\id_A,\id_A] \circ (f \oplus g) (\inl a)
     + [\id_A,\id_A] \circ (f \oplus g) (\inr a) \\
    &= [\id_A,\id_A] (\inl f(a)) + [\id_A,\id_A] (\inr g(a)) \\
    &= f(a) + g(a)
\end{align*}

\subsubsection*{Distributivity}
We can define a right distributivity natural isomorphism (and similarly for left distributivity)
as
\[
    \tau_{A,B,C} = A \tensor (B \oplus C)
    \xrightarrow{\langle \id_A \tensor \pi_1, \id_A \tensor \pi_2\rangle}
    (A \tensor B) \oplus (A \tensor C)
\]
If the biproduct is thought of as parallel worlds, then the distributivity of $\tensor$
over $\oplus$ represents the propogation of classical information (which measurement
branch are we in?) to the quantum data in $A$. The information is not considered
to be duplicated beacuse $B$ and $C$ never occur together.

\subsubsection*{Qubits: Putting it all together}

Recall that in $\Hilb$, the space of qubits is
\[ \Qubit = \{(a,b) \in \C^2 \mid a^2 + b^2=1 \}. \]
There is nothing special about this object categorically, 
but we can utilize the categorical structure to define
measurement of a qubit.

Let $\filter_0,\filter_1 : \Qubit \rightarrow I$ be morphisms in $\Hilb$ as follows:
\begin{align*} \begin{aligned}
    \filter_0(\ket{0}) &= 1 \\
    \filter_0(\ket{1}) &= 0 
\end{aligned} \qquad\qquad \begin{aligned}
    \filter_1(\ket{0}) &= 0 \\
    \filter_1(\ket{1}) &= 1
\end{aligned} \end{align*}
Define $\meas : \Qubit \rightarrow I \oplus I$ as $\meas = \langle \filter_0, \filter_1 \rangle$.
When applied to a qubit, $\meas$ produces two measurement branches, one of which 
contains the state of the qubit if $\ket{0}$ was measured, and one of which contains
the state if $\ket{1}$ was measured. In particular,
\[ \meas(\alpha\ket{0} + \beta\ket{1}) = \inl \alpha + \inr \beta. \]

\subsection{Teleportation}

What is the intended behavior of teleportation? We start out with a single qubit $\ket{\varphi}$
and end up in four distinct measurment brances: $\Qubit \oplus \Qubit \oplus \Qubit \oplus \Qubit$.
(We will write this sum as $4 \cdot \Qubit$.) The state of the qubit in each branch will reflect
the value $\ket{\varphi}$, but scaled by the factor $s^\dagger s$ where $s(1) = \frac 1 {\sqrt{2}}$.
Thus the intended behavior of teleportation is
\[ \Delta^4 := \langle s^\dagger s \bullet \id_\Qubit \rangle_{i=1}^{4} : \Qubit \rightarrow 4 \cdot \Qubit. \]
Next we will trace out the parts of the teleportation algorithm in terms of the categorical 
structures developed in this section.
First we wish to produce the pair of qubits entangled by Alice and Bob. We saw that this entanglement
is constructed as $s \bullet \lift{\id_{\Qubit}}$. Together with the state 
$\ket{\varphi}$ we so far have
\[
    \Qubit \xrightarrow{\rho_\Qubit} \Qubit \tensor I 
    \xrightarrow {\id_\Qubit \tensor s \bullet \lift{\id_\Qubit}} 
    \Qubit \tensor (\Qubit^* \tensor \Qubit)
\]
Now, since the second and third qubits are entangled together, we can think of them as being
spacially together. By applying associativity we model spacial separation:
\[
    \Qubit \tensor (\Qubit^* \tensor \Qubit) 
    \xrightarrow{\alpha_{\Qubit,\Qubit,\Qubit}}
    (\Qubit \tensor \Qubit^*) \tensor \Qubit
\]
The next step is a kind of entanglement, which we saw in \cref{sec:where} is the 
conjugate-transpose of the first. We represent this step as a unary morphism $U$, where 
$U = (H \tensor \id_\Qubit) \circ \CNOT$. In the teleportation algorithm, this becomes
\[
    (\Qubit \tensor \Qubit^*) \tensor \Qubit
    \xrightarrow{U \tensor \id_\Qubit}
    (\Qubit \tensor \Qubit^*) \tensor \Qubit.
\]
Next we perform measurement. Recalling $\filter_0$ and $\filter_1$ from the previous
section, let $\filter_{i,j} : \Qubit \tensor \Qubit^* \rightarrow I$ be
\[ \filter_{i,j} = \Qubit \tensor \Qubit^*
    \xrightarrow{\filter_i \tensor \filter_j^*}
    I \tensor I
    \xrightarrow{\lambda_I}
    I
\]
which expands to
\[
    \filter_{i,j}(\ket{i'j'}) = \begin{cases}
        1 & i=i' \text{ and } j=j' \\
        0 & \text{otherwise}
    \end{cases}
\]
%% \begin{align*}
%% \begin{aligned}
%%     \filter_{0,0} \ket{00} &= \sqrt{2} \\
%%     \filter_{0,0} \ket{01} &= 0 \\
%%     \filter_{0,0} \ket{10} &= 0 \\
%%     \filter_{0,0} \ket{11} &= 0 
%% \end{aligned}
%% \qquad
%% \begin{aligned}
%%     \filter_{0,1} \ket{00} &= 0 \\
%%     \filter_{0,1} \ket{01} &= \sqrt{2} \\
%%     \filter_{0,1} \ket{10} &= 0 \\
%%     \filter_{0,1} \ket{11} &= 0 
%% \end{aligned}
%% \qquad
%% \begin{aligned}
%%     \filter_{1,0} \ket{00} &= 0 \\
%%     \filter_{1,0} \ket{01} &= 0 \\
%%     \filter_{1,0} \ket{10} &= \sqrt{2} \\
%%     \filter_{1,0} \ket{11} &= 0 
%% \end{aligned}
%% \qquad
%% \begin{aligned}
%%     \filter_{1,1} \ket{00} &= 0 \\
%%     \filter_{1,1} \ket{01} &= 0 \\
%%     \filter_{1,1} \ket{10} &= 0 \\
%%     \filter_{1,1} \ket{11} &= -\sqrt{2} 
%% \end{aligned}
%% \end{align*}
Then consider
\[  (\Qubit \tensor \Qubit^*) \tensor \Qubit
    \xrightarrow{\langle \filter_{i,j} \rangle_{i,j} \tensor \id_\Qubit} 
    (4 \cdot I) \tensor \Qubit. 
\]
Now, the information regarding the result of the measurement is located on the left-hand-side of the tensor,
isolated from the remaining qubit. It is possible to think of the distribution of
tensor over $\oplus$ as the communication of the Newtonian information (what measurement
branch we are in) to the location of the qubit. 
\[
    (4 \cdot I) \tensor \Qubit
    \xrightarrow{ \tau }
    4 \cdot (I \tensor \Qubit) 
    \xrightarrow{ \bigoplus \lambda_\Qubit}
    4 \cdot \Qubit
\]
Finally, in each branch we need to perform the unitary correction $F_{i,j}$:
\[
    4 \cdot \Qubit
    \xrightarrow{ \bigoplus_{i,j} F_{i,j} }
    4 \cdot \Qubit
\]
Recall that we have defined
\begin{align*}
\begin{aligned}
    F_{0,0} \ket{0} &= \ket{0} \\
    F_{0,0} \ket{1} &= \ket{1} 
\end{aligned}
\qquad
\begin{aligned}
    F_{0,1} \ket{0} &= \ket{1} \\
    F_{0,1} \ket{1} &= \ket{0} 
\end{aligned}
\qquad
\begin{aligned}
    F_{1,0} \ket{0} &= \ket{0} \\
    F_{1,0} \ket{1} &= -\ket{1} 
\end{aligned}
\qquad
\begin{aligned}
    F_{1,1} \ket{0} &= -\ket{1} \\
    F_{1,1} \ket{1} &= \ket{0} 
\end{aligned}
\end{align*}

To prove correctness of the teleportation protocol, we show that $\Delta^4$ is
the composition of each of these steps:
\begin{align*}
    \Delta^4 =
    \Qubit 
    &\xrightarrow{\rho_\Qubit} \Qubit \tensor I 
    \xrightarrow {\id_\Qubit \tensor s \bullet \lift{\id_\Qubit}} 
    \Qubit \tensor (\Qubit^* \tensor \Qubit) \\
    &\xrightarrow{\alpha_{\Qubit,\Qubit,\Qubit}}
    (\Qubit \tensor \Qubit^*) \tensor \Qubit \\
    &\xrightarrow{U \tensor \id_\Qubit}
    (\Qubit \tensor \Qubit^*) \tensor \Qubit \\
    &\xrightarrow{\langle \filter_{i,j} \rangle_{i,j} \tensor \id_\Qubit} 
    (4 \cdot I) \tensor \Qubit \\
    &\xrightarrow{ \tau }
    4 \cdot (I \tensor \Qubit) 
    \xrightarrow{ \bigoplus \lambda_\Qubit} 
    4 \cdot \Qubit \\
    &\xrightarrow{ \bigoplus_{i,j} F_{i,j} }
    4 \cdot \Qubit.
\end{align*}
The first thing to notice is that 
\[
    \langle \filter_{i,j} \rangle_{i,j} \circ U = \langle \filter_{i,j} \circ U \rangle_{i,j}. 
\]
Furthermore, it happens to be the case that
\[ 
    \filter_{i,j} \circ U = \llower{s^\dagger \bullet F_{i,j}^\dagger}.
\]
Graphically, we can now represent a single branch of the measurement as follows:
\begin{center} \begin{tikzpicture}
    \tikzset{invisible/.style={rectangle, minimum height=1.6em}}
    \node[invisible] (source)   at (0,1.8)    {};
    \node[namenode]  (x)        at (2,0)      {$s \smash{\bullet\id_\Qubit}$};
    \node[conamenode](y)        at (4,1.4)    {$s^\dagger \smash{\bullet F_{i,j}^\dagger}$};
    \node[boxmode]   (z)        at (6,-.7)    {$F_{i,j}$};
    \node[invisible] (target)   at (8,-.7)    {};

    \path (source.north east) edge [above] node {} (y.north west);
    \path (x.north east) edge [above] node {} (y.south west);
    \path (x.south east) edge [above] node {} (z.west);
    \path (z.east) edge [above] node {} (target.west);
\end{tikzpicture} \end{center}
But using the compositionality property from \cref{fig:names}, this is equivalent to
\begin{center} \begin{tikzpicture}
    \node[boxmode] (y) at (2,0) {$s^\dagger \bullet F_{i,j}^\dagger$};
    \node[boxmode] (x) at (4,0) {$s \bullet \id_\Qubit$};
    \node[boxmode] (z) at (6,0) {$F_{i,j}$};
    \path (0,0) edge [above] node {} (y);
    \path (y)   edge [above] node {} (x);
    \path (x)   edge [above] node {} (z);
    \path (z)   edge [above] node {} (8,0);
\end{tikzpicture} \end{center}
Because $F_{i,j}$ is unitary, their composition is
\begin{align*}
    F_{i,j} \circ s \bullet \id_\Qubit \circ s^\dagger \bullet F_{i,j}^\dagger 
    &= s^\dagger s \bullet(F_{i,j} \circ \id_\Qubit \circ F_{i,j}^\dagger) \\
    &= s^\dagger s \bullet \id_\Qubit
\end{align*}
By definition,
$\Delta^4 = \langle s^\dagger s \bullet \id_\Qubit \rangle_{i=1}^4$, which completes the proof.

