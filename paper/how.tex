So far we have been talking about quantum computation in the physicist's language
of quantum mechanics---superposition, measurement, and wires. 
But if we want quantum computation to be an area of study for the sake of
computation itself, then computer scientists need to find a way to align
the language of quantum mechanics with the language of programming.
The approach we describe in this section is the work of \cite{abramsky2009categorical},
which axiomatizes the major properties of quantum mechanics and logic as a categorical model.
In this section we will provide an overview of their theory and use it to describe the
teleportation algorithm given in \Section{where}.

Category theory is a powerful tool for describing the algebraic structure of
various mathematical constructions. Category theory allows for axiomatic descriptions
of mathematical structures and relationships between structures. These axiomatic
properties can then be instantiated with concrete mathematical structures, like sets,
vector spaces, algebras, et cetera. 

In connection to the theory of computation, category theory can be used as a semantic
domain. Given a programming language, a categorical semantics is a categorical axiomatization
that is sound with respect to the language's operational semantics. 
Thus the language might have many different operational semantics that fit within the same
categorical framework. 
The benefit of axiomatization is that it illuminates exactly the mathematical structure
outlined by the language, and no more.

There is still much debate over the correct \emph{langauge} to describe quantum computation,
but the semantic domain is very well understood---Hilbert spaces.
Is there any reason to leave this familiar space and move to the abstract domain of category theory?
The hope is that by examining the fundamental structure of Hilbert spaces that makes it possible
to encode quantum algorithms, it will be easier to understand what a language for
quantum computation should look like. It is a step for working backwards, not forwards,
on the categorical semantic chain.

When developing categorical notions, we will use the category $\Hilb$ of finite-dimensional
hilbert spaces as a running example. The objects are finite-dimensional hilbert spaces,
and the morphisms are linear transformations. \jp{Not necessarily unitary...}

A certain amount of familiarity with category theory is expected: objects, morphisms,
functors and natural transformations.

\subsection{Symmetric Monoidal Categories and the No-Cloning Theorem}

\begin{definition}
A symmetric monoidal category $C$ consists of a bifunctor $\tensor$
and a unit $I$, along with natural isormorphisms:
\begin{align*}
    \text{Associativity:}&\qquad 
    \alpha_{A,B,C} : A \tensor (B \tensor C) \rightarrow (A \tensor B) \tensor C \\
    \text{Left and right units:}&\qquad 
    \lambda_A : I \tensor A \rightarrow A \text{ and } \rho_A : A \tensor I \rightarrow A \\
    \text{Symmetry:}&\qquad 
    \sigma_{A,B} : A \tensor B \rightarrow B \tensor A \\
\end{align*}
In addition, the monoidal structure must commute with the isomorphisms in 
various ways \jp{Cite}.
\end{definition}

\subsubsection*{Example}
The category $\Hilb$ is symmetric monoidal, where the tensor product
is the regular tensor product on vector spaces. Notice that $\tensor$ is
in fact a functor---it is defined both on objects ($H_1 \tensor H_2$ is a hilbert space)
and on morphisms

\jp{Move to What section}
A few properties of this tensor
product are:
\begin{enumerate}
    \item If $B_1$ is a basis of $H_1$ and $B_2$ is a basis of $H_2$
    then $B_1 \times B_2$ is a basis for $H_1 \tensor H_2$.
    \item For arbitrary $u_1 \in H_1$ and $u_2 \in H_2$, 
    $u_1 \tensor u_2$ is the equivalence class in $H_1 \tensor H_2$ induced
    by bilinearity:
    \begin{align*}
        \alpha(u_1 \tensor u_2) &= (\alpha u_1) \tensor u_2 \\
        \alpha(u_1 \tensor u_2) &= u_1 \tensor (\alpha u_2) \\
        (u_1 \tensor v) + (u_2 \tensor v) &= (u_1 + u_2) \tensor v \\
        (u \tensor v_1) + (u \tensor v_2) &= u \tensor (v_1 + v_2)
    \end{align*}
\end{enumerate}
The unit of the tensor is the underlying field, with the basis $B = \{ \unit \}$.

The tensor can be thought of as a new kind of composition. Traditional
sequential composition is given by the categorical composition of two morphisms:
\tikzstyle{boxmode} = [draw,minimum height=2em, minimum width=2em]
\begin{center} \begin{tikzpicture}
    \node[boxmode] (f) at (3,0) {$f$};
    \node[boxmode] (g) at (6,0) {$g$};

    \draw (0,0) -- (f);
    \draw (f) -- (g);
    \draw (g) -- (9,0);
\end{tikzpicture} \end{center}
whereas parallel composition is the result of tensoring two morphisms together:
\begin{center} \begin{tikzpicture}
    \node[boxmode] (f) at (3,1) {$f$};
    \node[boxmode] (g) at (3,-0.5) {$g$};

    \draw (0,1) -- (f);
    \draw (f) -- (6,1);
    \draw (0,-0.5) -- (g);
    \draw (g) -- (6,-0.5);
\end{tikzpicture} \end{center}


Along with monoidal categories comes a range of categorical structures tailored
to the monoidal structure. Relevent here are functors and natural transformations.

\begin{definition}
A symmetric monoidal functor is a functor $F : C \Rightarrow D$ between two symmetric
monoidal categories, along with a morphism
\[ m_I^F : I^D \rightarrow F I^C \]
in D and a natural transformation
\[ m_{A,B}^F : F A \tensor F B \rightarrow F(A \tensor B). \]
Again, these morphisms must commute with the monoidal structures in $C$ and $D$
in certain ways not outlined here. 
\end{definition}

A monoidal natural transformation between two symmetric monoidal functors
is a natural transformation that commutes with the monoidal components $m_I$ and $m_{A,B}$.

\subsection{Compact Closed Categories: Names and Conames}

\begin{definition}
A symmetric monoidal category is \emph{compact closed} if every object $A$ is
equipped with a dual object $A^*$, along with morphisms
\[ \eta_A : I \rightarrow A^* \tensor A \qquad\text{and}\qquad
   \varepsilon_A : A \tensor A^* \rightarrow I \]
provided
\[
    A \xrightarrow{\rho_A^{-1}} A \tensor I
    \xrightarrow{\id_A \tensor \eta_A} A\tensor(A^* \tensor A)
    \xrightarrow{\alpha_{A,A^*,A}} (A \tensor A^*)\tensor A
    \xrightarrow{\varepsilon_A \tensor \id_A} I \tensor A
    \xrightarrow{\lambda_A} A
\]
is the identity morphism on $A$, and
\[
    A \xrightarrow{\lambda_{A^*}^{-1}} I \tensor A^*
    \xrightarrow{\eta_A \tensor \id_{A^*}} (A^* \tensor A) \tensor A^*
    \xrightarrow{\alpha_{A^*,A,A^*}^{-1}} A^* \tensor (A \tensor A^*) 
    \xrightarrow{\id_{A^*} \tensor \varepsilon_A} A^* \tensor I
    \xrightarrow{\rho_{A^*}} A^*
\]
is the identity morphism on $A^*$.
\end{definition}

\subsubsection*{Example} In $\Hilb$, the dual
of a Hilbert space $H$ is the space of continuous linear maps from $H$
to the underlying field. For a basis element $b$, let $\delta_b$
be the linear map defined on basis elements as follows:
\[ \delta_b(b') = \begin{cases}
    1 &b'=b \\
    0 &b' \neq b
\end{cases}\]
For a basis $B$ of $H$, the set $\overline{B} = \{\delta_b \mid b \in B\}$
is a basis for $H^*$. Then we can define $\eta$ and $\varepsilon$ as follows:
\begin{align*}
    \eta_H(\unit) &= \sum_{b \in \basis(H)} \delta_b \tensor b \\
    \varepsilon_H(b \tensor \delta_{b'}) &= 
        \begin{cases}   
            \unit &b = b' \\
            \zero &b \neq b'
        \end{cases}
\end{align*}

From the unit $\eta$ and counit $\varepsilon$ we can induce a notion of currying
on arbitrary morphisms.

\begin{definition}
    Let $f : A \rightarrow B$ be a morphism in a compact closed category. The name of $f$,
    denoted $\lift f$, is
    \begin{align*}
        \lift f = 1 
         \xrightarrow{\eta_A} A^* \tensor A
         \xrightarrow{\id_{A^*} \tensor f} A^* \tensor B
    \end{align*}
    The coname of $f$, denoted $\llower f$, is
    \begin{align*}
        \llower f = A \tensor B^*
        \xrightarrow{f \tensor \id_{B^*}} B \tensor B^*
        \xrightarrow{\varepsilon_B} 1
    \end{align*}
\end{definition}
Graphically, we denote the name and coname of $f$ as
\tikzstyle{namenode} = [draw,shape border rotate=180, 
                        isosceles triangle,
                        node distance=2cm, minimum width=3em]
\tikzstyle{conamenode} = [draw,shape border rotate=0,
                        isosceles triangle,
                        node distance=2cm, minimum width=3em]

\begin{center} \begin{tikzpicture}
    \tikzset{invisible/.style={rectangle, minimum height=1.6em}}

    \node[namenode] (name) at (0,1) {$f$};
    \node[invisible] (destination) at (2,1) {};
    \path (name.north east) edge [above] node {$A^*$} (destination.north west);
    \path (name.south east) edge [above] node {$B$}   (destination.south west);

    \node[conamenode] (coname) at (0,-1) {$f$};
    \node[invisible] (source) at (-2,-1) {};
    \path (source.north east) edge [above] node {$A$} (coname.north west);
    \path (source.south east) edge [above] node {$B^*$} (coname.south west);
\end{tikzpicture} \end{center}
respectively.
In fact, it is also the case that every morphism $g : I \rightarrow A^* \tensor B$ can
be ``uncurried'', meaning that it is equivalent to some $\lift{f}$ for $f:A \rightarrow B$.
Similarly, all morphisms $A \tensor B^* \rightarrow I$ are equivalent to some coname.
In the case of $\eta$ and $\varepsilon$, we have
\[ \eta_A = \lift{\id_A} \qquad\text{and}\qquad \varepsilon_A = \llower{\id_A}. \]

From the commutativity of the unit and counit, we can then induce certain properties of the
$\lift{-}$ and $\llower{-}$ operations, for example those in \Figure{names}.

\begin{figure}
\begin{tabular}{cccc}
Absorbtion
&
\begin{tikzpicture}
    \tikzset{invisible/.style={rectangle, minimum height=1.6em}}
    \node[namenode] (f) at (1,1) {$f$};
    \node[boxmode]  (g) at (3,0.66) {$g$};
    \node[invisible] (target) at (5,1) {};
    \path (f.north east) edge [above] node {$A^*$}  (target.north west);
    \path (f.south east) edge [above] node {$B$}    (g);
    \path (g)            edge [above] node {$C$}    (target.south west);
\end{tikzpicture}
&
$=$
&
\begin{tikzpicture}
    \tikzset{invisible/.style={rectangle, minimum height=2.8em}}
    \node[namenode] (gf) at (1,1) {$g \circ f$};
    \node[invisible] (target) at (5,1) {};
    \path (gf.north east) edge [above] node {$A^*$}  (target.north west);
    \path (gf.south east) edge [above] node {$C$}    (target.south west);
\end{tikzpicture}
\\ \\
Compositionality
&
\begin{tikzpicture}
    \tikzset{invisible/.style={rectangle, minimum height=1.6em}}
    \node[invisible] (source)   at (0,.67)    {};
    \node[namenode]  (g)        at (2,0)    {$g$};
    \node[conamenode](f)        at (4,.67)    {$f$};
    \node[invisible] (target)   at (6,0)    {};

    \path (source.north east) edge [above] node {$A$} (f.north west);
    \path (g.north east) edge [above] node {$B^*$} (f.south west);
    \path (g.south east) edge [above] node {$C$} (target.south west);
\end{tikzpicture}
&
$=$
&
\begin{tikzpicture}
    \node[boxmode] (f) at (2,0) {$f$};
    \node[boxmode] (g) at (4,0) {$g$};
    \path (0,0) edge [above] node {$A$} (f);
    \path (f)   edge [above] node {$B$} (g);
    \path (g)   edge [above] node {$C$} (6,0);
\end{tikzpicture}

\end{tabular}
\caption{Commutativity of names and conames}
\label{fig:names}
\end{figure}


\subsection{Strong Compact Closure and the Inner Product}

So far we have not touched on the notion of inner product in terms of the category theory.
The inner product in the category of finite-dimensional hilbert spaces is sequilinear,
which means it is linear in the second argument and conjugate-linear in the first argument.
In other words,
\[ \Braket{\alpha \varphi_1 + \beta \varphi_2 \mid \psi} 
   = \overline{\alpha} \Braket{\varphi_1 \mid \psi} + \overline{\beta} \Braket{\varphi_2 \mid \psi}
\]
where $\overline{\alpha}$ is the complex conjugate of $\alpha$.


\begin{definition}
    A dagger category is a category $C$ equipped with a contravariant functor $(-)^\dagger$ 
    which is the identity on objects, and such that 
    \[ (g \circ f)^\dagger = f^\dagger \circ g^\dagger \qquad\text{and}\qquad f^{\dagger\dagger} = f. \]
\end{definition}

A morphism $f$ is called \emph{unitary} if $f$ is an isomorphism and $f^{-1}=f^\dagger$.

\subsubsection*{Example} In $\Hilb$, $(-)^\dagger$ is the conjugate-transpose.

\begin{definition}
    A strongly compact closed category is a dagger category which is also compact closed,
    where $\alpha$, $\lambda$, $\rho$ and $\sigma$ are all componentwise unitary transformations,
    where the dagger is a strong monoidal functor, meaning:
    \[ (f \tensor g)^\dagger = f^\dagger \tensor g^\dagger; \]
    and where
    \[ \varepsilon_A = A \tensor A^*
        \xrightarrow{\sigma_{A,A^*}} A^* \tensor A
        \xrightarrow{\eta_A^\dagger} I.
    \]
\end{definition}

In order to define the inner product, we need to define the domain of the inner product--states--and
the codomain of the inner product--scalars. Of course, category theory gives us no way to explicitly
talk about the content of objects, but we can simulate it using the monadic unit. Consider a morphism
$f : I \rightarrow A$. In $\Hilb$ where $I=\mathcal{C}$, $f$ is uniquely defined by where it sends
the basis element $1$ of $I$. Thus $f(1)$ uniquely identifies a vector in $A$. We call such morphisms
\emph{states} and refer to them as $\Ket{\varphi}$. A state $\Bra{\psi}$ is $\varphi^\dagger : A \rightarrow I$.

When $f : I \rightarrow I$, $f(1)$ picks out an element in the unit; for $\Hilb$ this is a complex
number. Therefore we treat such a morphism as a \emph{scalar} and refer to them as $\alpha$.
Scalars defined in this way give rise to a notion of scalar multiplication: for 
$\alpha : I \rightarrow I$ and $f : A \rightarrow B$, define
\[ \alpha \bullet f : A \xrightarrow{\rho_A^{-1}} A \tensor I
    \xrightarrow{f \tensor \alpha} B \tensor I
    \xrightarrow{\rho_B} B.
\]
It is easy to demonstrate that scalar multiplication is in fact linear in $\alpha$.

Now, the inner product of two states $\Ket{\varphi}$ and $\Ket{\psi}$
is just the scalar
\[ \Braket{\varphi \mid \psi} = \varphi^\dagger \circ \psi : I \rightarrow I. \]

As expected from quantum theory, unitary morphisms preserve the inner product:
\begin{align*}
    \Braket{U\circ \varphi \mid U \circ \psi}
    &= (U \circ \varphi)^\dagger \circ (U \circ \psi) \\
    &= \varphi^\dagger \circ U^\dagger \circ U \circ \psi \\
    &= \varphi^\dagger \circ \psi = \Braket{\varphi \mid \psi}
\end{align*}

In $\Hilb$, $(-)^*$ is the transpose operation and $(-)^\dagger$ is the conjugate-transpose.
We would like to formalize the relationship between the two kinds of duality. In particular, if 
$(-)_*$ is a covariant complex conjugation operation, then $f^\dagger = (f_*)^* = (f^*)_*$.
Categorically, define $(-)_*$ to act as $(-)^*$ on objects, and on morphisms $f: A \rightarrow B$,
define 
\[ f_* = (f^\dagger)^* : A^* \rightarrow B^*. \]

\subsection{Measurement via Biproducts}

Abramksy and Coecke's categorical model subscribes to the many-worlds interpretation of quantum
theory. That is, the act of measuring a qubit results in two possible branches---one in 
which the qubit was measured $\Ket{0}$ and one in which the qubit was measured $\Ket{1}$.
As a morphism this measurement will be modeled as
\[ \meas : \Qubit \rightarrow A \oplus A \]
where $\Qubit$ is the space of qubits and $A$ is some resulting state space.

\begin{definition}
    A category is said to have biproducts if for every $A_1,\ldots,A_n$
    there is an object $A_1 \oplus \cdots \oplus A_n$ along with
    projections $\pi_i$ and injections $\iota_i$ forming both a product and a coproduct.
    
    For $f : A \rightarrow C$ and $g : B \rightarrow C$, let $[f,g]$
    denote the morphism from $A \oplus B$ to $C$.

    For $f : A \rightarrow B$ and $g : A \rightarrow C$, let $\langle f,g\rangle$ denote
    the morphism from $A$ to $B \oplus C$.
\end{definition}

We can turn $\oplus$ into a bifunctor as follows: for $f_1:A_1 \rightarrow B_1$
and $f_2 : A_2 \rightarrow B_2$, define 
\begin{align*}
    f_1 \oplus f_2 &: A_1 \oplus A_2 \rightarrow B_1 \oplus B_2 \\
    f_1 \oplus f_2 &= [\iota_1 \circ f_1, \iota_2 \circ f_2].
\end{align*}

\subsubsection*{Example} In the category $\Hilb$, the 
biproduct is the direct sum.\footnote{Also the direct product; the two coincide
for finite-dimensional vector spaces.} The basis of $H_1 \oplus H_2$ is 
\[ \{\inl b_1 \mid b_1 \in B_1 \} \cup \{\inr b_2 \mid b_2 \in B_2 \} \]
where $B_1$ is a basis for $H_1$ and $B_2$ is a basis for $H_2$. In this case,
\begin{align*}
    [f,g](\inl a) &= f(a) \\
    [f,g](\inr a) &= g(a) \\
    \langle f,g \rangle (a) &= \inl f(a) + \inr g(a) \\
    f_1 \oplus f_2(\inl a_1) &= \inl f_1(a_1) \\
    f_1 \oplus f_2(\inr a_2) &= \inr f_2(a_2)
\end{align*}

\subsubsection*{Linearity of morphisms}

In any category with biproducts, we can define addition of morphisms as follows:
For $f, g : A \rightarrow B$, define $f + g : A \rightarrow B$ as
    \[ f + g : A
    \xrightarrow{\langle \id_A, \id_A \rangle}
    A \oplus A
    \xrightarrow{f \oplus g}
    B \oplus B  
    \xrightarrow{[\id_A,\id_A]}
    B
    \]
 \subsubsection*{Example} In $\Hilb$, addition of morphisms corresponds to addition of linear
transformations, as expected. That is,
\begin{align*}
    f+g(a)
    &= [\id_A,\id_A] \circ (f \oplus g) \circ \langle \id_A, \id_A \rangle (a) \\
    &= [\id_A,\id_A] \circ (f \oplus g) (\inl a + \inr a) \\
    &= [\id_A,\id_A] \circ (f \oplus g) (\inl a)
     + [\id_A,\id_A] \circ (f \oplus g) (\inr a) \\
    &= [\id_A,\id_A] (\inl f(a)) + [\id_A,\id_A] (\inr g(a)) \\
    &= f(a) + g(a)
\end{align*}

\subsubsection*{Distributivity}
We can define a right distributivity natural isomorphism (and similarly for left distributivity)
as
\[
    \tau_{A,B,C} = A \tensor (B \oplus C)
    \xrightarrow{\langle \id_A \tensor \pi_1, \id_A \tensor \pi_2\rangle}
    (A \tensor B) \oplus (A \tensor C)
\]
If the biproduct is thought of as parallel worlds, then the distributivity of $\tensor$
over $\oplus$ represents the propogation of quantum information $A$ to the two
quantum measurement branches $B$ and $C$. The information is not considered to be duplicated
because $B$ and $C$ never occur together. As we shall see in the teleportation
example, the distributivity really corresponds to classical transmission of data.
In particular, the speed of this transmission is limited by the speed of light.
