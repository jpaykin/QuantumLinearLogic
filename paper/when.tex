The categorical axiomatization presented in \Section{how} is one approach of many
to the mathematical formalization of quantum computation. In this section
we will briefly touch on some of the past work relating to quantum computation
and speculate on future directions.

\subsection{Related Work} 

One of the first generic models for quantum computation was the Quantum
Turing Machine (QTM), introduced by \cite{deutsch1985}. His machine allows for superpositions
of machine states, mimicking the state of qubits, and is equivalent to the
quantum circuits used by physicists. With the presence of a universal quantum turing
machine, all quantum algorithms can be expressed formally in terms of QTMs. 

\cite{selinger2006lambda,selinger2009quantum} describe a higher-order quantum lambda 
calculus which  takes an important step in introducing abstraction of first-class 
quantum circuits.
Their calculus is equipped with a linear type system, which uses subtyping of the
exponentials to distinguish classical from quantum values. The language introduces
a number of constants corresponding to $\operatorname{new}$ for pure state preparation,
$\operatorname{meas}$ for measurement, and constants corresponding to all unitary
transforamtions. The ability to introduce arbitrary unitary matrices is
important for expressing known quantum algorithms, but breaks the abstraction 
bewteen higher-order circuits and unitary matrices.

There are many attempts to develop usable quantum languages for future physical quantum
computers. \cite{omer2000quantum} developed the langauge QCL, which is a procedural
quantum language based on C. QCL can be used as a frontend for a quantum simulator,
although it is designed as a real language for quantum computers. Quipper is another
language, developed by \cite{green13quipper}, which takes a different approach, in that
it is a higher-order functional language embedded in Haskell. Compared to QCL,
the compilation of Quipper code to quantum circuits is formalized. 
Both languages allow for effectful computation, limited recursion, and automatic
generation of quantum code from classical functions.

On the semantic side of quantum langauges, there is extensive work in denotational 
models of quantum computation. \cite{coecke2005kindergarten} presents an
explicit connection between quantum circuit diagrams and category theoretic 
commuting diagrams in what he calls ``Kindergarten Quantum Mechanics.''
Along a similar vein, \cite{abramsky2006categorical} present quantum logic,
of which there is a large body of work, in the world of category theory with
the goal of making quantum algorithms more expressible. That work is a precursor
to Abramsky's model described in \Section{how}.

\subsection{Next steps}

The design of quantum programming languages is in some sense undirected,
due to the fact that physical quantum computers are still in their infancy.
\jp{???}


