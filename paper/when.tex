The categorical axiomatization presented in \cref{sec:how} is one approach of many
to the mathematical formalization of quantum computation. In this section
we will briefly touch on some of the past work relating to quantum computation
and speculate on future directions.

\subsection{Related Work} 

One of the first generic models for quantum computation was the Quantum
Turing Machine (QTM), introduced by \citet{deutsch1985}. His machine allows for superpositions
of machine states, mimicking the state of qubits, and is equivalent to the
quantum circuits used by physicists. With the presence of a universal quantum turing
machine, all quantum algorithms can be expressed formally in terms of QTMs. 

\Citet{selinger2006lambda,selinger2009quantum} describe a higher-order quantum lambda 
calculus which  takes an important step in introducing abstraction of first-class 
quantum circuits.
Their calculus is equipped with a linear type system, which uses subtyping of the
exponentials to distinguish Newtonian from quantum values. The language introduces
a number of constants: $\texttt{new}$ for pure state preparation,
$\texttt{meas}$ for measurement, and constants corresponding to all unitary
transforamtions. The ability to introduce arbitrary unitary matrices is
important for expressing known quantum algorithms, but breaks the abstraction 
bewteen higher-order circuits and unitary matrices. \citet{abramsky2009categorical}
address this by expressing teleportation and other algorithms \emph{purely} in
category-theoretic terms, thereby eliminating the need for arbitrary unary matrices.

There are many attempts to develop usable quantum languages for future physical quantum
computers. \citet{omer2000quantum} developed the langauge QCL, which is a procedural
quantum language based on C\@. QCL can be used as a frontend for a quantum simulator,
although it is designed as a real language for quantum computers. Quipper,
developed by \citet{green13quipper}, 
it is a higher-order functional language embedded in Haskell. Compared to QCL,
Quipper formalizes the compilation of code to quantum circuits.
Both languages allow for effectful computation, limited recursion, and automatic
generation of quantum code from classical functions.

On the semantic side of quantum langauges, there is extensive work in denotational 
models of quantum computation. \citet{coecke2005kindergarten} presents an
explicit connection between quantum circuit diagrams and category theoretic 
commuting diagrams in what he calls ``Kindergarten Quantum Mechanics.''
Along a similar vein, \citet{abramsky2006categorical} present quantum logic,
of which there is a large body of work, in the world of category theory with
the goal of making quantum algorithms more expressible. That work is a precursor
to Abramsky's model described in \cref{sec:how}.

\subsection{Next steps}

The way forward for quantum computing depends in some sense on engineering.
Physical quantum computers are still in their infancy, and the different
engineering design choices that are made will influence the computing design
choices.  The quest for feasable quantum computers is racing ahead day by day.
Researchers are able to factor larger and larger numbers using Shore's
algorithm, culminating in a factorization of $143$ by \citet{xu2012quantum}.  In
2012, \citet{ma2012quantum} succeeded in a large-scale teleportation of qubits,
across a distance of 143~kilometers.  There is even a commercial product, such
as that produced by D-Wave Systems \citep{johnson2011quantum}, which purports to
utilizes quantum superposition for specialized computation (although there is as
yet a great deal of skepticism as to whether D-Wave's machine truly takes
advantage of quantum mechanics, or is simply equivalent to a Newtonian
computer \citep{aaronson2014TIME}).

Yet no matter how powerful quantum computers get, understanding how their
computational behavior really works will be essential to actually using them.
The computer revolution would not have come about without the high-level
languages and logics that computer science brought to bear upon the silicon
machines.  So too will we need to learn how to think about what our quantum
computers are doing in order to tame the powerful new machines we will
eventually have at our disposal.
