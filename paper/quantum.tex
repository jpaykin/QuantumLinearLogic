\documentclass[12pt]{amsart}

\newif\ifdraft
\drafttrue

\usepackage[T1]{fontenc}
\usepackage[utf8]{inputenc}

\usepackage{lmodern}
\usepackage{microtype}
\usepackage{fullpage}

%    Q-circuit version 2
%    Copyright (C) 2004  Steve Flammia & Bryan Eastin
%    Last modified on: 9/16/2011
%
%    This program is free software; you can redistribute it and/or modify
%    it under the terms of the GNU General Public License as published by
%    the Free Software Foundation; either version 2 of the License, or
%    (at your option) any later version.
%
%    This program is distributed in the hope that it will be useful,
%    but WITHOUT ANY WARRANTY; without even the implied warranty of
%    MERCHANTABILITY or FITNESS FOR A PARTICULAR PURPOSE.  See the
%    GNU General Public License for more details.
%
%    You should have received a copy of the GNU General Public License
%    along with this program; if not, write to the Free Software
%    Foundation, Inc., 59 Temple Place, Suite 330, Boston, MA  02111-1307  USA

% Thanks to the Xy-pic guys, Kristoffer H Rose, Ross Moore, and Daniel Müllner,
% for their help in making Qcircuit work with Xy-pic version 3.8.  
% Thanks also to Dave Clader, Andrew Childs, Rafael Possignolo, Tyson Williams,
% Sergio Boixo, Cris Moore, Jonas Anderson, and Stephan Mertens for helping us test 
% and/or develop the new version.

\usepackage{xy}
\xyoption{matrix}
\xyoption{frame}
\xyoption{arrow}
\xyoption{arc}

\usepackage{ifpdf}
\ifpdf
\else
\PackageWarningNoLine{Qcircuit}{Qcircuit is loading in Postscript mode.  The Xy-pic options ps and dvips will be loaded.  If you wish to use other Postscript drivers for Xy-pic, you must modify the code in Qcircuit.tex}
%    The following options load the drivers most commonly required to
%    get proper Postscript output from Xy-pic.  Should these fail to work,
%    try replacing the following two lines with some of the other options
%    given in the Xy-pic reference manual.
\xyoption{ps}
\xyoption{dvips}
\fi

% The following resets Xy-pic matrix alignment to the pre-3.8 default, as
% required by Qcircuit.
\entrymodifiers={!C\entrybox}

\newcommand{\bra}[1]{{\left\langle{#1}\right\vert}}
\newcommand{\ket}[1]{{\left\vert{#1}\right\rangle}}
    % Defines Dirac notation. %7/5/07 added extra braces so that the commands will work in subscripts.
\newcommand{\qw}[1][-1]{\ar @{-} [0,#1]}
    % Defines a wire that connects horizontally.  By default it connects to the object on the left of the current object.
    % WARNING: Wire commands must appear after the gate in any given entry.
\newcommand{\qwx}[1][-1]{\ar @{-} [#1,0]}
    % Defines a wire that connects vertically.  By default it connects to the object above the current object.
    % WARNING: Wire commands must appear after the gate in any given entry.
\newcommand{\cw}[1][-1]{\ar @{=} [0,#1]}
    % Defines a classical wire that connects horizontally.  By default it connects to the object on the left of the current object.
    % WARNING: Wire commands must appear after the gate in any given entry.
\newcommand{\cwx}[1][-1]{\ar @{=} [#1,0]}
    % Defines a classical wire that connects vertically.  By default it connects to the object above the current object.
    % WARNING: Wire commands must appear after the gate in any given entry.
\newcommand{\gate}[1]{*+<.6em>{#1} \POS ="i","i"+UR;"i"+UL **\dir{-};"i"+DL **\dir{-};"i"+DR **\dir{-};"i"+UR **\dir{-},"i" \qw}
    % Boxes the argument, making a gate.
\newcommand{\meter}{*=<1.8em,1.4em>{\xy ="j","j"-<.778em,.322em>;{"j"+<.778em,-.322em> \ellipse ur,_{}},"j"-<0em,.4em>;p+<.5em,.9em> **\dir{-},"j"+<2.2em,2.2em>*{},"j"-<2.2em,2.2em>*{} \endxy} \POS ="i","i"+UR;"i"+UL **\dir{-};"i"+DL **\dir{-};"i"+DR **\dir{-};"i"+UR **\dir{-},"i" \qw}
    % Inserts a measurement meter.
    % In case you're wondering, the constants .778em and .322em specify
    % one quarter of a circle with radius 1.1em.
    % The points added at + and - <2.2em,2.2em> are there to strech the
    % canvas, ensuring that the size is unaffected by erratic spacing issues
    % with the arc.
\newcommand{\measure}[1]{*+[F-:<.9em>]{#1} \qw}
    % Inserts a measurement bubble with user defined text.
\newcommand{\measuretab}[1]{*{\xy*+<.6em>{#1}="e";"e"+UL;"e"+UR **\dir{-};"e"+DR **\dir{-};"e"+DL **\dir{-};"e"+LC-<.5em,0em> **\dir{-};"e"+UL **\dir{-} \endxy} \qw}
    % Inserts a measurement tab with user defined text.
\newcommand{\measureD}[1]{*{\xy*+=<0em,.1em>{#1}="e";"e"+UR+<0em,.25em>;"e"+UL+<-.5em,.25em> **\dir{-};"e"+DL+<-.5em,-.25em> **\dir{-};"e"+DR+<0em,-.25em> **\dir{-};{"e"+UR+<0em,.25em>\ellipse^{}};"e"+C:,+(0,1)*{} \endxy} \qw}
    % Inserts a D-shaped measurement gate with user defined text.
\newcommand{\multimeasure}[2]{*+<1em,.9em>{\hphantom{#2}} \qw \POS[0,0].[#1,0];p !C *{#2},p \drop\frm<.9em>{-}}
    % Draws a multiple qubit measurement bubble starting at the current position and spanning #1 additional gates below.
    % #2 gives the label for the gate.
    % You must use an argument of the same width as #2 in \ghost for the wires to connect properly on the lower lines.
\newcommand{\multimeasureD}[2]{*+<1em,.9em>{\hphantom{#2}} \POS [0,0]="i",[0,0].[#1,0]="e",!C *{#2},"e"+UR-<.8em,0em>;"e"+UL **\dir{-};"e"+DL **\dir{-};"e"+DR+<-.8em,0em> **\dir{-};{"e"+DR+<0em,.8em>\ellipse^{}};"e"+UR+<0em,-.8em> **\dir{-};{"e"+UR-<.8em,0em>\ellipse^{}},"i" \qw}
    % Draws a multiple qubit D-shaped measurement gate starting at the current position and spanning #1 additional gates below.
    % #2 gives the label for the gate.
    % You must use an argument of the same width as #2 in \ghost for the wires to connect properly on the lower lines.
\newcommand{\control}{*!<0em,.025em>-=-<.2em>{\bullet}}
    % Inserts an unconnected control.
\newcommand{\controlo}{*+<.01em>{\xy -<.095em>*\xycircle<.19em>{} \endxy}}
    % Inserts a unconnected control-on-0.
\newcommand{\ctrl}[1]{\control \qwx[#1] \qw}
    % Inserts a control and connects it to the object #1 wires below.
\newcommand{\ctrlo}[1]{\controlo \qwx[#1] \qw}
    % Inserts a control-on-0 and connects it to the object #1 wires below.
\newcommand{\targ}{*+<.02em,.02em>{\xy ="i","i"-<.39em,0em>;"i"+<.39em,0em> **\dir{-}, "i"-<0em,.39em>;"i"+<0em,.39em> **\dir{-},"i"*\xycircle<.4em>{} \endxy} \qw}
    % Inserts a CNOT target.
\newcommand{\qswap}{*=<0em>{\times} \qw}
    % Inserts half a swap gate.
    % Must be connected to the other swap with \qwx.
\newcommand{\multigate}[2]{*+<1em,.9em>{\hphantom{#2}} \POS [0,0]="i",[0,0].[#1,0]="e",!C *{#2},"e"+UR;"e"+UL **\dir{-};"e"+DL **\dir{-};"e"+DR **\dir{-};"e"+UR **\dir{-},"i" \qw}
    % Draws a multiple qubit gate starting at the current position and spanning #1 additional gates below.
    % #2 gives the label for the gate.
    % You must use an argument of the same width as #2 in \ghost for the wires to connect properly on the lower lines.
\newcommand{\ghost}[1]{*+<1em,.9em>{\hphantom{#1}} \qw}
    % Leaves space for \multigate on wires other than the one on which \multigate appears.  Without this command wires will cross your gate.
    % #1 should match the second argument in the corresponding \multigate.
\newcommand{\push}[1]{*{#1}}
    % Inserts #1, overriding the default that causes entries to have zero size.  This command takes the place of a gate.
    % Like a gate, it must precede any wire commands.
    % \push is useful for forcing columns apart.
    % NOTE: It might be useful to know that a gate is about 1.3 times the height of its contents.  I.e. \gate{M} is 1.3em tall.
    % WARNING: \push must appear before any wire commands and may not appear in an entry with a gate or label.
\newcommand{\gategroup}[6]{\POS"#1,#2"."#3,#2"."#1,#4"."#3,#4"!C*+<#5>\frm{#6}}
    % Constructs a box or bracket enclosing the square block spanning rows #1-#3 and columns=#2-#4.
    % The block is given a margin #5/2, so #5 should be a valid length.
    % #6 can take the following arguments -- or . or _\} or ^\} or \{ or \} or _) or ^) or ( or ) where the first two options yield dashed and
    % dotted boxes respectively, and the last eight options yield bottom, top, left, and right braces of the curly or normal variety.  See the Xy-pic reference manual for more options.
    % \gategroup can appear at the end of any gate entry, but it's good form to pick either the last entry or one of the corner gates.
    % BUG: \gategroup uses the four corner gates to determine the size of the bounding box.  Other gates may stick out of that box.  See \prop.

\newcommand{\rstick}[1]{*!L!<-.5em,0em>=<0em>{#1}}
    % Centers the left side of #1 in the cell.  Intended for lining up wire labels.  Note that non-gates have default size zero.
\newcommand{\lstick}[1]{*!R!<.5em,0em>=<0em>{#1}}
    % Centers the right side of #1 in the cell.  Intended for lining up wire labels.  Note that non-gates have default size zero.
\newcommand{\ustick}[1]{*!D!<0em,-.5em>=<0em>{#1}}
    % Centers the bottom of #1 in the cell.  Intended for lining up wire labels.  Note that non-gates have default size zero.
\newcommand{\dstick}[1]{*!U!<0em,.5em>=<0em>{#1}}
    % Centers the top of #1 in the cell.  Intended for lining up wire labels.  Note that non-gates have default size zero.
\newcommand{\Qcircuit}{\xymatrix @*=<0em>}
    % Defines \Qcircuit as an \xymatrix with entries of default size 0em.
\newcommand{\link}[2]{\ar @{-} [#1,#2]}
    % Draws a wire or connecting line to the element #1 rows down and #2 columns forward.
\newcommand{\pureghost}[1]{*+<1em,.9em>{\hphantom{#1}}}
    % Same as \ghost except it omits the wire leading to the left. 

\let\bra\relax
\let\ket\relax
\usepackage{braket}

\usepackage[svgnames]{xcolor}

\newcommand{\asz}[1]{\ifdraft\textcolor{violet}{[ASZ: #1]}\fi}

\newcommand*{\bit}{\mathbb{B}}
\newcommand*{\qubit}{\mathbb{Q}}

\newcommand*{\term}[1]{\emph{#1}}

\newcommand*{\latin}[1]{\textit{#1}}
\newcommand*{\IE}{\latin{i.e.}}

\usepackage[section]{placeins} % Floats should stay in their section

\usepackage[ colorlinks
           , pagebackref
           , linktoc = all
           , pdfusetitle ]{hyperref}
\usepackage[all]{hypcap}
\renewcommand*{\backrefalt}[4]{%
  \ifcase #1 %
    No citations.%
  \or
    Cited on page #2.%
  \else
    Cited on pages #2.%
  \fi
}
\hypersetup{ linkcolor = DarkBlue
           , citecolor = DarkGreen
           , urlcolor  = DarkGoldenrod
           , filecolor = Orange
           , menucolor = DarkRed
           , runcolor  = OrangeRed }

% Cleveref (must be loaded late---after hyperref, no less!)
\usepackage[noabbrev,capitalize,nameinlink]{cleveref}
\newcommand{\creflastconjunction}{, and~}

\begin{document}

\section{Quantum computation}\label{sec:quantum-computation}

When we change from classical\footnote{By \term{classical} here, we do not mean
``affirming the law of the excluded middle'', as in classical logic
 s.\ intuitionistic logic; we mean ``following the laws of traditional Newtonian
physics'', as in classical physics vs.\ quantum mechanics.  This naming
collision is unfortunate; however, in this paper, we do not discuss
classical-logic--inspired computation, and so the term will always refer to the
physical concept.\asz{This probably belongs earlier.}} to quantum computation,
our primitive data type changes from the classical bit to the quantum
\term{qubit} (short for \emph{qu}antum \emph{bit}).  At first glance, qubits
behave much like bits: just as measuring a bit produces one of the two values
$0$ or $1$, measuring a qubit produces one of the two values $\ket{0}$ or
$\ket{1}$.  However, qubits behave very differently when \emph{not} being
measured.  This is because qubits exhibit two properties that classical bits do
not:\asz{\term{superposition} and \term{entanglement}. -- does this belong
here?}
\begin{enumerate}
  \item \term{Superposition}: A qubit, prior to measurement, can be a
    \emph{combination} of $\ket{0}$ and $\ket{1}$.

  \item \term{Entanglement}: Two (or more) qubits can be related in such a way
    that measuring the value of the first provides information about the value
    of the second.\asz{This explanation doesn't make clear why entanglement is
    magic, and I don't really know how to explain this without the math.  I'll
    come back to this.}
\end{enumerate}

To explore the computational properties these give rise to, we will explore four
different algorithms that require more of these features:
\begin{enumerate}
  \item Bitwise negation (\cref{sec:bitwise-negation-hl}), which is classical;
  \item Simulating a fair coin (\cref{sec:fair-coin-hl}), which requires
    superposition;
  \item Agreeing on the result of a fair coin (\cref{sec:agree-coin-hl}), which
    requires entanglement; and
  \item Quantum teleportation (\cref{sec:teleportation-hl}), which makes
    nontrivial use of both superposition and entanglement.
\end{enumerate}

\subsection{Bitwise negation}\label{sec:bitwise-negation-hl}
Given a qubit (or, equally well, a classical bit), we can send it through a not
gate, $X$, which changes $\ket{0}$ into $\ket{1}$ and $\ket{1}$ into $\ket{0}$;
the quantum circuit diagram\asz{We need to explain how to read these.} for this
is in \cref{fig:bitwise-negation}, demonstrating the bitwise negation of
$\ket{0}$.

\begin{figure}
  \centerline{\Qcircuit{
    \lstick{\ket{0}} & \gate{X} & \qw & \lstick{\ket{1}}
  }}
  \caption{Bitwise negation (of $\ket{0}$)}\label{fig:bitwise-negation}
\end{figure}

\subsection{Simulating a fair coin}\label{sec:fair-coin-hl}
As we said above, a qubit may be in a \term{superposition} of $\ket{0}$ and
$\ket{1}$; such states are called \term{mixed states}.  The simplest
superposition is one which is ``$50\%$ $\ket{0}$ and $50\%$ $\ket{1}$''; to
produce this, we use the quantum \term{Hadamard gate} $H$, which takes both
$\ket{0}$ and $\ket{1}$ to such mixed states.\footnote{It takes them to two
\emph{different} such states -- see \asz{some later section}.}  Then, when this
mixed state is measured, the measurement is \emph{probabilistic}: it has a 50\%
chance of being $\ket{0}$ and a 50\% chance of being $\ket{1}$.


\subsection{Agreeing on the result of a fair coin}\label{sec:agree-coin-hl}
\asz{Maybe we only need this (and the preceding two) at the concrete level.}

\subsection{Quantum teleportation}\label{sec:teleportation-hl}
Quantum teleportation allows us to do something very special: given some shared
state between two parties, sending only two classical bits will allow us to
transmit an entire qubit (\IE, two complex numbers).  Suppose Alice wishes to
send a single qubit to Bob.  Then the algorithm proceeds as follows.

\begin{enumerate}
  \item \emph{Preparation.} Alice and Bob meet, and entangle two qubits, $a$ and
    $b$, so that they are guaranteed to be equal but which qubit they will be
    measured as is arbitrary.  Then they leave; Alice takes qubit $a$ with her,
    and Bob takes qubit $b$ with him.

  \item \emph{Composition.}  Sometime later, Alice places her message qubit $m$
    into some arbitrary state $\ket{\psi}$.\asz{I \emph{think} kets aren't only
    used for pure states.}

  \item \emph{Measurement.}  Alice now wishes to send $\phi$ to Bob.  In order
    to do this, Alice entangles $m$ and $a$ via a rotation\asz{check this
    terminology} of the two: all pure states ($\ket{00}$, $\ket{01}$,
    $\ket{10}$, and $\ket{11}$) are transformed into superpositions of two
    different pure states.  Since $a$ was already entangled with Bob's qubit
    $b$, this means that $m$ is entangled with $b$ as well.\asz{Should we split
    this next off into a separate step?}  Alice then measures $a$ and $b$,
    getting the two classical bits $m_M$ and $a_M$.

  \item \emph{Classical communication.}  Alice sends Bob the classical message
    $(m_M,a_M)$, via any arbitrary classical channel.

  \item \asz{I decided this isn't clear without the details.  Maybe I'll try
    explaining this to random victims -- er, friends -- and see how clear I can
    make it.}
\end{enumerate}

\end{document}

% ASZ: Notes that I want to use later:

% The set of qubits is ℚ; this is ambiguous, so we explain it.
, which we denote $\qubit$.\footnote{We do not have much
call to refer to rational numbers, so we use $\qubit$ to refer exclusively to
qubits, and never to rational numbers.}

% The real definition of a simple mixed state
we write this \[ \frac{1}{\sqrt{2}}\ket{0} + \frac{1}{\sqrt{2}}\ket{1} =
\frac{\ket{0} + \ket{1}}{\sqrt{2}}. \]

% ASZ: I use this to compile my LaTeX document; you can ignore it
(defun my-compile ()
  (interactive)
  (shell-command (concat
    "pdflatex -halt-on-error quantum.tex && "
    "osascript ~/misc/Skim-open-revert.scpt quantum.pdf &")))
(local-set-key (kbd "C-c C-c") #'my-compile)
(local-set-key (kbd "C-c C-l") #'my-compile)
