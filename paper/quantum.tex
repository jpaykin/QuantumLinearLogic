\documentclass[12pt]{amsart}

\newif\ifdraft
\drafttrue

\usepackage[T1]{fontenc}
\usepackage[utf8]{inputenc}

\usepackage{lmodern}
\usepackage{microtype}
\usepackage{fullpage}

\input{Qcircuit}
\let\bra\relax
\let\ket\relax
\usepackage{braket}

\usepackage[svgnames]{xcolor}

\newcommand{\asz}[1]{\ifdraft\textcolor{violet}{[ASZ: #1]}\fi}

\newcommand*{\bit}{\mathbb{B}}
\newcommand*{\qubit}{\mathbb{Q}}

\newcommand*{\term}[1]{\emph{#1}}

\newcommand*{\latin}[1]{\textit{#1}}
\newcommand*{\IE}{\latin{i.e.}}

\usepackage[section]{placeins} % Floats should stay in their section

\usepackage[ colorlinks
           , pagebackref
           , linktoc = all
           , pdfusetitle ]{hyperref}
\usepackage[all]{hypcap}
\renewcommand*{\backrefalt}[4]{%
  \ifcase #1 %
    No citations.%
  \or
    Cited on page #2.%
  \else
    Cited on pages #2.%
  \fi
}
\hypersetup{ linkcolor = DarkBlue
           , citecolor = DarkGreen
           , urlcolor  = DarkGoldenrod
           , filecolor = Orange
           , menucolor = DarkRed
           , runcolor  = OrangeRed }

% Cleveref (must be loaded late---after hyperref, no less!)
\usepackage[noabbrev,capitalize,nameinlink]{cleveref}
\newcommand{\creflastconjunction}{, and~}

\begin{document}

\section{Quantum computation}\label{sec:quantum-computation}

When we change from classical\footnote{By \term{classical} here, we do not mean
``affirming the law of the excluded middle'', as in classical logic
 s.\ intuitionistic logic; we mean ``following the laws of traditional Newtonian
physics'', as in classical physics vs.\ quantum mechanics.  This naming
collision is unfortunate; however, in this paper, we do not discuss
classical-logic--inspired computation, and so the term will always refer to the
physical concept.\asz{This probably belongs earlier.}} to quantum computation,
our primitive data type changes from the classical bit to the quantum
\term{qubit} (short for \emph{qu}antum \emph{bit}).  At first glance, qubits
behave much like bits: just as measuring a bit produces one of the two values
$0$ or $1$, measuring a qubit produces one of the two values $\ket{0}$ or
$\ket{1}$.  However, qubits behave very differently when \emph{not} being
measured.  This is because qubits exhibit two properties that classical bits do
not:\asz{\term{superposition} and \term{entanglement}. -- does this belong
here?}
\begin{enumerate}
  \item \term{Superposition}: A qubit, prior to measurement, can be a
    \emph{combination} of $\ket{0}$ and $\ket{1}$.

  \item \term{Entanglement}: Two (or more) qubits can be related in such a way
    that measuring the value of the first provides information about the value
    of the second.\asz{This explanation doesn't make clear why entanglement is
    magic, and I don't really know how to explain this without the math.  I'll
    come back to this.}
\end{enumerate}

To explore the computational properties these give rise to, we will explore four
different algorithms that require more of these features:
\begin{enumerate}
  \item Bitwise negation (\cref{sec:bitwise-negation-hl}), which is classical;
  \item Simulating a fair coin (\cref{sec:fair-coin-hl}), which requires
    superposition;
  \item Agreeing on the result of a fair coin (\cref{sec:agree-coin-hl}), which
    requires entanglement; and
  \item Quantum teleportation (\cref{sec:teleportation-hl}), which makes
    nontrivial use of both superposition and entanglement.
\end{enumerate}

\subsection{Bitwise negation}\label{sec:bitwise-negation-hl}
Given a qubit (or, equally well, a classical bit), we can send it through a not
gate, $X$, which changes $\ket{0}$ into $\ket{1}$ and $\ket{1}$ into $\ket{0}$;
the quantum circuit diagram\asz{We need to explain how to read these.} for this
is in \cref{fig:bitwise-negation}, demonstrating the bitwise negation of
$\ket{0}$.

\begin{figure}
  \centerline{\Qcircuit{
    \lstick{\ket{0}} & \gate{X} & \qw & \lstick{\ket{1}}
  }}
  \caption{Bitwise negation (of $\ket{0}$)}\label{fig:bitwise-negation}
\end{figure}

\subsection{Simulating a fair coin}\label{sec:fair-coin-hl}
As we said above, a qubit may be in a \term{superposition} of $\ket{0}$ and
$\ket{1}$; such states are called \term{mixed states}.  The simplest
superposition is one which is ``$50\%$ $\ket{0}$ and $50\%$ $\ket{1}$''; to
produce this, we use the quantum \term{Hadamard gate} $H$, which takes both
$\ket{0}$ and $\ket{1}$ to such mixed states.\footnote{It takes them to two
\emph{different} such states -- see \asz{some later section}.}  Then, when this
mixed state is measured, the measurement is \emph{probabilistic}: it has a 50\%
chance of being $\ket{0}$ and a 50\% chance of being $\ket{1}$.


\subsection{Agreeing on the result of a fair coin}\label{sec:agree-coin-hl}
\asz{Maybe we only need this (and the preceding two) at the concrete level.}

\subsection{Quantum teleportation}\label{sec:teleportation-hl}
Quantum teleportation allows us to do something very special: given some shared
state between two parties, sending only two classical bits will allow us to
transmit an entire qubit (\IE, two complex numbers).  Suppose Alice wishes to
send a single qubit to Bob.  Then the algorithm proceeds as follows.

\begin{enumerate}
  \item \emph{Preparation.} Alice and Bob meet, and entangle two qubits, $a$ and
    $b$, so that they are guaranteed to be equal but which qubit they will be
    measured as is arbitrary.  Then they leave; Alice takes qubit $a$ with her,
    and Bob takes qubit $b$ with him.

  \item \emph{Composition.}  Sometime later, Alice places her message qubit $m$
    into some arbitrary state $\ket{\psi}$.\asz{I \emph{think} kets aren't only
    used for pure states.}

  \item \emph{Measurement.}  Alice now wishes to send $\phi$ to Bob.  In order
    to do this, Alice entangles $m$ and $a$ via a rotation\asz{check this
    terminology} of the two: all pure states ($\ket{00}$, $\ket{01}$,
    $\ket{10}$, and $\ket{11}$) are transformed into superpositions of two
    different pure states.  Since $a$ was already entangled with Bob's qubit
    $b$, this means that $m$ is entangled with $b$ as well.\asz{Should we split
    this next off into a separate step?}  Alice then measures $a$ and $b$,
    getting the two classical bits $m_M$ and $a_M$.

  \item \emph{Classical communication.}  Alice sends Bob the classical message
    $(m_M,a_M)$, via any arbitrary classical channel.

  \item \asz{I decided this isn't clear without the details.  Maybe I'll try
    explaining this to random victims -- er, friends -- and see how clear I can
    make it.}
\end{enumerate}

\end{document}

% ASZ: Notes that I want to use later:

% The set of qubits is ℚ; this is ambiguous, so we explain it.
, which we denote $\qubit$.\footnote{We do not have much
call to refer to rational numbers, so we use $\qubit$ to refer exclusively to
qubits, and never to rational numbers.}

% The real definition of a simple mixed state
we write this \[ \frac{1}{\sqrt{2}}\ket{0} + \frac{1}{\sqrt{2}}\ket{1} =
\frac{\ket{0} + \ket{1}}{\sqrt{2}}. \]

% ASZ: I use this to compile my LaTeX document; you can ignore it
(defun my-compile ()
  (interactive)
  (shell-command (concat
    "pdflatex -halt-on-error quantum.tex && "
    "osascript ~/misc/Skim-open-revert.scpt quantum.pdf &")))
(local-set-key (kbd "C-c C-c") #'my-compile)
(local-set-key (kbd "C-c C-l") #'my-compile)
