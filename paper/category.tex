
Quantum computation is traditionally presented in the physicist's language
of quantum mechanics---that is, superposition, measurement and wires. 
As quantum computation moves into the realm of higher-order and thus more principled
programming languages, it becomes clear that computer scientists need a way to 
connect the language of quantum mechanics with the language of programming. 
For example, Quipper\cite{green13quipper} uses higher-order features including
functions and monads, but is essentially still a circuit description language.
In order to more cohesively understand the semantics of quantum computation,
Abramsky and Coecke\cite{abramsky2009categorical} describe a thorough 
axiomatization of the major properties of quantum computation and logic
as a categorical model. In this work we will use their model as a guide
for the semantic underpinnings of a number of simple quantum algorithms.


\subsection{Symmetric Monoidal Categories and the No-Cloning Theorem}

A symmetric monoidal category is a category $C$, along with a bifunctor $\tensor$
and a unit $1$, and with natural isormorphisms:
\begin{align*}
    \text{Associativity:}&\qquad \alpha_{A,B,C} : A \tensor (B \tensor C) \rightarrow (A \tensor B) \tensor C \\
    \text{Left and right units:}&\qquad \lambda_A : 1 \tensor A \rightarrow A \text{ and } \rho_A : A \tensor 1 \rightarrow A \\
    \text{Symmetry:}&\qquad \sigma_{A,B} : A \tensor B \rightarrow B \tensor A \\
\end{align*}
In addition, the monoidal structure must commute with the isomorphisms in various ways \jp{Cite}

The no-cloning theorem says there is no way to produce two copies of a single arbitrary quantum state.

\subsection{Compact Closed Categories: Names and Conames}

\subsection{Strong Compact Closure and the Inner Product}

\subsection{Measurement via Biproducts}
