
Quantum computation is traditionally presented in the physicist's language
of quantum mechanics---that is, superposition, measurement and wires. 
As quantum computation moves into the realm of higher-order and thus more principled
programming languages, it becomes clear that computer scientists need a way to 
connect the language of quantum mechanics with the language of programming. 
For example, Quipper\cite{green13quipper} uses higher-order features including
functions and monads, but is essentially still a circuit description language.
In order to more cohesively understand the semantics of quantum computation,
Abramsky and Coecke\cite{abramsky2009categorical} describe a thorough 
axiomatization of the major properties of quantum computation and logic
as a categorical model. In this work we will use their model as a guide
for the semantic underpinnings of a number of simple quantum algorithms.


\subsection{Symmetric Monoidal Categories and the No-Cloning Theorem}

\begin{definition}
A symmetric monoidal category is a category $C$, along with a bifunctor $\tensor$
and a unit $1$, and with natural isormorphisms:
\begin{align*}
    \text{Associativity:}&\qquad 
    \alpha_{A,B,C} : A \tensor (B \tensor C) \rightarrow (A \tensor B) \tensor C \\
    \text{Left and right units:}&\qquad 
    \lambda_A : 1 \tensor A \rightarrow A \text{ and } \rho_A : A \tensor 1 \rightarrow A \\
    \text{Symmetry:}&\qquad 
    \sigma_{A,B} : A \tensor B \rightarrow B \tensor A \\
\end{align*}
In addition, the monoidal structure must commute with the isomorphisms in 
various ways \jp{Cite}.
\end{definition}

\subsubsection*{Example}
The category of hilbert spaces is symmetric monoidal, where the tensor product
is the regular tensor product on vector spaces. A few properties of this tensor
product are:
\begin{enumerate}
    \item If $B_1$ is a basis of $H_1$ and $B_2$ is a basis of $H_2$
    then $B_1 \times B_2$ is a basis for $H_1 \tensor H_2$.
    \item For arbitrary $u_1 \in H_1$ and $u_2 \in H_2$, 
    $u_1 \tensor u_2$ is the equivalence class in $H_1 \tensor H_2$ induced
    by bilinearity:
    \begin{align*}
        \alpha(u_1 \tensor u_2) &= (\alpha u_1) \tensor u_2 \\
        \alpha(u_1 \tensor u_2) &= u_1 \tensor (\alpha u_2) \\
        (u_1 \tensor v) + (u_2 \tensor v) &= (u_1 + u_2) \tensor v \\
        (u \tensor v_1) + (u \tensor v_2) &= u \tensor (v_1 + v_2)
    \end{align*}
\end{enumerate}
The unit of the tensor is the underlying field, with the basis $B = \{ \unit \}$.


Along with monoidal categories comes a range of categorical structures tailored
to the monoidal structure: functors, natural transformations, and adjunctions.

\begin{definition}
A symmetric monoidal functor is a functor $F : C \Rightarrow D$ between two symmetric
monoidal categories, along with a morphism
\[ m_1^F : 1^D \rightarrow F 1^C \]
in C (or D?) and a natural transformation
\[ m_{A,B}^F : F A \tensor F B \rightarrow F(A \tensor B). \]
Again, these morphisms must commute with the monoidal structures in $C$ and $D$
in certain ways not outlined here. 
\end{definition}

A monoidal natural transformation between two symmetric monoidal functors
is a natural transformation which commutes with the monoidal components $m_1$ and $m_{A,B}$.
A monoidal adjunction is simply an adjunction where the unit and counit
are monoidal natural transformations.


The no-cloning theorem says there is no way to produce two copies of a single 
arbitrary quantum state. To be more precise, the no-cloning theorem states that
there is no unitary transformation which can duplicate every possible quantum state.

\begin{theorem}[No-Cloning (cite)]
    There is no unitary transformation acting on qubits as follows:
    \[ U(\ket{0}\tensor\ket{\varphi}) = \ket{\varphi}\tensor\ket{\varphi}.\footnotemark \]
    \footnotetext{Unitary transformations are, in traditional quantum mechanics,
    always square matrices, so in this case we ``pad'' the transformation with an
    arbitrary \jp{???} state.}
\end{theorem}
\begin{proof}
    Suppose there were such a transformation. We may expand an arbitrary $\ket{\varphi}$
    as $\varphi = \alpha\ket{0} + \beta\ket{1}$. By the definition of $U$, we know
    \begin{align*} 
        U(\ket{0} \tensor \ket{\varphi})
        &= \ket{\varphi} \tensor \ket{\varphi} \\
        &= (\alpha\ket{0} + \beta\ket{1}) \tensor (\alpha\ket{0} + \beta\ket{1}) \\
        &= \alpha^2 \ket{00} + \alpha\beta\ket{01} + \alpha\beta\ket{10} + \beta^2\ket{11}.
    \end{align*}
    However, since all unitary transformations are necessarily linear, we can also compute
    \begin{align*}
        U(\ket{0} \tensor \ket{\varphi})
        &= \alpha U(\ket{0} \tensor \ket{0}) + \beta U(\ket{0} \tensor \ket{1}) \\
        &= \alpha \ket{00} + \beta \ket{11}.
    \end{align*}
    These two expansions are unequal, thus we have reached a contradiction. 
\end{proof}

So what does this have to do with monoidal categories? 
The use of linearity in the proof of the no-cloning theorem is exactly the overlap.
Without additional structure, monoidal categories do not allow arbitrary
morphisms $A \rightarrow A \tensor A$. Abramsky and Coecke\cite{abramsky2009categorical}
prove the corresponding property holds for their entire categorical model,
but for now it suffices to say that the structure added in the remainder
of this section does not allow arbitrary duplication on the monoidal structure.

\subsection{Compact Closed Categories: Names and Conames}

\begin{definition}
A symmetric monoidal category is \emph{compact closed} if every object $A$ is
equipped with a dual object $A^*$, along with morphisms
\[ \eta_A : 1 \rightarrow A^* \tensor A \qquad\text{and}\qquad
   \varepsilon_A : A \tensor A^* \rightarrow 1 \]
provided
\[
    A \xrightarrow{\rho_A^{-1}} A \tensor 1
    \xrightarrow{\id_A \tensor \eta_A} A\tensor(A^* \tensor A)
    \xrightarrow{\alpha_{A,A^*,A}} (A \tensor A^*)\tensor A
    \xrightarrow{\varepsilon_A \tensor \id_A} 1 \tensor A
    \xrightarrow{\lambda_A} A
\]
is the identity morphism on $A$, and
\[
    A \xrightarrow{\lambda_{A^*}^{-1}} 1 \tensor A^*
    \xrightarrow{\eta_A \tensor \id_{A^*}} (A^* \tensor A) \tensor A^*
    \xrightarrow{\alpha_{A^*,A,A^*}^{-1}} A^* \tensor (A \tensor A^*) 
    \xrightarrow{\id_{A^*} \tensor \varepsilon_A} A^* \tensor 1
    \xrightarrow{\rho_{A^*}} A^*
\]
is the identity morphism on $A^*$.
\end{definition}

\subsubsection*{Example} In the category of (finite-dimensional) hilbert spaces, the dual
of a hilbert space $H$ is space of continuous linear maps from $H$
to the base field. For a basis element $b$, let $\delta_b$
be the linear map defined on basis elements as follows:
\[ \delta_b(b') = \begin{cases}
    1 &b'=b \\
    0 &b' \neq b
\end{cases}\]
For a basis $B$ of $H$, the set $\overline{B} = \{\delta_b \mid b \in B\}$
is a basis for $H^*$. Then we can define $\eta$ and $\varepsilon$ as follows:
\begin{align*}
    \eta_H(\unit) &= \sum_{b \in \basis(H)} \delta_b \tensor b \\
    \varepsilon_H(b \tensor \delta_{b'}) &= 
        \begin{cases}   
            \unit &b = b' \\
            \zero &b \neq b'
        \end{cases}
\end{align*}

The tensor can be thought of as a new kind of composition. Traditional
sequential composition is given by the categorical composition of two morphisms:
\tikzstyle{boxmode} = [draw,minimum height=2em, minimum width=2em]
\begin{center} \begin{tikzpicture}
    \node[boxmode] (f) at (3,0) {$f$};
    \node[boxmode] (g) at (6,0) {$g$};

    \draw (0,0) -- (f);
    \draw (f) -- (g);
    \draw (g) -- (9,0);
\end{tikzpicture} \end{center}
whereas parallel composition is the result of tensoring two morphisms together:
\begin{center} \begin{tikzpicture}
    \node[boxmode] (f) at (3,1) {$f$};
    \node[boxmode] (g) at (3,-0.5) {$g$};

    \draw (0,1) -- (f);
    \draw (f) -- (6,1);
    \draw (0,-0.5) -- (g);
    \draw (g) -- (6,-0.5);
\end{tikzpicture} \end{center}

From the unit $\eta$ and counit $\varepsilon$ we can induce a notion of currying
on arbitrary morphisms.

\begin{definition}
    Let $f : A \rightarrow B$ be a morphism in a compact closed category. The name of $f$,
    denoted $\lift f$, is
    \begin{align*}
        \lift f = 1 
         \xrightarrow{\eta_A} A^* \tensor A
         \xrightarrow{\id_{A^*} \tensor f} A^* \tensor B
    \end{align*}
    The coname of $f$, denoted $\llower f$, is
    \begin{align*}
        \llower f = A \tensor B^*
        \xrightarrow{f \tensor \id_{B^*}} B \tensor B^*
        \xrightarrow{\varepsilon_B} 1
    \end{align*}
\end{definition}
Graphically, we denote the name and coname of $f$ as
\tikzstyle{namenode} = [draw,shape border rotate=180, 
                        isosceles triangle,
                        node distance=2cm, minimum width=3em]
\tikzstyle{conamenode} = [draw,shape border rotate=0,
                        isosceles triangle,
                        node distance=2cm, minimum width=3em]

\begin{center} \begin{tikzpicture}
    \tikzset{invisible/.style={rectangle, minimum height=1.6em}}

    \node[namenode] (name) at (0,1) {$f$};
    \node[invisible] (destination) at (2,1) {};
    \path (name.north east) edge [above] node {$A^*$} (destination.north west);
    \path (name.south east) edge [above] node {$B$}   (destination.south west);

    \node[conamenode] (coname) at (0,-1) {$f$};
    \node[invisible] (source) at (-2,-1) {};
    \path (source.north east) edge [above] node {$A$} (coname.north west);
    \path (source.south east) edge [above] node {$B^*$} (coname.south west);
\end{tikzpicture} \end{center}
respectively.
In fact, it is also the case that every morphism $g : 1 \rightarrow A^* \tensor B$ can
be ``uncurried'', or that is turned into a morphism from $A$ to $B$, and vise versa
for morphisms $A \tensor B^* \rightarrow 1$. In the case of $\eta$ and $\varepsilon$, we have
\[ \eta_A = \lift{\id_A} \qquad\text{and}\qquad \varepsilon_A = \llower{\id_A}. \]

From the commutativity of the unit and counit, we can then induce certain properties of the
$\lift{-}$ and $\llower{-}$ operations, seen in \Figure{names}

\begin{figure}
\begin{tabular}{cccc}
Absorbtion
&
\begin{tikzpicture}
    \tikzset{invisible/.style={rectangle, minimum height=1.6em}}
    \node[namenode] (f) at (1,1) {$f$};
    \node[boxmode]  (g) at (3,0.66) {$g$};
    \node[invisible] (target) at (5,1) {};
    \path (f.north east) edge [above] node {$A^*$}  (target.north west);
    \path (f.south east) edge [above] node {$B$}    (g);
    \path (g)            edge [above] node {$C$}    (target.south west);
\end{tikzpicture}
&
$=$
&
\begin{tikzpicture}
    \tikzset{invisible/.style={rectangle, minimum height=2.8em}}
    \node[namenode] (gf) at (1,1) {$g \circ f$};
    \node[invisible] (target) at (5,1) {};
    \path (gf.north east) edge [above] node {$A^*$}  (target.north west);
    \path (gf.south east) edge [above] node {$C$}    (target.south west);
\end{tikzpicture}
\\ \\
Compositionality
&
\begin{tikzpicture}
    \tikzset{invisible/.style={rectangle, minimum height=1.6em}}
    \node[invisible] (source)   at (0,.67)    {};
    \node[namenode]  (g)        at (2,0)    {$g$};
    \node[conamenode](f)        at (4,.67)    {$f$};
    \node[invisible] (target)   at (6,0)    {};

    \path (source.north east) edge [above] node {$A$} (f.north west);
    \path (g.north east) edge [above] node {$B^*$} (f.south west);
    \path (g.south east) edge [above] node {$C$} (target.south west);
\end{tikzpicture}
&
$=$
&
\begin{tikzpicture}
    \node[boxmode] (f) at (2,0) {$f$};
    \node[boxmode] (g) at (4,0) {$g$};
    \path (0,0) edge [above] node {$A$} (f);
    \path (f)   edge [above] node {$B$} (g);
    \path (g)   edge [above] node {$C$} (6,0);
\end{tikzpicture}

\end{tabular}
\caption{Commutativity of names and conames}
\label{fig:names}
\end{figure}


\subsection{Strong Compact Closure and the Inner Product}

So far we have not touched on the notion of inner product in terms of the category theory.
The inner product in the category of finite-dimensional hilbert spaces is sequilinear,
which means it is linear in the second argument and conjugatte-linear in the first argument.
In other words,
\[ \Braket{\alpha \varphi_1 + \beta \varphi_2 \mid \psi} 
   = \overline{\alpha} \Braket{\varphi_1 \mid \psi} + \overline{\beta} \Braket{\varphi_2 \mid \psi}
\]
where $\overline{\alpha}$ is the complex conjugate of $\alpha$.


\begin{definition}
    A dagger category is a category $C$ equipped with a contravariant functor $(-)^\dagger$ 
    which is the identity on objects, and such that 
    \[ (g \circ f)^\dagger = f^\dagger \circ g^\dagger \qquad\text{and}\qquad f^{\dagger\dagger} = f. \]
\end{definition}

A morphism $f$ is called \emph{unitary} if $f$ is an isomorphism and $f^{-1}=f^\dagger$.

\begin{definition}
    A strongly compact closed category is a dagger category which is also compact closed,
    where $\alpha$, $\lambda$, $\rho$ and $\sigma$ are all componentwise unitary transformations,
    where the dagger is a strong monoidal functor, meaning:
    \[ (f \tensor g)^\dagger = f^\dagger \tensor g^\dagger; \]
    and where
    \[ \varepsilon_A = A \tensor A^*
        \xrightarrow{\sigma_{A,A^*}} A^* \tensor A
        \xrightarrow{\eta_A^\dagger} 1.
    \]
\end{definition}

The inner product is intrinsically tied to Dirac notation; the inner product of two states
$\Ket{\varphi}$ and $\Ket{\psi}$ is $\Braket{\varphi \mid \psi}$. The codomain of the inner product
should be a scalar, which in the category theory can be represented as a morphism from $1$ to $1$.
In the category of hilbert spaces for example, any morphism $f : \mathbb{C} \rightarrow \mathbb{C}$
can be uniquely identified with $f(1) \in \mathbb(C)$, the usual notion of a complex scalar.

With that in mind, we define a state or ket of the system to be a morphism $\varphi : 1 \rightarrow A$,
also denoted $\Ket{\varphi}$ for emphasis. Respectively, a co-state is a morphism
$\varphi^\dagger : A \rightarrow 1$. The inner product of two states $\Ket{\varphi}$ and $\Ket{\psi}$
is the scalar
\[ \Braket{\varphi \mid \psi} = \varphi^\dagger \circ \psi : 1 \rightarrow 1. \]

As expected from quantum theory, unitary morphisms preserve the inner product:
\begin{align*}
    \Braket{U\circ \varphi \mid U \circ \psi}
    &= (U \circ \varphi)^\dagger \circ (U \circ \psi) \\
    &= \varphi^\dagger \circ U^\dagger \circ U \circ \psi \\
    &= \varphi^\dagger \circ \psi = \Braket{\varphi \mid \psi}
\end{align*}


\subsection{Measurement via Biproducts}

Abramksy and Coecke's categorical model subscribes to the many-worlds interpretation of quantum
theory. That is, the act of measuring a qubit results in two possible branches---one in 
which the qubit was measured $\Ket{0}$ and one in which the qubit was measured $\Ket{1}$.
As a morphism this measurement will be modeled as
\[ \meas : \Qubit \rightarrow A \oplus A \]
where $\Qubit$ is the space of qubits and $A$ is some resulting state space.

\begin{definition}
    A category is said to have biproducts if for every $A_1,\ldots,A_n$
    there is an object $A_1 \oplus \cdots \oplus A_n$ along with
    projections and injections forming both a product and a coproduct.
    
    For $f : A \rightarrow C$ and $g : B \rightarrow C$, let $[f,g]$
    denote the morphism from $A \oplus B$ to $C$.

    For $f : A \rightarrow B$ and $g : A \rightarrow C$, let $\langle f,g\rangle$ denote
    the morphism from $A$ to $B \oplus C$.
\end{definition}

\subsubsection*{Example} In the category of finite-dimensional hilbert spaces, the 
biproduct is the direct sum.\footnote{Also the direct product; the two coincide
for finite-dimensional vector spaces.}

\subsubsection*{Linearity of morphisms}

 In any category with biproducts, we can define addition and scalar multiplication
of morphisms as follows:
\begin{enumerate}
    \item For $\alpha : 1 \rightarrow 1$ and $f : A \rightarrow B$, define
    $\alpha \cdot f : A \rightarrow B$ as
    \[ \alpha \cdot f = A
    \xrightarrow{\lambda_A} 1 \tensor A
    \xrightarrow{\alpha \tensor \id_A} 1 \tensor A
    \xrightarrow{\lambda_A^{-1}} A.
    \]
    \item For $f, g : A \rightarrow B$, define $f + g : A \rightarrow B$ as
    \[ f + g : A
    \xrightarrow{\langle \id_A, \id_A \rangle}
    A \oplus A
    \xrightarrow{f \oplus g}
    B \oplus B  
    \xrightarrow{[\id_A,\id_A]}
    B
    \]
\end{enumerate}

\subsubsection*{Distributivity}
We can define a right distributivity natural isomorphism (and similarly for left distributivity)
as
\[
    \tau_{A,B,C} = A \tensor (B \oplus C)
    \xrightarrow{\langle \id_A \tensor \pi_1, \id_A \tensor \pi_2\rangle}
    (A \tensor B) \oplus (A \tensor C)
\]
If the biproduct is thought of as parallel worlds, then the distributivity of $\tensor$
over $\oplus$ represents the propogation of classical information $A$ to the two
quantum measurement branches $B$ and $C$. The classical information is not duplicated
because $B$ and $C$ never occur together. As we shall see in the teleportation
example, the distributivity really corresponds to classical transmission of data.
In particular, the speed of this transmission is limited by the speed of light.


