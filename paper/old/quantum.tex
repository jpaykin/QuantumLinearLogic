\documentclass[12pt]{amsart}
\newif\ifdraft
\drafttrue

\usepackage[T1]{fontenc}
\usepackage[utf8]{inputenc}

\usepackage{lmodern}
\usepackage{microtype}
\usepackage{fullpage}
\usepackage{etoolbox}
\usepackage{amsmath,amsfonts,amsthm}
\usepackage{thmtools}
\usepackage{mathtools}
\usepackage{graphicx,caption,subcaption} %subfigures

% The newclude package tries to define this command (but only when thmtools is
% included?)
\makeatletter
\def\restore@saved@for@newclude{%
  \undef\save@for@newclude
  \undef\restore@saved@for@newclude
}
\def\save@for@newclude#1{
  \cslet{save@old@\string#1}#1
  \let#1\relax
  \apptocmd\restore@saved@for@newclude
    {\letcs#1{save@old@\string#1}\csundef{save@old@\string#1}}%
    {}{}
}
\save@for@newclude\g@prependto@macro
\usepackage{newclude}
\restore@saved@for@newclude
\makeatother

\usepackage{natbib}
\setcitestyle{authoryear,round,semicolon,aysep={,},yysep={,}}

\usepackage{newunicodechar}
\newunicodechar{ψ}{\psi}

%    Q-circuit version 2
%    Copyright (C) 2004  Steve Flammia & Bryan Eastin
%    Last modified on: 9/16/2011
%
%    This program is free software; you can redistribute it and/or modify
%    it under the terms of the GNU General Public License as published by
%    the Free Software Foundation; either version 2 of the License, or
%    (at your option) any later version.
%
%    This program is distributed in the hope that it will be useful,
%    but WITHOUT ANY WARRANTY; without even the implied warranty of
%    MERCHANTABILITY or FITNESS FOR A PARTICULAR PURPOSE.  See the
%    GNU General Public License for more details.
%
%    You should have received a copy of the GNU General Public License
%    along with this program; if not, write to the Free Software
%    Foundation, Inc., 59 Temple Place, Suite 330, Boston, MA  02111-1307  USA

% Thanks to the Xy-pic guys, Kristoffer H Rose, Ross Moore, and Daniel Müllner,
% for their help in making Qcircuit work with Xy-pic version 3.8.  
% Thanks also to Dave Clader, Andrew Childs, Rafael Possignolo, Tyson Williams,
% Sergio Boixo, Cris Moore, Jonas Anderson, and Stephan Mertens for helping us test 
% and/or develop the new version.

\usepackage{xy}
\xyoption{matrix}
\xyoption{frame}
\xyoption{arrow}
\xyoption{arc}

\usepackage{ifpdf}
\ifpdf
\else
\PackageWarningNoLine{Qcircuit}{Qcircuit is loading in Postscript mode.  The Xy-pic options ps and dvips will be loaded.  If you wish to use other Postscript drivers for Xy-pic, you must modify the code in Qcircuit.tex}
%    The following options load the drivers most commonly required to
%    get proper Postscript output from Xy-pic.  Should these fail to work,
%    try replacing the following two lines with some of the other options
%    given in the Xy-pic reference manual.
\xyoption{ps}
\xyoption{dvips}
\fi

% The following resets Xy-pic matrix alignment to the pre-3.8 default, as
% required by Qcircuit.
\entrymodifiers={!C\entrybox}

\newcommand{\bra}[1]{{\left\langle{#1}\right\vert}}
\newcommand{\ket}[1]{{\left\vert{#1}\right\rangle}}
    % Defines Dirac notation. %7/5/07 added extra braces so that the commands will work in subscripts.
\newcommand{\qw}[1][-1]{\ar @{-} [0,#1]}
    % Defines a wire that connects horizontally.  By default it connects to the object on the left of the current object.
    % WARNING: Wire commands must appear after the gate in any given entry.
\newcommand{\qwx}[1][-1]{\ar @{-} [#1,0]}
    % Defines a wire that connects vertically.  By default it connects to the object above the current object.
    % WARNING: Wire commands must appear after the gate in any given entry.
\newcommand{\cw}[1][-1]{\ar @{=} [0,#1]}
    % Defines a classical wire that connects horizontally.  By default it connects to the object on the left of the current object.
    % WARNING: Wire commands must appear after the gate in any given entry.
\newcommand{\cwx}[1][-1]{\ar @{=} [#1,0]}
    % Defines a classical wire that connects vertically.  By default it connects to the object above the current object.
    % WARNING: Wire commands must appear after the gate in any given entry.
\newcommand{\gate}[1]{*+<.6em>{#1} \POS ="i","i"+UR;"i"+UL **\dir{-};"i"+DL **\dir{-};"i"+DR **\dir{-};"i"+UR **\dir{-},"i" \qw}
    % Boxes the argument, making a gate.
\newcommand{\meter}{*=<1.8em,1.4em>{\xy ="j","j"-<.778em,.322em>;{"j"+<.778em,-.322em> \ellipse ur,_{}},"j"-<0em,.4em>;p+<.5em,.9em> **\dir{-},"j"+<2.2em,2.2em>*{},"j"-<2.2em,2.2em>*{} \endxy} \POS ="i","i"+UR;"i"+UL **\dir{-};"i"+DL **\dir{-};"i"+DR **\dir{-};"i"+UR **\dir{-},"i" \qw}
    % Inserts a measurement meter.
    % In case you're wondering, the constants .778em and .322em specify
    % one quarter of a circle with radius 1.1em.
    % The points added at + and - <2.2em,2.2em> are there to strech the
    % canvas, ensuring that the size is unaffected by erratic spacing issues
    % with the arc.
\newcommand{\measure}[1]{*+[F-:<.9em>]{#1} \qw}
    % Inserts a measurement bubble with user defined text.
\newcommand{\measuretab}[1]{*{\xy*+<.6em>{#1}="e";"e"+UL;"e"+UR **\dir{-};"e"+DR **\dir{-};"e"+DL **\dir{-};"e"+LC-<.5em,0em> **\dir{-};"e"+UL **\dir{-} \endxy} \qw}
    % Inserts a measurement tab with user defined text.
\newcommand{\measureD}[1]{*{\xy*+=<0em,.1em>{#1}="e";"e"+UR+<0em,.25em>;"e"+UL+<-.5em,.25em> **\dir{-};"e"+DL+<-.5em,-.25em> **\dir{-};"e"+DR+<0em,-.25em> **\dir{-};{"e"+UR+<0em,.25em>\ellipse^{}};"e"+C:,+(0,1)*{} \endxy} \qw}
    % Inserts a D-shaped measurement gate with user defined text.
\newcommand{\multimeasure}[2]{*+<1em,.9em>{\hphantom{#2}} \qw \POS[0,0].[#1,0];p !C *{#2},p \drop\frm<.9em>{-}}
    % Draws a multiple qubit measurement bubble starting at the current position and spanning #1 additional gates below.
    % #2 gives the label for the gate.
    % You must use an argument of the same width as #2 in \ghost for the wires to connect properly on the lower lines.
\newcommand{\multimeasureD}[2]{*+<1em,.9em>{\hphantom{#2}} \POS [0,0]="i",[0,0].[#1,0]="e",!C *{#2},"e"+UR-<.8em,0em>;"e"+UL **\dir{-};"e"+DL **\dir{-};"e"+DR+<-.8em,0em> **\dir{-};{"e"+DR+<0em,.8em>\ellipse^{}};"e"+UR+<0em,-.8em> **\dir{-};{"e"+UR-<.8em,0em>\ellipse^{}},"i" \qw}
    % Draws a multiple qubit D-shaped measurement gate starting at the current position and spanning #1 additional gates below.
    % #2 gives the label for the gate.
    % You must use an argument of the same width as #2 in \ghost for the wires to connect properly on the lower lines.
\newcommand{\control}{*!<0em,.025em>-=-<.2em>{\bullet}}
    % Inserts an unconnected control.
\newcommand{\controlo}{*+<.01em>{\xy -<.095em>*\xycircle<.19em>{} \endxy}}
    % Inserts a unconnected control-on-0.
\newcommand{\ctrl}[1]{\control \qwx[#1] \qw}
    % Inserts a control and connects it to the object #1 wires below.
\newcommand{\ctrlo}[1]{\controlo \qwx[#1] \qw}
    % Inserts a control-on-0 and connects it to the object #1 wires below.
\newcommand{\targ}{*+<.02em,.02em>{\xy ="i","i"-<.39em,0em>;"i"+<.39em,0em> **\dir{-}, "i"-<0em,.39em>;"i"+<0em,.39em> **\dir{-},"i"*\xycircle<.4em>{} \endxy} \qw}
    % Inserts a CNOT target.
\newcommand{\qswap}{*=<0em>{\times} \qw}
    % Inserts half a swap gate.
    % Must be connected to the other swap with \qwx.
\newcommand{\multigate}[2]{*+<1em,.9em>{\hphantom{#2}} \POS [0,0]="i",[0,0].[#1,0]="e",!C *{#2},"e"+UR;"e"+UL **\dir{-};"e"+DL **\dir{-};"e"+DR **\dir{-};"e"+UR **\dir{-},"i" \qw}
    % Draws a multiple qubit gate starting at the current position and spanning #1 additional gates below.
    % #2 gives the label for the gate.
    % You must use an argument of the same width as #2 in \ghost for the wires to connect properly on the lower lines.
\newcommand{\ghost}[1]{*+<1em,.9em>{\hphantom{#1}} \qw}
    % Leaves space for \multigate on wires other than the one on which \multigate appears.  Without this command wires will cross your gate.
    % #1 should match the second argument in the corresponding \multigate.
\newcommand{\push}[1]{*{#1}}
    % Inserts #1, overriding the default that causes entries to have zero size.  This command takes the place of a gate.
    % Like a gate, it must precede any wire commands.
    % \push is useful for forcing columns apart.
    % NOTE: It might be useful to know that a gate is about 1.3 times the height of its contents.  I.e. \gate{M} is 1.3em tall.
    % WARNING: \push must appear before any wire commands and may not appear in an entry with a gate or label.
\newcommand{\gategroup}[6]{\POS"#1,#2"."#3,#2"."#1,#4"."#3,#4"!C*+<#5>\frm{#6}}
    % Constructs a box or bracket enclosing the square block spanning rows #1-#3 and columns=#2-#4.
    % The block is given a margin #5/2, so #5 should be a valid length.
    % #6 can take the following arguments -- or . or _\} or ^\} or \{ or \} or _) or ^) or ( or ) where the first two options yield dashed and
    % dotted boxes respectively, and the last eight options yield bottom, top, left, and right braces of the curly or normal variety.  See the Xy-pic reference manual for more options.
    % \gategroup can appear at the end of any gate entry, but it's good form to pick either the last entry or one of the corner gates.
    % BUG: \gategroup uses the four corner gates to determine the size of the bounding box.  Other gates may stick out of that box.  See \prop.

\newcommand{\rstick}[1]{*!L!<-.5em,0em>=<0em>{#1}}
    % Centers the left side of #1 in the cell.  Intended for lining up wire labels.  Note that non-gates have default size zero.
\newcommand{\lstick}[1]{*!R!<.5em,0em>=<0em>{#1}}
    % Centers the right side of #1 in the cell.  Intended for lining up wire labels.  Note that non-gates have default size zero.
\newcommand{\ustick}[1]{*!D!<0em,-.5em>=<0em>{#1}}
    % Centers the bottom of #1 in the cell.  Intended for lining up wire labels.  Note that non-gates have default size zero.
\newcommand{\dstick}[1]{*!U!<0em,.5em>=<0em>{#1}}
    % Centers the top of #1 in the cell.  Intended for lining up wire labels.  Note that non-gates have default size zero.
\newcommand{\Qcircuit}{\xymatrix @*=<0em>}
    % Defines \Qcircuit as an \xymatrix with entries of default size 0em.
\newcommand{\link}[2]{\ar @{-} [#1,#2]}
    % Draws a wire or connecting line to the element #1 rows down and #2 columns forward.
\newcommand{\pureghost}[1]{*+<1em,.9em>{\hphantom{#1}}}
    % Same as \ghost except it omits the wire leading to the left. 

%\let\bra\relax
%\let\ket\relax
\usepackage{braket}

\usepackage[svgnames]{xcolor}

\usepackage{tikz}
\usetikzlibrary{calc,shapes,arrows,decorations.pathreplacing}
%\makeatletter
\newcommand*{\QCircuitSetup}[1]{%
  % #1 = length of wire.
  \def\wirelength{#1}%
  %
  \def\labeledket##1##2{##1 \quad \mathrlap{\ket{##2}}{\phantom{\ket{ψ}}}}%
  \tikzset{wire label/.style={minimum width=1.5em}}%
  %
  \tikzset{gate/.style={rectangle, minimum width=1em, draw=black, fill=white}}%
  \def\cgateradius{6pt}%
  %
  \def\hadamard at##1;{%
    \node[gate] at ##1 {$H$} ;%
  }%
  %
  \def\control at##1;{%
    \filldraw[black] ##1 circle (\cgateradius) ;%
  }%
  %
  \def\cnot at##1from##2to##3;{%
    \control at  (##1,##2) ;%
    \draw[black] (##1,##3) circle (\cgateradius) ;%
    \draw[black] (##1,##2) -- ($(##1,##3)+(0,-\cgateradius)$) ;%
  }%
  %
  \def\wire at##1for##2from##3to##4;{%
    \draw (0,##1)%
          node[wire label, left]  {$\labeledket{##2}{##3}$}%
          --%
          (\wirelength,##1)%
          node[wire label, right] {$\ket{##4}$} ;%
  }%
  %
  \def\swire at##1from##2;{%
    \draw (0,##1)%
          node[wire label, left] {$\ket{##2}$}%
          --%
          (\wirelength,##1) ;%
  }%
  %
  \tikzset{cwire/.style={double distance=2pt}}%
  %
  \def\finally from##1to##2state##3;{%
    \draw[decorate,decoration={brace, amplitude=1em},xshift=0.2cm]
      ($(\wirelength,##1)+(0,0.2)$)%
      --%
      ($(\wirelength,##2)+(0,-0.2)$)%
      node[midway,right=0.4cm] {$##3$} ;%
  }%
  %
  \def\send cbit at##1from##2to##3;{%
    \draw[cwire] (##1,##2) -- (##1,##3) ;%
    \control at (##1,##2) ;%
  }%
  %
  \def\qstate at##1below by##2from##3to##4state##5;{%
    \draw[loosely dashed, draw=gray]
      ($(##1,##3)+(0,1)$)%
      --%
      ($(##1,##4)+(0,-##2)$)%
      node[below, draw, solid, thin] {$##5$} ;%
  }%
}
\makeatother

\newcommand*{\tikzmark}[1]{\tikz[overlay,remember picture] \node (#1) {};}

\usepackage[section]{placeins} % Floats should stay in their section

\usepackage[ colorlinks
           , pagebackref
           , linktoc = all
           , pdfusetitle ]{hyperref}
\usepackage[all]{hypcap}
\usepackage{fix-natbib-hyperref} % From a TeX.SE answer
\renewcommand*{\backrefalt}[4]{%
  \ifcase #1 %
    No citations.%
  \or
    Cited on page #2.%
  \else
    Cited on pages #2.%
  \fi
}
\hypersetup{ linkcolor = DarkBlue
           , citecolor = DarkGreen
           , urlcolor  = DarkGoldenrod
           , filecolor = Orange
           , menucolor = DarkRed
           , runcolor  = OrangeRed }

% Cleveref (must be loaded late---after hyperref, no less!)
\usepackage[noabbrev,capitalize,nameinlink]{cleveref}
\newcommand{\creflastconjunction}{, and~}

\makeatletter

\newcommand{\asz}[1]{\ifdraft\textcolor{violet}{[ASZ: #1]}\fi}
\newcommand{\jp}[1]{\ifdraft\textcolor{orange}{[JP: #1]}\fi}

\declaretheoremstyle[
  headfont=\normalfont\bfseries,
  notefont=\normalfont\bfseries, notebraces={(}{)},
  bodyfont=\normalfont\itshape
]{myplain}

\declaretheorem[style=myplain]{theorem}
\declaretheorem[style=myplain]{lemma}
\declaretheorem[style=myplain]{definition}

\newcount\qcmark@ds@count
\newcommand*{\qcmark@make@ds}[1]{%
  \qcmark@ds@count=#1%
  \advance\qcmark@ds@count by -1
  \edef\qcmark@ds{}%
  \loop
    \edef\qcmark@ds{d\qcmark@ds}%
    \advance\qcmark@ds@count by -1
  \ifnum\qcmark@ds@count>0\repeat}

% \ar takes a real @ :-)
\makeatother
\def\qcmarkAT{@}
\makeatletter

% The format of this command is for ease of \expandafter-ing
\edef\qcmarkCMD#1&#2\relax{%
  \noexpand\ar \qcmarkAT{--}[]+<#2,1em>;[#1]+<#2,-1em>%
}

\newcommand*{\qcmark}[2][0em]{%
  \qcmark@make@ds{#2}%
  \expandafter\qcmarkCMD\qcmark@ds&#1\relax
}

\newcommand{\id}{\operatorname{id}}
\newcommand{\unit}{\mathbf{1}}
\newcommand{\zero}{\mathbf{0}}
\newcommand{\basis}{\operatorname{basis}}
\newcommand{\lift}[1]{\ulcorner #1 \urcorner}
\newcommand{\llower}[1]{\llcorner #1 \lrcorner}
\newcommand{\meas}{\operatorname{meas}}
\newcommand{\Qubit}{Q}
\newcommand{\CNOT}{\mathit{CNOT}}
\newcommand*{\metersym}{\tikz \draw[rotate=45]
  (1.1em,0) arc (0:90:1.1em)
  (0.55em,0.55em) -- ++(0.825em,0)
;}
\newcommand*{\thinnerspace}{\tmspace+\thinmuskip{.08333em}} % Half a \,
\newcommand*{\Z}{\mathbb{Z}}
\newcommand*{\R}{\mathbb{R}}
\newcommand*{\C}{\mathbb{C}}
\newcommand*{\T}{\mathsf{T}}
\newcommand*{\bit}{\mathbb{B}}
\newcommand*{\qubit}{\mathbb{Q}}
\newcommand*{\tensor}{\otimes}
\newcommand*{\cross}{\times}
\newcommand*{\mtimes}{\cdot}
\newcommand*{\pos}{\hphantom{-}}
% The definition of \targ from QCircuit.tex, with the \qw removed
\newcommand*{\qctarget}{*+<.02em,.02em>{\xy ="i","i"-<.39em,0em>;"i"+<.39em,0em>
    **\dir{-}, "i"-<0em,.39em>;"i"+<0em,.39em>
    **\dir{-},"i"*\xycircle<.4em>{} \endxy}}
\makeatother % \Qcircuit needs a real @ sign
\newcommand*{\QcircuitDemo}[2][\thinnerspace]{%
  #1\raisebox{.8ex}{\Qcircuit@C=1em{#2}}#1%
}
\makeatletter

\let\picvector\vector
\let\vector\relax
\def\vector@end{\vector@end}
\def\vector@last{\vector@last}
\def\vector@grab#1:#2 #3\vector@end{%
  \def\vector@temp{#3}%
  \ifx\vector@temp\vector@last
    #2%
  \else
    #2#1\vector@grab#1:#3\vector@end
  \fi
}
  
\newcommand{\vector}[2][\\]{%
  \begin{pmatrix} \vector@grab#1:#2 \vector@last\vector@end \end{pmatrix}%
}

\newcommand*{\term}[1]{\emph{#1}}

\newcommand*{\latin}[1]{\textit{#1}}
\newcommand*{\IE}{\latin{i.e.}}
\newcommand*{\EG}{\latin{e.g.}}

% References
\newcommand*\crefme[1]{\textcolor{red}{\{!#1!\}}}
\newcommand*\citeme[1]{\textcolor{red}{\{?#1?\}}}

\makeatother

\newcommand{\Hilb}{\textsc{Hilb}}
\newcommand{\inl}{\operatorname{inl}}
\newcommand{\inr}{\operatorname{inr}}
\newcommand{\filter}{\operatorname{filter}}


\begin{document}

\section{Quantum computation}\label{sec:quantum-computation}

When we change from classical\footnote{By \term{classical} here, we do not mean
``affirming the law of the excluded middle'', as in classical logic
vs.\ intuitionistic logic; we mean ``following the laws of traditional Newtonian
physics'', as in classical physics vs.\ quantum mechanics.  This naming
collision is unfortunate; however, in this paper, we do not discuss
classical-logic--inspired computation, and so the term will always refer to the
physical concept.\asz{This probably belongs earlier.}} to quantum computation,
our primitive data type changes from the classical bit to the quantum
\term{qubit} (short for \emph{qu}antum \emph{bit}).  At first glance, qubits
behave much like bits: just as measuring a bit produces one of the two values
$0$ or $1$, measuring a qubit produces one of the two values $\ket{0}$ or
$\ket{1}$.  However, qubits behave very differently when \emph{not} being
measured.  This is because qubits exhibit two properties that classical bits do
not:\asz{\term{superposition} and \term{entanglement}.
\begin{enumerate}
  \item \term{Superposition}: A qubit, prior to measurement, can be a
    \emph{combination} of $\ket{0}$ and $\ket{1}$.

  \item \term{Entanglement}: Two (or more) qubits can be related in such a way
    that measuring the value of the first provides information about the value
    of the second.\asz{This explanation doesn't make clear why entanglement is
    magic, and I don't really know how to explain this without the math.  I'll
    come back to this.}
\end{enumerate}

To explore the computational properties these give rise to, we will explore four
different algorithms that require more of these features:
\begin{enumerate}
  \item Bitwise negation (\cref{sec:bitwise-negation-hl}), which is classical;
  \item Simulating a fair coin (\cref{sec:fair-coin-hl}), which requires
    superposition;
  \item Agreeing on the result of a fair coin (\cref{sec:agree-coin-hl}), which
    requires entanglement; and
  \item Quantum teleportation (\cref{sec:teleportation-hl}), which makes
    nontrivial use of both superposition and entanglement.
\end{enumerate}

\subsection{Bitwise negation}\label{sec:bitwise-negation-hl}
Given a qubit (or, equally well, a classical bit), we can send it through a not
gate, $X$, which changes $\ket{0}$ into $\ket{1}$ and $\ket{1}$ into $\ket{0}$;
the quantum circuit diagram\asz{We need to explain how to read these.} for this
is in \cref{fig:bitwise-negation}, demonstrating the bitwise negation of
$\ket{0}$.

\begin{figure}
  \centerline{\Qcircuit{
    \lstick{\ket{0}} & \gate{X} & \qw & \lstick{\ket{1}}
  }}
  \caption{Bitwise negation (of $\ket{0}$)}\label{fig:bitwise-negation}
\end{figure}

\subsection{Simulating a fair coin}\label{sec:fair-coin-hl}
As we said above, a qubit may be in a \term{superposition} of $\ket{0}$ and
$\ket{1}$; such states are called \term{mixed states}.  The simplest
superpositions are those which are ``$50\%$ $\ket{0}$ and $50\%$ $\ket{1}$''; to
produce this, we use the quantum \term{Hadamard gate} $H$, which takes both
$\ket{0}$ and $\ket{1}$ to such mixed states.\footnote{It takes them to two
\emph{different} such states -- see \asz{some later section}.}  Then, when this
mixed state is measured, the measurement is \emph{probabilistic}: it has a 50\%
chance of being $\ket{0}$ and a 50\% chance of being $\ket{1}$.
\jp{More should be said about measurement.}


\subsection{Agreeing on the result of a fair coin}\label{sec:agree-coin-hl}
\asz{Maybe we only need this (and the preceding two) at the concrete level.}

\subsection{Quantum teleportation}\label{sec:teleportation-hl}
Quantum teleportation allows us to do something very special: given some shared
state between two parties, sending only two classical bits will allow us to
transmit an entire qubit (\IE, two complex numbers).  Suppose Alice wishes to
send a single qubit to Bob.  Then the algorithm proceeds as follows.

\begin{enumerate}
  \item \emph{Preparation.} Alice and Bob meet, and entangle two qubits, $a$ and
    $b$, so that they are guaranteed to be equal but which qubit they will be
    measured as is arbitrary.  Then they leave; Alice takes qubit $a$ with her,
    and Bob takes qubit $b$ with him.

  \item \emph{Composition.}  Sometime later, Alice places her message qubit $m$
    into some arbitrary state $\ket{\psi}$.\asz{I \emph{think} kets aren't only
    used for pure states.}

  \item \emph{Measurement.}  Alice now wishes to send $\psi$ to Bob.  In order
    to do this, Alice entangles $m$ and $a$ via a rotation\asz{check this
    terminology} of the two: all pure states ($\ket{00}$, $\ket{01}$,
    $\ket{10}$, and $\ket{11}$) are transformed into superpositions of two
    different pure states.  Since $a$ was already entangled with Bob's qubit
    $b$, this means that $m$ is entangled with $b$ as well.\asz{Should we split
    this next off into a separate step?}  Alice then measures $a$ and $b$,
    getting the two classical bits $m_M$ and $a_M$.

  \item \emph{Classical communication.}  Alice sends Bob the classical message
    $(m_M,a_M)$, via any arbitrary classical channel.

  \item \asz{I decided this isn't clear without the details.  Maybe I'll try
    explaining this to random victims -- er, friends -- and see how clear I can
    make it.} 
\end{enumerate}

\section{Quantum teleportation}\label{sec:teleportation}

There are at least three ways to think about the quantum teleportation
algorithm.  The first is to say that it's a method for transfering an arbitrary
quantum state from one person two another; this focuses on the ``teleportation''
part.  The second is to say that it's a way to send two complex numbers (a
quantum state) to another person by sending them two bits, as long as you share
some pre-arranged material (an entangled state); this focuses on the way that
quantum computation allows unusually efficient computation.  The third way is to
say that quantum teleportation is the construction of ``a `single-use
isomorphism' between the (otherwise isomorphic) types [of a qubit] and [of two
bits]''  (Selinger and Valiron, 2009\asz{reify this citation}); that is, it
allows us to turn a qubit into two bits, and turn those two bits back into a
qubit – but only \emph{once}.

We now walk through a standard approach to the quantum teleportation algorithm,
as seen in Abramsky and Coecke (2008) and Selinger and Valiron (2009)\asz{reify
these citations}.  The algorithm's quantum circuit presentation is given in
\cref{fig:quantum-teleportation-circuit}

\begin{figure}
\def\Qm{0}
\def\Qa{-2}
\def\Qb{-4}
\begin{tikzpicture}
  \def\labeledket#1#2{\mathllap{#1 \qquad} \ket{#2}}
  \tikzset{wire label/.style={minimum width=1.5em}}
  
  \tikzset{gate/.style={rectangle, minimum width=1em, draw=black, fill=white}}
  \def\cgateradius{4pt}
  
  \def\hadamard at#1;{
    \node[gate] at #1 {$H$} ;
  }

  \def\control at#1;{
    \filldraw[black] #1 circle (\cgateradius) ;
  }
  
  \def\cnot at#1from#2to#3;{
    \control at  (#1,#2) ;
    \draw[black] (#1,#3) circle (\cgateradius) ;
    \draw[black] (#1,#2) -- ($(#1,#3)+(0,-\cgateradius)$) ;
  }
  
  \def\wire at#1for#2from#3to#4;{
    \draw (0,#1)
          node[wire label, left]  {$\labeledket{#2}{#3}$}
          --
          (14,#1)
          node[wire label, right] {$\ket{#4}$} ;
  }
  
  \tikzset{cwire/.style={double distance=2pt}}

  \def\send cbit at#1from#2to#3;{
    \draw[cwire] (#1,#2) -- (#1,#3) ;
    \control at (#1,#2) ;
  }
  
  \wire at \Qm for m from ψ to x ;
  \wire at \Qa for a from 0 to y ;
  \wire at \Qb for b from 0 to ψ ;
  
  \hadamard at (1.5,\Qa) ;
  \cnot at 3 from \Qa to \Qb ;
  
  \cnot at 5 from \Qm to \Qa ;
  \hadamard at (6.5,\Qm) ;

  \pgfmathsetmacro\Qma{((\Qm)+(\Qa))/2}
  \pgfmathsetmacro\meassize{ 2cm               % gate width
                           + 1.5em             % extra for plain gate size
                           - 1.4pt - 0.35 pt } % line width adjustment
  \pgfmathsetmacro\meassizehalf{\meassize/2}
  \pgfmathsetmacro\measwidthflat{2.5cm - \meassizehalf pt}

  \draw[cwire] (9,\Qm) -- (14,\Qm) ;
  \draw[cwire] (9,\Qa) -- (14,\Qa) ;
  
  \path[fill=white,draw=black]
    ($(8,\Qma) + (0,\meassizehalf pt)$)
    -- ++(\measwidthflat pt,0)
    arc (90:-90:\meassizehalf pt)
    -- ++(-\measwidthflat pt,0)
    -- cycle ;

  \coordinate (meas meter flat mid)
    at ($(8,\Qma) + (\meassizehalf pt, -0.4em) + (-0.25em,0)$) ;
  \draw (meas meter flat mid) ++(-2em,0) .. controls ++(2em,1em) .. ++(4em,0)
        (meas meter flat mid) ++(-0.375em,0) -- ++(0.75em,1.5em) ;

  \send cbit at 12.25 from \Qm to \Qb ;
  \send cbit at 12.75 from \Qa to \Qb ;
  \node[gate] at (12.5,\Qb) {$F_{x,y}$} ;
\end{tikzpicture}
\caption{The quantum circuit diagram for quantum teleportation.}%
\label{fig:quantum-teleportation-circuit}
\end{figure}

\section{Quantum teleportation, with matrices}\label{sec:teleportation-matrices}

\asz{Trying again}

Quantum teleportation allows us to move a $1$-qubit state from Alice to Bob
simply by sending two classical bits, as long as Alice and Bob share an
entangled pair of qubits.  The algorithm is as follows.

\begin{enumerate}
  \item \emph{Preparation.}  Alice and Bob meet, and entangle two qubits, $a$
    and $b$, which were initially each in the state $\ket{0}$.  These qubits
    are entangled into the state \[ \frac{\ket{00} + \ket{11}}{\sqrt{2}} \] by
    the operation
    \begin{align*}
      P
      &= C(H \tensor I) \\
      &= C
         \left[ \frac{1}{\sqrt{2}}
                \begin{pmatrix} 1 & 1 \\ -1 & -1 \end{pmatrix}
                \tensor
                \begin{pmatrix} 1 & 0 \\  0 &  1 \end{pmatrix} \right] \\
      &= C
         \frac{1}{\sqrt{2}}
         \begin{pmatrix}
           1 & 0 &  1 &  0 \\
           0 & 1 &  0 &  1 \\
           1 & 0 & -1 &  0 \\
           0 & 1 &  0 & -1
         \end{pmatrix} \\
      &= \begin{pmatrix}
           1 & 0 & 0 & 0 \\
           0 & 1 & 0 & 0 \\
           0 & 0 & 0 & 1 \\
           0 & 0 & 1 & 0
         \end{pmatrix}
         \frac{1}{\sqrt{2}}
         \begin{pmatrix}
           1 & 0 &  1 &  0 \\
           0 & 1 &  0 &  1 \\
           1 & 0 & -1 &  0 \\
           0 & 1 &  0 & -1
         \end{pmatrix} \\
      &= \frac{1}{\sqrt{2}}
         \begin{pmatrix}
           1 & 0 &  1 &  0 \\
           0 & 1 &  0 &  1 \\
           0 & 1 &  0 & -1 \\
           1 & 0 & -1 &  0
         \end{pmatrix}.
    \end{align*}
    Thus, we have \[ P\ket{00} = P\begin{pmatrix} 1 \\ 0 \\ 0 \\ 0 \end{pmatrix}
    = \frac{1}{\sqrt{2}} \begin{pmatrix} 1 \\ 0 \\ 0 \\ 1 \end{pmatrix} =
    \frac{\ket{00} + \ket{11}}{\sqrt{2}}, \] as we wanted.

  \item \emph{Composition.}  At some later time, Alice places her message qubit
    $m$ into some arbitrary state $\ket{ψ}$.

  \item \emph{Measurement.}  Alice now wishes to send the state $\ket{ψ}$ to
    Bob.  To do this, Alice measures qubits $m$ and $a$ in a \emph{different}
    basis, which involves entangling $m$ and $a$ before measuring them.  This
    means that $m$ is entangled with $b$ as well, which is key.  Entanglement of
    $m$ and $a$ is the reverse of the entanglement of $a$ and $b$:
    \begin{align*}
      M
      &= (H \tensor I)C \\
      &= \left[ \frac{1}{\sqrt{2}}
                \begin{pmatrix} 1 & 1 \\ -1 & -1 \end{pmatrix}
                \tensor
                \begin{pmatrix} 1 & 0 \\  0 &  1 \end{pmatrix} \right]
         C \\
      &= \frac{1}{\sqrt{2}}
         \begin{pmatrix}
           1 & 0 &  1 &  0 \\
           0 & 1 &  0 &  1 \\
           1 & 0 & -1 &  0 \\
           0 & 1 &  0 & -1
         \end{pmatrix}
         C\\
      &= \frac{1}{\sqrt{2}}
         \begin{pmatrix}
           1 & 0 &  1 &  0 \\
           0 & 1 &  0 &  1 \\
           1 & 0 & -1 &  0 \\
           0 & 1 &  0 & -1
         \end{pmatrix}
         \begin{pmatrix}
           1 & 0 & 0 & 0 \\
           0 & 1 & 0 & 0 \\
           0 & 0 & 0 & 1 \\
           0 & 0 & 1 & 0
         \end{pmatrix} \\
      &= \frac{1}{\sqrt{2}}
         \begin{pmatrix}
           1 & 0 &  0 &  1 \\
           0 & 1 &  1 &  0 \\
           1 & 0 &  0 & -1 \\
           0 & 1 & -1 &  0
         \end{pmatrix}
    \end{align*}
    Since $H^2 = I$, $I^2 = I$, and $C^2 = I$, we know that $M = P^{-1}$ (which
    we may also see by noting that $M = P^\dag = P^{\mathsf{T}}$, and $M$ and
    $P$ are unitary).  Thus, we know that $M$ measures in the following basis:
    \begin{alignat*}{4}
      M\frac{\ket{00} + \ket{11}}{\sqrt{2}}
        &= M\frac{1}{\sqrt{2}}\begin{pmatrix} 1 \\ 0 \\ 0 \\ 1 \end{pmatrix}
       &&= \frac{1}{2}\begin{pmatrix} 1+1 \\ 0 \\ 1-1 \\ 0 \end{pmatrix}
        &= \begin{pmatrix} 1 \\ 0 \\ 0 \\ 0 \end{pmatrix}
       &&= \ket{00} \\
      %
      M\frac{\ket{01} + \ket{10}}{\sqrt{2}}
        &= M\frac{1}{\sqrt{2}}\begin{pmatrix} 0 \\ 1 \\ 1 \\ 0 \end{pmatrix}
       &&= \frac{1}{2}\begin{pmatrix} 0 \\ 1+1 \\ 0 \\ 1-1 \end{pmatrix}
        &= \begin{pmatrix} 0 \\ 1 \\ 0 \\ 0 \end{pmatrix}
       &&= \ket{01} \\
      %
      M\frac{\ket{00} - \ket{11}}{\sqrt{2}}
        &= M\frac{1}{\sqrt{2}}\begin{pmatrix} 1 \\ 0 \\ 0 \\ -1 \end{pmatrix}
       &&= \frac{1}{2}\begin{pmatrix} 1-1 \\ 0 \\ 1+1 \\ 0 \end{pmatrix}
        &= \begin{pmatrix} 0 \\ 0 \\ 1 \\ 0 \end{pmatrix}
       &&= \ket{10} \\
      %
      M\frac{\ket{01} - \ket{10}}{\sqrt{2}}
        &= M\frac{1}{\sqrt{2}}\begin{pmatrix} 0 \\ 1 \\ -1 \\ 0 \end{pmatrix}
       &&= \frac{1}{2}\begin{pmatrix} 0 \\ 1-1 \\ 0 \\ 1+1 \end{pmatrix}
        &= \begin{pmatrix} 0 \\ 0 \\ 0 \\ 1 \end{pmatrix}
       &&= \ket{11}
    \end{alignat*}
\end{enumerate}

In other words, teleportation boils down to the following: $T\ket{ψ00} = v'$,
where if we measure the first two bits of $v'$ to get $\hat{v}' \tensor
\ket{ψ'}$, we have $F_{\hat{v}'}\ket{ψ'} = \ket{ψ}$.

\section{Category Theory Background}
\jp{If we want this section, include the following: categories,
functors, natural transformations, adjunctions.}


\section{Categorical Interpretations}

Quantum computation is traditionally presented in the physicist's language
of quantum mechanics---that is, superposition, measurement and wires. 
As quantum computation moves into the realm of higher-order and thus more principled
programming languages, it becomes clear that computer scientists need a way to 
connect the language of quantum mechanics with the language of programming. 
For example, Quipper\cite{green13quipper} uses higher-order features including
functions and monads, but is essentially still a circuit description language.
In order to more cohesively understand the semantics of quantum computation,
Abramsky and Coecke\cite{abramsky2009categorical} describe a thorough 
axiomatization of the major properties of quantum computation and logic
as a categorical model. In this work we will use their model as a guide
for the semantic underpinnings of a number of simple quantum algorithms.


\subsection{Symmetric Monoidal Categories and the No-Cloning Theorem}

A symmetric monoidal category is a category $C$, along with a bifunctor $\tensor$
and a unit $1$, and with natural isormorphisms:
\begin{align*}
    \text{Associativity:}&\qquad 
    \alpha_{A,B,C} : A \tensor (B \tensor C) \rightarrow (A \tensor B) \tensor C \\
    \text{Left and right units:}&\qquad 
    \lambda_A : 1 \tensor A \rightarrow A \text{ and } \rho_A : A \tensor 1 \rightarrow A \\
    \text{Symmetry:}&\qquad 
    \sigma_{A,B} : A \tensor B \rightarrow B \tensor A \\
\end{align*}
In addition, the monoidal structure must commute with the isomorphisms in 
various ways \jp{Cite}.

Along with monoidal categories comes a range of categorical structures tailored
to the monoidal structure: functors, natural transformations, and adjunctions.

A symmetric monoidal functor is a functor $F : C \Rightarrow D$ between two symmetric
monoidal categories, along with a morphism
\[ m_1^F : 1^D \rightarrow F 1^C \]
in C (or D?) and a natural transformation
\[ m_{A,B}^F : F A \tensor F B \rightarrow F(A \tensor B). \]
Again, these morphisms must commute with the monoidal structures in $C$ and $D$
in certain ways not outlined here. 

A monoidal natural transformation between two symmetric monoidal functors
is a natural transformation which commutes with the monoidal components $m_1$ and $m_{A,B}$.
A monoidal adjunction is simply an adjunction where the unit and counit
are monoidal natural transformations.

The no-cloning theorem says there is no way to produce two copies of a single 
arbitrary quantum state. To be more precise, the no-cloning theorem states that
there is no unitary transformation which can duplicate every possible quantum state.

\begin{theorem}[No-Cloning (cite)]
    There is no unitary transformation acting on qubits as follows:
    \[ U(\ket{0}\tensor\ket{\varphi}) = \ket{\varphi}\tensor\ket{\varphi}.\footnotemark \]
    \footnotetext{Unitary transformations are, in traditional quantum mechanics,
    always square matrices, so in this case we ``pad'' the transformation with an
    arbitrary \jp{???} state.}
\end{theorem}
\begin{proof}
    Suppose there were such a transformation. We may expand an arbitrary $\ket{\varphi}$
    as $\varphi = \alpha\ket{0} + \beta\ket{1}$. By the definition of $U$, we know
    \begin{align*} 
        U(\ket{0} \tensor \ket{\varphi})
        &= \ket{\varphi} \tensor \ket{\varphi} \\
        &= (\alpha\ket{0} + \beta\ket{1}) \tensor (\alpha\ket{0} + \beta\ket{1}) \\
        &= \alpha^2 \ket{00} + \alpha\beta\ket{01} + \alpha\beta\ket{10} + \beta^2\ket{11}.
    \end{align*}
    However, since all unitary transformations are necessarily linear, we can also compute
    \begin{align*}
        U(\ket{0} \tensor \ket{\varphi})
        &= \alpha U(\ket{0} \tensor \ket{0}) + \beta U(\ket{0} \tensor \ket{1}) \\
        &= \alpha \ket{00} + \beta \ket{11}.
    \end{align*}
    These two expansions are unequal, thus we have reached a contradiction. 
\end{proof}

So what does this have to do with monoidal categories? 
The use of linearity in the proof of the no-cloning theorem is exactly the overlap.
Without additional structure, monoidal categories do not allow arbitrary
morphisms $A \rightarrow A \tensor A$. Abramsky and Coecke\cite{abramsky2009categorical}
prove the corresponding property holds for their entire categorical model,
but for now it suffices to say that the structure added in the remainder
of this section does not allow arbitrary duplication on the monoidal structure.

\subsection{Compact Closed Categories: Names and Conames}

\subsection{Strong Compact Closure and the Inner Product}

\subsection{Measurement via Biproducts}


\section{Quantum Algorithms}

\bibliographystyle{plain}
\bibliography{../bibliography}


\end{document}

% ASZ: Notes that I want to use later:

% The set of qubits is ℚ; this is ambiguous, so we explain it.
, which we denote $\qubit$.\footnote{We do not have much
call to refer to rational numbers, so we use $\qubit$ to refer exclusively to
qubits, and never to rational numbers.}

% The real definition of a simple mixed state
we write this \[ \frac{1}{\sqrt{2}}\ket{0} + \frac{1}{\sqrt{2}}\ket{1} =
\frac{\ket{0} + \ket{1}}{\sqrt{2}}. \]

% ASZ: I use this to compile my LaTeX document; you can ignore it
(defun my-compile ()
  (interactive)
  (save-buffer)
  (shell-command (concat
    "pdflatex -halt-on-error quantum.tex && "
    "osascript ~/misc/Skim-open-revert.scpt quantum.pdf &")))
(local-set-key (kbd "C-c C-l") #'my-compile)
