Now that we have seen all these pieces, we can put them all together in the
\term{quantum teleportation} algorithm, whose circuit diagram can be seen in
\cref{qcd:teleportation}.  Although the no-cloning theorem
(\cref{thm:no-cloning}) prevents us from copying arbitrary qubits,
we are allowed to copy arbitrary \emph{Newtonian} bits, which are not
restricted by the laws of linearity.
In the teleportation algorithm, instead of transmitting a qubit
through space (represented by a pair
of complex numbers) we will see that it suffices to send a (duplicated) pair of 
Newtonian bits.

\begin{figure}
  \centerline{\Qcircuit{
    \lstick{%
      \tikz[overlay]
        \draw[decorate,decoration={brace, amplitude=1em}]
        (0,-4.1em) -- (0,1em)
        node[midway,left=1.25em] {Alice} ;%
      \ket{\psi}}
      & \qw      & \qw
      & \qw
      & \ctrl{1} & \gate{H}
      & \multimeasureD{1}{\metersym} & \cw
      & % This stuff is shifted left a little
        *!<.5em,.025em>-=-<.5em,.2em>{\bullet} % Control
        \cw
        \ar @{=} @<-.5em> [dd] % Classical wire down 2 spaces
    & \rstick{\ket{x}} \cw
    \\
    \lstick{\ket{0}}
      & \gate{H} & \ctrl{1}
      & \qw
      & \targ    & \qw
      & \ghost{\metersym} & \cw
      & % This stuff is shifted right a little
        *!<-.5em,.025em>-=-<.5em,.2em>{\bullet}
        \cw
        \ar @{=} @<.5em> [d]
    & \rstick{\ket{y}} \cw
    \\
    \lstick{%
      \tikz[overlay,baseline]
        \node[left=-.25em] {\raisebox{2.3ex}{Bob\hspace{1.1em}$\big\{$}} ;%
      \ket{0}}
      & \qw      & \targ
      & \qw
      & \qw      & \qw
      & \qw & \qw
      & \gate{F_{x,y}}
    & \rstick{\ket{\psi}} \qw
    %
    % Is this clearer?
    % \gategroup{2}{2}{3}{3}{1em}{--}
    % \gategroup{1}{5}{2}{6}{1em}{--}
    % \gategroup{1}{7}{2}{7}{1em}{--}
    % \gategroup{1}{9}{2}{9}{2em}{--}
    % \gategroup{3}{9}{3}{9}{1em}{--}
  }}
  \caption{The quantum circuit diagram for quantum teleportation.}
  \label{qcd:teleportation}
\end{figure}

The quantum teleportation algorithm for Alice to send a qubit state $\ket{\psi}
= \psi_0\ket{0} + \psi_1\ket{1}$ to Bob proceeds in five steps.
\begin{enumerate}
\item \term{Initial entanglement.}  Alice and Bob meet up, and entangle two
  qubits.  Alice then blasts off in a rocket with one of the two qubits, leaving
  Bob behind with the other.
\item \term{Message entanglement.}  Far away, Alice takes a qubit in state
  $\ket{\psi}$ that she wishes to send to Bob, and entangles it (differently!)
  with her other qubit.  Transitively, the message qubit is now entangled with
  Bob's qubit.
\item \term{Measurement.}  Alice measures her two qubits, obtaining two
  Newtonian bits $x$ and $y$.  This puts Bob's qubit into a scrambled version of
  $\ket{\psi}$; precisely how it is scrambled depends on what is measured.
\item \term{Newtonian communication.}  Alice sends $x$ and $y$ to Bob via some
  ordinary Newtonian means.
\item \term{Fixing.}  Bob receives Alice's Newtonian transmission, and applies a
  unitary matrix $F_{x,y}$ to his qubit to undo the scrambling.  His qubit is
  now in the state $\ket{\psi}$.
\end{enumerate}
Initially, the state of the system is $\ket{\Psi_0} = \ket{\psi00}$; the
teleportation algorithm transforms this into the state $\ket{\Psi_4} =
\ket{xy\psi}$.  To understand how this happens, we break down the state of the
system after each stage of the algorithm.

\subsection{Initial entanglement}\label{sec:initial-entanglement}
To start with, Alice and Bob meet up and, by applying a Hadamard and controlled
not gate as in \cref{qcd:entangle-2-equal}, obtain a pair of qubits in the state
$((\ket{00} + \ket{11})/\sqrt{2}$.  The matrix applied to these qubits is $E =
\CNOT \mtimes (H \tensor I)$.  At this point, the whole system is in the
state \[ \ket{\Psi_1} = \ket{\psi} \tensor \frac{\ket{00} + \ket{11}}{\sqrt{2}}
= \frac{\ket{\psi00} + \ket{\psi11}}{\sqrt{2}}. \] Alice then blasts off with
her qubits.

\subsection{Message entanglement}\label{sec:message-entanglement}
Alice wants to send $\ket{\psi}$ to Bob, so she applies the opposite of
\cref{qcd:entangle-2-equal}: first a controlled not gate, and then a Hadamard
gate, which is defined by the matrix $(H \tensor I) \mtimes \CNOT$.  Since
$\CNOT$ and $H$ are both Hermitian, we know that $E^\dag = (\CNOT \mtimes (H
\tensor I))^\dag = (H \tensor I)^\dag \mtimes \CNOT^\dag = (H \tensor I) \mtimes
\CNOT$, and so this form of entanglement is simply $E^\dag$, the inverse of the
first.  However, we are applying $E^\dag$ to different qubits than we applied
$E$ to initially, so $E$'s effects are not undone.

To see the effect of $E^\dag$, recall that we are starting in the state \[
  \ket{\Psi_1} = \frac{\ket{\psi00} + \ket{\psi11}}{\sqrt{2}}.
\] Expanding out
the definition of $\psi$ lets us write this as \[
  \ket{\Psi_1} = \frac{ \psi_0\ket{000} + \psi_1\ket{100}
                      + \psi_0\ket{011} + \psi_1\ket{111}}
                      {\sqrt{2}}.
\]  To apply $E^\dag$ to the first two qubits (that is, to compute $(E^\dag
\tensor I_2)\ket{\Psi_1}$), we apply $\CNOT$ to the first two qubits and the $H$
to the very first one.  Applying $\CNOT$ produces the state \[
  \ket{\Psi_2'} = \frac{ \psi_0\ket{000} + \psi_1\ket{110}
                       + \psi_0\ket{011} + \psi_1\ket{101}}
                       {\sqrt{2}},
\] and applying $H$ to that gives the final state \[
  \ket{\Psi_2} = \frac{ \psi_0(\ket{000} + \ket{100})
                      + \psi_1(\ket{010} - \ket{110})
                      + \psi_0(\ket{011} + \ket{111})
                      + \psi_1(\ket{001} - \ket{101})}
                      {2}.
\]  Reordering the states, this is written as
\begin{gather*}
  \ket{\Psi_2} = \frac{ \psi_0\ket{000}
                      + \psi_1\ket{001}
                      + \psi_1\ket{010}
                      + \psi_0\ket{011}
                      + \psi_0\ket{100}
                      - \psi_1\ket{110}
                      - \psi_1\ket{101}
                      + \psi_0\ket{111}}
                      {2} \\
               = \frac 1 2 \vector{\pos\psi_0 \pos\psi_1 \pos\psi_1 \pos\psi_0
                         \pos\psi_0    -\psi_1    -\psi_1 \pos\psi_0}.
\end{gather*}
The state is thoroughly entangled: although the $\psi_0$s and $\psi_1$s repeat
their pattern of four twice, the two negative signs prevent the state from being
separable.

\subsection{Measurement}\label{sec:measurement}
Alice now measures the first two qubits of $\ket{\Psi_2}$ so that she can send
them to Bob.  Looking at the definition of $\ket{\Psi_2}$, we see that this will
put us in worlds given by adjacent coefficients in the vector: if we measure
$\ket{00}$, then we are working with \[ \frac{\psi_0\ket{000} +
\psi_1\ket{001}}{2}, \] and so on.  Importantly, this vector has the property
that grouping it up into pairs always produces one $\psi_0$ and one
$\pm\psi_1$.  Thus, we know two things: first, that which outcome we get is
uniformly random; and second, that the resulting state will have all the
information necessary to reconstruct $\psi$.  Measuring the qubits, then, brings
us into one four possible worlds:
\begingroup
\def\ZX#1{\hphantom{ZX}\mathllap{#1}}
\begin{alignat*}{3}
  \ket{\Psi_{3,00}}
     &= \pos\psi_0\ket{000} + \psi_1\ket{001}
    &&= \ket{00} \tensor \ZX{  }\ket{\psi}
    &&= \ket{00} \tensor{\ket{\psi'_{00}}} \\
  \ket{\Psi_{3,01}}
     &= \pos\psi_1\ket{010} + \psi_0\ket{011}
    &&= \ket{01} \tensor \ZX{ X}\ket{\psi}
    &&= \ket{01} \tensor{\ket{\psi'_{01}}} \\
  \ket{\Psi_{3,10}}
     &= \pos\psi_0\ket{100} - \psi_1\ket{101}
    &&= \ket{10} \tensor \ZX{Z }\ket{\psi}
    &&= \ket{10} \tensor{\ket{\psi'_{10}}} \\
  \ket{\Psi_{3,11}}
     &=   -\psi_1\ket{110} + \psi_0\ket{111}
    &&= \ket{11} \tensor \ZX{ZX}\ket{\psi}
    &&= \ket{11} \tensor{\ket{\psi'_{11}}},
\end{alignat*}
\endgroup
where $X$ is bitwise negation as before, and $Z$ is the unitary
matrix \[ Z = \begin{pmatrix} 1 & \pos0 \\ 0 & -1 \end{pmatrix} \] that flips
the sign of the probability amplitude of $\ket{1}$.

\subsection{Newtonian communication}\label{sec:newtonian-communication}
Now that the whole system is in the state $\ket{\Psi_{3,xy}} =
\ket{xy}\tensor\ket{\psi'_{xy}}$, Alice needs to let Bob know just which of the
$\ket{\psi'_{xy}}$ he has.  Thus, she simply sends him the two bits $(x,y)$.
These two bits alone are not enough to learn anything about $\ket{\psi}$; as we
saw above, they were obtained uniformly at random. Furthermore, without
these two bits, Bob's qubit is essentially random.
Bob cannot learn anything from his
qubit without measuring it, and -- since he does not know which of the
$\ket{xy}$ Alice measured -- we can see that the aggregate probability of
measuring $\ket{0}$ is equal to that of measuring $\ket{1}$ when we look at all
of the $\ket{\Psi_{3,xy}}$ taken together.  Thus, no information about
$\ket{\psi}$ is present in either the Newtonian bits or Bob's qubit; it is only
by bringing them together that Bob can learn what $\ket{\psi}$ was!%
\footnote{This means that quantum mechanics cannot send information faster than
  light, and so does not violate relativity; until the Newtonian bits are sent
  in an ordinary manner (which means at most at the speed of light), Bob does
  not know what $\psi$ was.  This also means that intercepting the classical
  message provides no information, so as long as the qubits are private, this
  scheme is also secure.}

\subsection{Fixing}\label{sec:fixing}
Bob receives Alice's message of $(x,y)$, and uses this to undo the manner in
which his qubit $\ket{\psi'_{xy}}$ was scrambled.
\begin{enumerate}
\item If Bob receives $(0,0)$, then his qubit is simply $\psi$, so he does
  nothing.

\item If Bob receives $(0,1)$, then his qubit is $X\ket{\psi}$; to recover
  $\ket{\psi}$, he applies $X$ again, since bitwise negation is self-inverse.

\item If Bob receives $(1,0)$, then his qubit is $Z\ket{\psi}$; to recover
  $\ket{\psi}$, he applies $Z$ again, since ordinary negation is self-inverse.

\item If Bob receives $(1,1)$, then his qubit is $ZX\ket{\psi}$; to recover
  $\ket{\psi}$, he applies $XZ$, since he must undo the transformations in the
  opposite order.
\end{enumerate}
Thus, we can define the family of matrices $F_{x,y}$ as
\begingroup
\makeatletter
\def\mpos{\@ifnextchar-{}\pos}
\def\Mfmt#1 #2 \\ #3 #4\relax{\begin{pmatrix}
  #1 & \mpos#2 \\
  #3 & \mpos#4
\end{pmatrix}}
\def\M#1{\Mfmt#1\relax}
\begin{alignat*}{2}
  F_{0,0} &=  I_2 &&= \M{1 0  \\ 0  1} \\
  F_{0,1} &=  X   &&= \M{0 1  \\ 1  0} \\
  F_{1,0} &=  Z   &&= \M{1 0  \\ 0 -1} \\
  F_{0,1} &=  XZ  &&= \M{0 -1 \\ 1  0},
\end{alignat*}
\endgroup
and Bob computes \[ \ket{\Psi_4} = (I_4 \tensor F_{x,y})\ket{\Psi_{3,xy}} =
\ket{xy} \tensor F_{x,y}\ket{\psi'_{xy}} = \ket{xy\psi}. \]

\subsection{The whole algorithm}\label{sec:whole-algorithm}
In the end, Alice has consumed two qubits to allow Bob to know -- without
measuring anything -- that he now has a qubit in the state $\ket{\psi}$ that
Alice was sending him.  Another way of putting this is that the quantum
teleportation algorithm computes
\begin{align*}
  \ket{\Psi_2}
    &= \big[(E^\dag \tensor I_2) \mtimes (I_2 \tensor E)\big]\ket{00\psi} \\
    &= \bigg[
         \Big(\big[(H \tensor I_2) \mtimes \CNOT\big] \tensor I_2\Big)
         \mtimes
         \Big(I_2 \tensor \big[\CNOT \mtimes (H \tensor I_2)\big]\Big)
       \bigg]\ket{00\psi}.
\end{align*}
We encourage the reader to work out these matrices for themself; the process of
doing so can be extremely edifying.
