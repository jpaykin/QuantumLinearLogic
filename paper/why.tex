Quantum mechanics underpins our most accurate theories of
reality,\footnotemark{} but our standard models of computation are resolutely
Newtonian: we operate on bits according to a set of mechanistic rules; knowing
these rules is enough to know exactly how a computation will turn out.  Yes, we
are limited by the halting problem and Rice's theorem, but then so too is
Newtonian mechanics limited by chaotic behavior.  All is, eventually, knowable.

\footnotetext{Along with general relativity, but reconciling quantum mechanics
  with general relatively is beyond the scope of this paper.}

Quantum mechanics tells us that the world that we live in is fundamentally
\emph{stochastic}.  What we see when we measure is intrinsically random, and
there is no fact of the matter about the situation before we do.\footnotemark{}
This alone would be enough to wake us up from the Newtonian dream, but quantum
mechanics is yet stranger: two quantum systems may become \term{entangled}, so
that measuring them will produce correlated results, no matter how widely
separated they become!

\footnotetext{In this paper, we take a \term{many-worlds} view of quantum
  mechanics (although other views are possible).  In particular, the Bohmian
  pilot-wave interpretation of quantum mechanics does have definite states
  without measurement, but is inherently \emph{non-local}, which violates
  relativity.  \citeme{}}

Physics has recovered from this shock, and indeed has taken ahold of quantum
mechanics with gusto.  Sadly, computer science (and, it must be admitted, daily
life) has mostly ignored this tremendous awakening, preferring instead to
pretend that actions always have the same consequences, and
computation is predicatable.  
But as we will see in \cref{sec:when}, there is a growing research community studying
\term{quantum computation}: the design of computational models that better reflect
physical reality.

And take advantage they do: it turns out that quantum computation is in many ways more
powerful than the Newtonian computation we know and love.  While quantum
computers can be simulated with Newtonian Turing machines, this appears to
require an exponential slowdown.  \citeme{}  Engaging with quantum computation
on its own terms will prove to be a fruitful way to think about
algorithms we have never been able to effectively express before.

These research efforts are not totally divorced from the work that computer
scientists have already done.  The \term{no-cloning theorem} is a fundamental
result in quantum mechanics, due to \citet{wootters1982single} and
\citet{dieks1982communication} (independently), that says no operation can
duplicate an arbitrary quantum state.  Dually, the \term{no-deleting} theorem,
due to \citet{pati2000impossibility}, says that quantum states cannot be discarded.  These
constraints are strongly reminiscent of \term{linearity}.

Linear logic is a logical formalism that rejects the traditional logical
notions of weakening and contraction; linear propositions can be viewed as
precious resources, which are consumed upon use and may not be discarded.
Linear propositions, in other words, are always used exactly once in a proof.

In this paper, we discuss quantum computation on its own terms, and then
demonstrate how it relates to a categorical model whose internal language is
reminiscent of linear logic.  The model we discuss is due to \citet{abramsky2009categorical};
we offer an alternative presentation of the same ideas.

In this section, we have hopefully convinced you that quantum computation is an
important avenue of study.  In \cref{sec:what}, we will present in more detail
the fragment of quantum mechanics we need for quantum computation, and in
\cref{sec:where} we will go through the gory details of \term{quantum
teleportation}, a paradigmatic example of a quantum algorithm that cannot be
Newtonian.  \Cref{sec:how} will relate quantum mechanics to the categorical model,
and explain how the model gives rise to the expected quantum behavior.  Finally,
\cref{sec:when} wraps up our discussion, and touches briefly on both related
work and where we think quantum computation may be going in the future.
