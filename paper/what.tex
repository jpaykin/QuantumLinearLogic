From quantum mechanics, we learn that the world is defined by Hilbert spaces
that evolve according to unitary transformations.\citeme{find something}

\begin{definition}\label{def:hilbert-space}
  \asz{This is a (slightly modified) quote from the Wikipedia article on Hilbert
  spaces; we need something better.}  ``A \term{Hilbert space} $H$ is a complex
  inner product space that is also a complete metric space with respect to the
  distance function induced by the inner product.''
\end{definition}

\begin{definition}\label{def:conjugate-transpose}
  The \term{conjugate transpose} of a matrix $M = (M_{i,j}) \in \C^{m \cross
  n}$, written $M^\dag$, is given by taking the complex conjugate of each
  element of $M^\T$; in other words, $M^\dag = (\overline{M_{j,i}})$.  A matrix
  $M$ is \term{Hermitian} if $M = M^\dag$.
\end{definition}

\begin{definition}\label{def:unitary-matrix}
  A \term{unitary matrix} is a square matrix $U \in \C^{n \cross n}$ such that
  $U^\dag U = I$.\asz{Wikipedia defines a unitary \emph{transformation} as an
  isomorphism of Hilbert spaces, or a linear transformation that preserves the
  inner product.  Should we mention this?}
\end{definition}

We also need to briefly touch on notation.  We adopt the physicists' convention
of writing $\ket{\psi}$ for an element of a Hilbert space $H$; this is referred
to as a \term{ket}.  The dual of a ket is a \term{bra}, written $\bra{\phi}$,
which is an element of $H$'s dual space.  When working with vectors in $\C^n$,
as we will be doing, we can define $\bra{\psi} = \ket{\psi}^\dag$: where kets
are column vectors, bras are row vectors.  The notation $\braket{\phi | \psi}$
then denotes the inner product of $\ket{\phi}$ and $\ket{\psi}$, which is just
matrix multiplication in the case of $\C^n$.  Importantly, this means that we
have now also adopted the physicist's convention that the inner product is
linear in its \emph{second} argument, and conjugate-linear in its \emph{first}.

Fortunately, for quantum computation, we are only interested in very particular
Hilbert spaces.  These are the spaces that arise from considering the single
fundamental primitive datum of quantum computation: the \term{qubit}.  A qubit
is a \emph{qu}antum \emph{b}it: when we have a classical bit, we know that it is
either $0$ or $1$; similarly, when we measure a qubit, we know that we will get
either $\ket{0}$ or $\ket{1}$.  The difference between a qubit and a bit lies in
the word ``measure'': as long as we \emph{don't} measure a qubit, it can take on
all sorts of crazy states.  Formally, we can represent a qubit as an element of
$\C^2$ subject to some normalization constraints:

\begin{definition}\label{def:qubit}
  A \term{qubit} is a vector of two complex numbers \[ \vector{a b} \in \C^2 \]
  such that $|a|^2 + |b|^2 = 1$.  We define $\ket{0}$ and $\ket{1}$ to be the
  basis vectors \[ \ket{0} = \vector{1 0} \text{ and } \ket{1} = \vector{0
  1}, \] and will thus often write the above qubit as $a\ket{0} + b\ket{1}$.  We
  call $a$ and $b$ \term{probability amplitudes}, for reasons which will become
  clearer later.
\end{definition}

Now that we have this definition, we can define our first unitary operator:
bitwise negation, which we define as \[
  X = \begin{pmatrix} 0 & 1 \\
                      1 & 0 \end{pmatrix}.
\] Seeing that $X$ is unitary is straightforward, since it is both self-inverse
and Hermitian.  It is also easy to see that $X\ket{0} = \ket{1}$ and $X\ket{1} =
\ket{0}$, justifying the name bitwise negation.  But qubits are more than just
zeroes and ones.  If we apply bitwise negation to an arbitrary qubit $a\ket{0} +
b\ket{1}$, we get \[
  X(a\ket{0} + b\ket{1}) =
  \begin{pmatrix}
    0 & 1 \\
    1 & 0
  \end{pmatrix}
  \vector{a b} =
  \vector{b a} =
  b\ket{0} + a\ket{1}.
\]  Thus, bitwise negation swaps the probability amplitudes of an arbitrary
qubit.

\begin{figure}
  \centerline{\Qcircuit{
    \lstick{\ket{0}} & \gate{X} & \rstick{\ket{1}} \qw
  }}
  \caption{The quantum circuit diagram for applying the bitwise not operator to
    a single qubit; here, it is applied to the value $\ket{0}$.}
  \label{qcd:bitwise-not-0}
\end{figure}

We can also represent applying the unitary transformation $X$ diagrammatically
through the use of a \term{quantum circuit diagram}, such as we see in
\cref{qcd:bitwise-not-0}.  The horizontal lines are \term{wires}, and are read
from left to right.  A box which spans some wires and contains a term is a
\term{gate}; the effect of this gate is given by the unitary matrix within the
box, and the whole thing is to be read as applying that matrix to the given
qubits.\footnotemark{}  The qubit labels at the left ends of the wires are the
input values; the qubit labels at the right end are the output values.
Sometimes, we will give a qubit a name as well as an initial value; the name
will be placed even further left.

\footnotetext{Our mathematically inclined readers might notice here that this is
  reminiscent of category theoretic diagrams, including composition happening in
  diagrammatic order.  This is indeed the case, which will be explained in more
  detail in \cref{sec:how}.}
