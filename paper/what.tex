The behavior of quantum mechanics is described by Hilbert spaces, which
evolve via unitary transformations. In this section we will develop
some of the quantum mechanical background necessary to understand the quantum
teleportation algorithm. We expect the user to have some knowledge of vector spaces and
linear transformations with respect to vectors and matrices. For further
background on linear algebra, see \cite{gallier13}.

\begin{definition}[Hilbert space]\label{def:hilbert-space}
  A \term{Hilbert space} is a complex vector space equipped with an inner product
  $\braket{\varphi | \psi}$
  such that the distance function induced by the inner product is a complete
  metric space.
\end{definition}

In this section we will only be considering spaces of the form $\C^m$.
In this context, matrices fully generalize linear transformations.

\begin{definition}[Conjugate transpose]\label{def:conjugate-transpose}
  The \term{conjugate transpose} of a matrix $M = (M_{i,j}) \in \C^{m \cross
  n}$, written $M^\dag$, is given by taking the complex conjugate of each
  element of $M^\T$; in other words, $M^\dag = (\overline{M_{j,i}})$.
\end{definition}

\begin{definition}[Hermitian matrix]\label{def:hermitian}
  A matrix $M$ is \term{Hermitian} if $M = M^\dag$.
\end{definition}

\begin{definition}[Unitary matrix]\label{def:unitary-matrix}
  A \term{unitary matrix} is a square matrix $U \in \C^{n \cross n}$ such that
  $U^\dag U = I$. 
\end{definition}
A property of unitary matrices is that they preserve the inner product:
\[
    \braket{U \varphi | U \psi} = \braket{\varphi | \psi}.
\]

We also need to briefly touch on notation.  We adopt the physicists' convention
of writing $\ket{\psi}$ for an element of a Hilbert space $H$; this is referred
to as a \term{ket}. In $\C^n$, a ket is a column vector.
The dual of a ket is a \term{bra}, written $\bra{\phi}$,
which is a corresponding row vector; in fact we can define $\bra{\psi} = \ket{\psi}^\dag$.
The inner product of two column vectors, then, is exactly
\[ \braket{\phi | \psi} = \bra{\phi} \mtimes \ket{\psi} = \ket{\phi}^\dag \ket{\psi} \]
where $\mtimes$ corresponds to matrix multiplication. Importantly, this means that we
have now also adopted the physicist's convention that the inner product is
linear in its \emph{second} argument, and conjugate-linear in its \emph{first}.

For quantum computation, we are only interested in very particular
Hilbert spaces---the spaces that arise from considering the \term{qubit}, the single
primitive datum of quantum computation.  A qubit
is a \emph{qu}antum \emph{b}it: when we have a Newtonian\footnotemark{} bit, we
know that it is either $0$ or $1$; similarly, when we measure a qubit, we know
that we will get either $\ket{0}$ or $\ket{1}$.  The difference between a qubit
and a bit lies in the word ``measure'': as long as we \emph{don't} measure a
qubit, it can take on all sorts of crazy states.  Formally, we can represent a
qubit as an element of $\C^2$ subject to some normalization constraints.

\begin{definition}[Qubit]\label{def:qubit}
  A \term{qubit} is a vector of two complex numbers \[ \vector{\alpha \beta} \in \C^2 \]
  such that $|\alpha|^2 + |\beta|^2 = 1$.  In general, we say that two qubits are the
  same if they differ only by a constant factor (even if one is non-normalized).
  We define $\ket{0}$ and $\ket{1}$ to be the basis vectors \[ \ket{0} =
  \vector{1 0} \text{ and } \ket{1} = \vector{0 1}, \] and will thus often write
  the above qubit as $\alpha\ket{0} + \beta\ket{1}$.  The general, non-basis, qubit
  states, where $\alpha \ne 0$ and $\beta \ne 0$, are termed \term{superpositions}.  We
  call the coefficients $\alpha$ and $\beta$ \term{probability amplitudes}, for reasons
  which will become clear later.
\end{definition}

\footnotetext{The usual term for this is \term{classical}: we contrast quantum
  bits with classical bits, quantum computation with classical computation, and
  in general quantum physics with classical physics.  However, in computer
  science, we use the term \term{classical} to refer to classical logic (that
  is, logic that affirms the law of the excluded middle).  Quantum computation
  is as constructive as they come; the overloading of the term classical is
  merely an (unfortunate) coincidence.  To avoid ambiguity, we cleave to the
  computer scientists' convention of using ``classical'' only for classical
  logic, and refer to all our non-quantum computational models as Newtonian
  instead of classical.}

Now that we have this definition, we can define our first unitary operator:
bitwise negation, which we define as \[
  X = \begin{pmatrix} 0 & 1 \\
                      1 & 0 \end{pmatrix}.
\] Seeing that $X$ is unitary is straightforward, since it is both self-inverse
and Hermitian.  It is also easy to see that $X\ket{0} = \ket{1}$ and $X\ket{1} =
\ket{0}$, justifying the name bitwise negation.  But qubits are more than just
zeroes and ones.  If we apply bitwise negation to an arbitrary qubit $\alpha\ket{0} +
\beta\ket{1}$, we get \[
  X(\alpha\ket{0} + \beta\ket{1}) =
  \begin{pmatrix}
    0 & 1 \\
    1 & 0
  \end{pmatrix}
  \vector{\alpha \beta} =
  \vector{\beta \alpha} =
  \beta\ket{0} + \alpha\ket{1}.
\]  Thus, bitwise negation swaps the probability amplitudes of an arbitrary
qubit.

\begin{figure}
\centering
\begin{subfigure}{.4\textwidth}
  \centerline{\Qcircuit{
    \lstick{\ket{0}} & \gate{X} & \rstick{\ket{1}} \qw
  }}
  \caption{Bitwise not operator applied to the single qubit $\ket{0}$.}
  \label{qcd:bitwise-not-0}
\end{subfigure}
\qquad
\begin{subfigure}{.4\textwidth}
  \centerline{\Qcircuit{
    \lstick{\ket{0}} &
    \gate{H} &
    \rstick{\displaystyle\frac{\ket{0}+\ket{1}}{\sqrt{2}}} \qw
  }}
  \caption{Hadamard transform applied to the single qubit $\ket{0}$.}
  \label{qcd:hadamard-0}
\end{subfigure}
\caption{Basic quantum circuit diagrams.}
\label{qcd:example-diag}
\end{figure}

Despite this ability to apply bitwise negation to non-basis vectors, it is still
essentially a traditional computational operation.  Can we define an operation
that will give us access to the more interesting qubit states we have opened
ourselves up to?  For this, we can define the \term{Hadamard transform} \[
  H = \frac{1}{\sqrt{2}}
      \begin{pmatrix} 1 & \pos1 \\
                      1 &    -1 \end{pmatrix}.
\] This matrix is also unitary since it is also self-inverse and Hermitian,
although this time we need a scaling factor of $1/\sqrt{2}$ to ensure that.  The
Hadamard transform will turn either of the basis vectors $\ket{0}$ or $\ket{1}$
into evenly-weighted superpositions of $\ket{0}$ and $\ket{1}$:
\begin{gather*}
  H\ket{0}
  = \frac{1}{\sqrt{2}}\begin{pmatrix} 1 & \pos1 \\ 1 & -1 \end{pmatrix}
    \vector{1 0}
  = \frac{1}{\sqrt{2}} \vector{1 1}
  = \frac{\ket{0} + \ket{1}}{\sqrt{2}}
\intertext{and}
  H\ket{1}
  = \frac{1}{\sqrt{2}}\begin{pmatrix} 1 & \pos1 \\ 1 & -1 \end{pmatrix}
    \vector{0 1}
  = \frac{1}{\sqrt{2}} \vector{\pos1 -1}
  = \frac{\ket{0} - \ket{1}}{\sqrt{2}}.
\end{gather*}
In the general case, we have \[
  H(a\ket{0} + b\ket{1})
  = \frac{1}{\sqrt{2}}\begin{pmatrix} 1 & \pos1 \\ 1 & -1 \end{pmatrix}
    \vector{\alpha \beta}
  = \vector{\alpha+\beta \alpha-\beta}
  = \frac{(\alpha+\beta)\ket{0} + (\alpha-\beta)\ket{1}}{\sqrt{2}},
\]
so the Hadamard transform distributes the probability amplitudes of an arbitrary
qubit across both of its basis components.


We can also represent applying unitary transformations diagrammatically through
the use of a \term{quantum circuit diagram}, such as we see in
\cref{qcd:bitwise-not-0,qcd:hadamard-0}; \cref{qcd:bitwise-not-0} represents the
computation $X\ket{0} = \ket{1}$ and \cref{qcd:hadamard-0} represents the
computation $H\ket{0} = (\ket{0} + \ket{1})/\sqrt{2}$.  In these diagrams, the
horizontal lines are \term{wires} representing single qubits, and are read from
left to right.  A box which spans some wires and contains a term is a
\term{gate}; the effect of this gate is given by the unitary matrix within the
box, and the whole thing is to be read as applying that matrix to the given
qubits.\footnotemark{} The qubit labels at the left ends of the wires are the
input values (customarily either $\ket{0}$ or an input to the circuit); the
qubit labels at the right end are the output values.  Sometimes, we will give a
qubit a name as well as an initial value; the name will be placed even further
left.

\footnotetext{Our mathematically inclined readers might notice here that this is
  reminiscent of category theoretic diagrams, including composition happening in
  diagrammatic order.  This is indeed the case, which will be explained in more
  detail in \cref{sec:how}.}

This brings us nicely to the next property we must discuss: we talk above about
wire\emph{s}, plural.  But we haven't yet said how to define multi-qubit states!
Given two qubits $\ket{\phi}$ and $\ket{\psi}$, where $\ket{\phi} =
\phi_0\ket{0} + \phi_1\ket{1}$ and $\ket{\psi} = \psi_0\ket{0} + \psi_1\ket{1}$,
how can we talk about the system consisting of both $\ket{\phi}$ and
$\ket{\psi}$?  To those of us used to the traditional Newtonian worldview, this
is an odd question; can't we just use the pair $(\ket{\phi},\ket{\psi})$?  After
all, vector spaces are closed under the Cartesian product.

But here, too, quantum mechanics departs from our intuitions.  In quantum
mechanics, when we combine two state spaces, we take their \emph{tensor} product
instead.  For a fully formal definition of the tensor product
consult \cite{gallier13}; we will give some of the relevant operations for the
finite-dimensional vector spaces here:
\begin{enumerate}
    \item For arbitrary $u_1 \in H_1$ and $u_2 \in H_2$, 
    $u_1 \tensor u_2$ is the equivalence class in $H_1 \tensor H_2$ induced
    by bilinearity:
    \begin{align*}
        \alpha(u_1 \tensor u_2) &= (\alpha u_1) \tensor u_2 \\
        \alpha(u_1 \tensor u_2) &= u_1 \tensor (\alpha u_2) \\
        (u_1 \tensor v) + (u_2 \tensor v) &= (u_1 + u_2) \tensor v \\
        (u \tensor v_1) + (u \tensor v_2) &= u \tensor (v_1 + v_2)
    \end{align*}
    \item If $B_1$ is a basis of $H_1$ and $B_2$ is a basis of $H_2$
    then $B_1 \times B_2$ is a basis for $H_1 \tensor H_2$.
\end{enumerate}

\begin{definition}[Tensor product]\label{def:tensor-product}
  The \term{tensor product of vectors} $\ket{\phi} \in \C^m$ and $\ket{\psi} \in
  \C^n$ with zero-indexed entries $\phi_i$ and $\psi_j$, respectively, is the
  vector in $\C^{mn}$ whose entries are the pairwise products of the entries in
  $\ket{\phi}$ and the entries in $\ket{\psi}$.  More precisely, we have \[
    \ket{\phi} \tensor \ket{\psi} =
    \vector{\phi_0\ket{\psi} \phi_1\ket{\psi} \vdots{} \phi_{m-1}\ket{\psi}} =
    \vector{\phi_0\psi_0 \vdots{} \phi_0\psi_{n-1}
            \phi_1\psi_0 \vdots{} \phi_1\psi_{n-1}
            \vdots{}
            \phi_{m-1}\psi_0 \vdots{} \phi_{m-1}\psi_{n-1}}
    \in \C^{mn}.
  \]  We will also write this as $\ket{\phi}\ket{\psi}$ or $\ket{\phi\psi}$ if there
  is no ambiguity.
 

  % "Internal" to the submatrix -- spacing is hard to come by here.
  \def\icdots{\mathclap{\cdots}}
  \def\iddots{\mathclap{\ddots}}
  The \term{tensor product of matrices} $U \in \C^{m \cross n}$ and $V \in \C^{p
  \cross q}$ with one-indexed\footnotemark{} entries $u_{ij}$ and $v_{ij}$,
  respectively, is the $mp \cross nq$ matrix whose entries are the pairwise
  products of the entries in $U$ and the entries in $V$.  More precisely, we
  have 
  \begin{gather*}
    U \tensor V =
    \begin{pmatrix}
      u_{00}V & u_{10}V & \cdots & u_{m'0}V \\
      u_{01}V & u_{11}V & \cdots & u_{m'1}V \\
      \vdots & \vdots & \ddots & \vdots \\
      u_{0n'}V & u_{1n'}V & \cdots & u_{m'n'}V
    \end{pmatrix} \\ =
    \left(\begin{array}{@{}*{2}{c@{\;\;}ccc}cc@{\;\;}ccc@{}}
      u_{00}v_{00} & \icdots & u_{00}v_{p'0} &
      u_{10}v_{00} & \icdots & u_{10}v_{p'0} &
      \cdots &
      u_{m'0}v_{00} & \icdots & u_{m'0}v_{p'0}
      \\
      \vdots & \iddots & \vdots &
      \vdots & \iddots & \vdots &
      \cdots &
      \vdots & \iddots & \vdots
      \\
      u_{00}v_{0q'} & \icdots & u_{00}v_{p'q'} &
      u_{10}v_{0q'} & \icdots & u_{10}v_{p'q'} &
      \cdots &
      u_{m'0}v_{0q'} & \icdots & u_{m'0}v_{p'q'}
      \\
      \vdots & \vdots & \vdots &
      \vdots & \vdots & \vdots &
      \ddots &
      \vdots & \vdots & \vdots
      \\
      u_{0n'}v_{00} & \icdots & u_{0n'}v_{p'0} &
      u_{1n'}v_{00} & \icdots & u_{1n'}v_{p'0} &
      \cdots &
      u_{m'n'}v_{00} & \icdots & u_{m'n'}v_{p'0}
      \\
      \vdots & \iddots & \vdots &
      \vdots & \iddots & \vdots &
      \cdots &
      \vdots & \iddots & \vdots
      \\
      u_{0n'}v_{0q'} & \icdots & u_{0n'}v_{p'q'} &
      u_{1n'}v_{0q'} & \icdots & u_{1n'}v_{p'q'} &
      \cdots &
      u_{m'n'}v_{0q'} & \icdots & u_{m'n'}v_{p'q'}
    \end{array}\right).% \\ \in \C^{mp \cross nq} % Add back in if there's space
  \end{gather*}
  where $m'=m-1$, $n'=n-1$, $p'=p-1$ and $q'=q-1$.

  \footnotetext{Solely for space reasons.}
\end{definition}

Thus, when we are considering $n$-qubit systems, we find ourselves in a
$2^n$-dimensional complex vector space.  The basis vectors in this space are the
tensor products of the basis vectors in the original space; for instance, a
two-qubit system has the four basis vectors \[
\begin{array}{c@{\qquad}c}
  \ket{00} = \ket{0} \tensor \ket{0} =
  \vector{1 0} \tensor \vector{1 0} = \vector{1 0 0 0} &
  \ket{01} = \ket{0} \tensor \ket{1} =
  \vector{1 0} \tensor \vector{0 1} = \vector{0 1 0 0} \\[7ex]
  \ket{10} = \ket{1} \tensor \ket{0} =
  \vector{0 1} \tensor \vector{1 0} = \vector{0 0 1 0} &
  \ket{11} = \ket{1} \tensor \ket{1} =
  \vector{0 1} \tensor \vector{0 1} = \vector{0 0 0 1}
\end{array}
\]  Conveniently, this means that we can take the basis vector $\ket{b_{n-1}
\cdots b_0}$, where each of the $b_i$ are either $0$ or $1$, and read the name
$b_{n-1} \cdots b_0$ as a binary number $B$; if we do this, we find that this
basis vector is the one containing $0$s everywhere except for a $1$ at
(zero-based) index $B$.

Now that we can consider multi-qubit states, we can revisit the no-cloning
theorem mentioned in \cref{sec:why}.  The no-cloning theorem says there is no
way to produce two copies of a single arbitrary quantum state.
With tensor products, we can state and prove this theorem formally.

\begin{theorem}[No-Cloning \citep{wootters1982single,dieks1982communication}]\label{thm:no-cloning}
  There is no unitary transformation that duplicates an arbitrary qubit; that
  is, there does not exist a unitary matrix $U \in \C^{2 \cross 2}$ such that \[
  U(\ket{0}\tensor\ket{\psi}) = \ket{\psi}\tensor\ket{\psi}.\]
\end{theorem}
\begin{proof}
  Suppose there was such a transformation.
  Consider an arbitrary qubit $\ket{\psi} = \alpha\ket{0} + \beta\ket{1}$. 
  By the duplication property of $U$, we know that
  \[ U(\ket{0} \tensor \ket{\psi}) = \ket{\psi} \tensor \ket{\psi},
  \] 
  which expands out to 
  \[
      U(\ket{0} \tensor \ket{\psi})
    = (\alpha\ket{0} + \beta\ket{1}) \tensor (\alpha\ket{0} + \beta\ket{1})
    = \alpha^2    \ket{00}
    + \alpha\beta \ket{01}
    + \alpha\beta \ket{10}
    + \beta^2     \ket{11}.
  \]  On the other hand, since all unitary transformations are linear, we can
  expand out our definition of $\ket{\psi}$ to get that \[
      U(\ket{0} \tensor \ket{\psi})
    = \alpha U(\ket{0} \tensor \ket{0}) + \beta U(\ket{0} \tensor \ket{1}),
  \] and we can use our assumption that $U$ duplicates qubits to rewrite this
  as \[
    U(\ket{0} \tensor \ket{\psi}) = \alpha \ket{00} + \beta \ket{11}.
  \]  Putting these equalities together, we have \[
      \alpha^2    \ket{00}
    + \alpha\beta \ket{01}
    + \alpha\beta \ket{10}
    + \beta^2     \ket{11}
    = \alpha \ket{00} + \beta \ket{11},
  \] which we can break apart into \[
    \alpha^2 = \alpha,\ \beta^2 = \beta,\text{ and } \alpha\beta = 0.
  \]  Thus, either $\alpha = 0$ and $\beta = \pm1$, or $\beta = 0$ and $\alpha =
  \pm1$.  Consequently, the only qubits which can be duplicated by a unitary
  transformation are $\pm\ket{0}$ and $\pm\ket{1}$.
\end{proof}

There is another way to see that $U$ does not exist for arbitrary qubits. In the
first expansion of $U(\ket{0} \tensor \ket{\psi})$, we can factor the state into two
1-qubit states:
\[ (\alpha\ket{0} + \beta\ket{1}) \tensor (\alpha\ket{0} + \beta\ket{1}). \]
However it is impossible to perform such a factorization for the second expansion,
$\alpha\ket{00} + \beta\ket{11}$.

%% The tensor product has another important property besides allowing us to state
%% and prove the no-cloning theorem and increasing the dimension of the vector
%% spaces involved multiplicatively instead of additively.  We know that given
%% vector spaces $U$ and $V$, all elements of $U \cross V$ are of the form $(u,v)$
%% for some $u \in U$ and $v \in V$.  But when we take the tensor product of $U$
%% and $V$ instead, we find that there exist elements of $U \tensor V$ that aren't
%% of the form $u \tensor v$ for any $u \in U$ and $v \in V$.  We call states of
%% the form $u \tensor v$ \term{separable}, and all other states \term{entangled}.

\begin{definition}[Separable and entangled states]%
\label{def:separable-entangled}
  A multi-qubit system is \term{separable} if it can be written as the tensor
  product of multiple smaller systems, and \term{entangled} if it cannot.  A
  state may be separable in one context but not in another; for instance, given
  the entangled two-qubit states $\ket{\phi}$ and $\ket{\psi}$, we may say that
  $\ket{\phi\psi}$ is entangled when thinking about one-qubit states, but that
  it is separable when thinking about two-qubit states.
\end{definition}

This is all well and good, but how do we produce entangled states?  To do this,
we need to introduce our first multi-qubit operator: the \term{controlled not}
gate, also known as the \term{cnot} gate.  You might expect a more traditional
choice like an and gate or an or gate.  However, recall that all quantum
operators must be unitary, and hence take $n$ qubits to $n$ qubits; ``and'' and
``or'' lose information.  The controlled not gate takes two qubits as input: a
control qubit and a target qubit.  The control qubit is unmodified; the second
qubit is bitwise negated if the control qubit is $\ket{1}$, and unmodified if
the control qubit is $\ket{0}$.  We can define this with the matrix \[
  \CNOT = \begin{pmatrix} 1 & 0 & 0 & 0 \\
                          0 & 1 & 0 & 0 \\
                          0 & 0 & 0 & 1 \\
                          0 & 0 & 1 & 0 \end{pmatrix}.
\] Seeing that $\CNOT$ is unitary is straightforward, since (like $X$ and $H$)
it is both self-inverse and Hermitian.  We can verify that, given a single-qubit
state $\ket{\psi} = \psi_0\ket{0} + \psi_1\ket{1}$, we have $\CNOT\ket{0\psi} =
\ket{0\psi}$ and $\CNOT\ket{1\psi} = \ket{1} \tensor X\ket{\psi}$:
\begin{align*}
  \CNOT\ket{0\psi} &=
  \begin{pmatrix} 1 & 0 & 0 & 0 \\
                  0 & 1 & 0 & 0 \\
                  0 & 0 & 0 & 1 \\
                  0 & 0 & 1 & 0 \end{pmatrix}
  \left[\vector{1 0} \tensor \vector{\psi_0 \psi_1}\right] =
  \begin{pmatrix} 1 & 0 & 0 & 0 \\
                  0 & 1 & 0 & 0 \\
                  0 & 0 & 0 & 1 \\
                  0 & 0 & 1 & 0 \end{pmatrix}
  \vector{\psi_0 \psi_1 0 0} \\ &=
  \vector{\psi_0 \psi_1 0 0} =
  \vector{1 0} \tensor \vector{\psi_0 \psi_1} =
  \ket{0\psi}
\intertext{and}
  \CNOT\ket{1\psi} &=
  \begin{pmatrix} 1 & 0 & 0 & 0 \\
                  0 & 1 & 0 & 0 \\
                  0 & 0 & 0 & 1 \\
                  0 & 0 & 1 & 0 \end{pmatrix}
  \left[\vector{0 1} \tensor \vector{\psi_0 \psi_1}\right] =
  \begin{pmatrix} 1 & 0 & 0 & 0 \\
                  0 & 1 & 0 & 0 \\
                  0 & 0 & 0 & 1 \\
                  0 & 0 & 1 & 0 \end{pmatrix}
  \vector{0 0 \psi_0 \psi_1} \\ &=
  \vector{0 0 \psi_1 \psi_0} =
  \vector{0 1} \tensor \vector{\psi_1 \psi_0} =
  \ket{1} \tensor X\ket{\psi}.
\end{align*}
However, things get more interesting when the first qubit is not simply
$\ket{0}$ or $\ket{1}$.  For instance, consider what happens when applying the
controlled not gate to the control $(\ket{0} + \ket{1})/\sqrt{2}$ and the target
$\ket{0}$.  Informally, since the control is equally both $\ket{0}$ and
$\ket{1}$, it should equally leave the target alone and invert it.  But since
the inversion happens only with a control of $\ket{1}$, the two qubits will be
correlated: we should expect that either both are $\ket{0}$ or both are
$\ket{1}$.  And indeed, when we work this out, we get the state $(\ket{00} +
\ket{11})/\sqrt{2}$:
\begin{align*}
  \CNOT\left[\frac{\ket{0} + \ket{1}}{\sqrt{2}} \tensor \ket{0}\right] &=
  \begin{pmatrix} 1 & 0 & 0 & 0 \\
                  0 & 1 & 0 & 0 \\
                  0 & 0 & 0 & 1 \\
                  0 & 0 & 1 & 0 \end{pmatrix}
  \left[\frac{1}{\sqrt{2}}\vector{1 1} \tensor \vector{1 0}\right] =
  \begin{pmatrix} 1 & 0 & 0 & 0 \\
                  0 & 1 & 0 & 0 \\
                  0 & 0 & 0 & 1 \\
                  0 & 0 & 1 & 0 \end{pmatrix}
  \left[\frac{1}{\sqrt{2}} \vector{1 0 1 0} \right] \\ &=
  \frac{1}{\sqrt{2}} \vector{1 0 0 1} =
  \frac{\ket{00} + \ket{11}}{\sqrt{2}}.
\end{align*}
And this state's correlation -- guaranteeing that both qubits must be the same,
but \emph{not} what value they must take -- means that it is entangled.  


\begin{figure}
  \centerline{\Qcircuit{
    \lstick{\ket{1}} & \ctrl{1} & \rstick{\ket{1}} \qw \\
    \lstick{\ket{0}} & \targ    & \rstick{\ket{1}} \qw \\
  }}
  \caption{The quantum circuit diagram for applying the controlled not gate to
    two qubits; here, it is applied to the initial state $\ket{10}$.}
  \label{qcd:cnot-00}
\end{figure}

\begin{figure}
  \centerline{\Qcircuit{
    \lstick{\ket{0}} &
      \gate{H} &
      \qw & \qw \qcmark{2} & \qw &
      \ctrl{1} &
    \rstick{\tikz[overlay]
      \draw[decorate,decoration={brace, amplitude=1em}]
      (0,1em) -- (0,-4.1em)
      node[midway,right=1.5em]
        {$\displaystyle \frac{\ket{00} + \ket{11}}{\sqrt{2}}$} ;} \qw
    \\
    \lstick{\ket{0}} &
      \qw &
      \qw & \dstick{\raisebox{-2.67em}{$\displaystyle
              \frac{\ket{0} + \ket{1}}{\sqrt{2}} \tensor \ket{0}$}} \qw &
            \qw &
      \targ &
    \qw \\
    & \\ & % For space -- but TODO, a little too much!
  }}
  \caption{The quantum circuit diagram for applying the controlled not gate to
    two qubits; here, it is applied to the initial state $\ket{00}$.}
  \label{qcd:entangle-2-equal}
\end{figure}

Diagrammatically, we represent the controlled not gate, not with a box, but with
a \term{control}~``\QcircuitDemo[\,]{\control}'' and a
\term{target}~``\QcircuitDemo{\qctarget}''; this represents applying $\CNOT$ to
the two-qubit state $\ket{\mathit{control}}\ket{\mathit{target}}$.
\Cref{qcd:cnot-00} shows what this looks like when evaluating $\CNOT\ket{00}$.
This is a no-op, so the diagram is pretty boring; \cref{qcd:entangle-2-equal}
instead shows how to use the controlled not gate to entangle two qubits so that
they must be equal; specifically, into the state $(\ket{00} +
\ket{11})/\sqrt{2}$.  First, we use the Hadamard gate to put the first qubit
into the state $(\ket{0} + \ket{1})/\sqrt{2}$; then, we use the controlled not
gate to entangle the qubit, as we discussed above.

If we wanted to write this circuit out as a matrix, though, what would we do?
We have applied the Hadamard gate only to the first qubit, even though we have a
two-qubit quantum state.  The answer is that when we apply a unitary matrix $U$
to some qubits, we tensor it with appropriately sized identity matrices to get a
matrix that acts on all the qubits at once.  This works because $(U \tensor
V)(\ket{\phi} \tensor \ket{\psi}) = U\ket{\phi} \tensor V\ket{\psi}$; this is
easy to see, but we omit the proof, as it is entirely straightforward but messy
algebra.

Consequently, to entangle two $\ket{0}$ qubits, we apply the matrix $\CNOT
\mtimes (H \tensor I_2)$ (remember, diagrammatic composition goes the opposite
order from the composition or multiplication operator).  Applying $H \tensor
I_2$ produces the state marked at the dashed line; applying $\CNOT$ entangles
the two qubits.  Working out the matrix algebra directly, we get
\begin{align*}
  \CNOT \mtimes (H \tensor I_2) \mtimes \ket{00}
  &= \begin{pmatrix}
       1 & 0 & 0 & 0 \\
       0 & 1 & 0 & 0 \\
       0 & 0 & 0 & 1 \\
       0 & 0 & 1 & 0
     \end{pmatrix}
     \left[
       \frac{1}{\sqrt{2}}
       \begin{pmatrix}
         1 & \pos1 \\
         1 &    -1
       \end{pmatrix}
       \tensor
       \begin{pmatrix}
         1 & 0 \\
         0 & 1
       \end{pmatrix}
     \right]
     \ket{00} \\
  &= \frac{1}{\sqrt{2}}
     \begin{pmatrix}
       1 & 0 & 0 & 0 \\
       0 & 1 & 0 & 0 \\
       0 & 0 & 0 & 1 \\
       0 & 0 & 1 & 0
     \end{pmatrix}
     \begin{pmatrix}
       1 & 0 & \pos1 & \pos0 \\
       0 & 1 & \pos0 & \pos1 \\
       1 & 0 &    -1 & \pos0 \\
       0 & 1 & \pos0 &    -1
     \end{pmatrix}
     \ket{00} \\
  &= \frac{1}{\sqrt{2}}
     \begin{pmatrix}
       1 & 0 & \pos1 & \pos0 \\
       0 & 1 & \pos0 & \pos1 \\
       0 & 1 & \pos0 &    -1 \\
       1 & 0 &    -1 & \pos0
     \end{pmatrix}
     \ket{00} \\
  &= \frac{1}{\sqrt{2}}
     \begin{pmatrix}
       1 & 0 & \pos1 & \pos0 \\
       0 & 1 & \pos0 & \pos1 \\
       0 & 1 & \pos0 &    -1 \\
       1 & 0 &    -1 & \pos0
     \end{pmatrix}
     \vector{1 0 0 0} \\
  &= \frac{1}{\sqrt{2}}\vector{1 0 0 1}
   = \frac{\ket{00} + \ket{11}}{\sqrt{2}}.
\end{align*}

While there is much more we could say about quantum mechanics, most of it is
outside the scope of the paper.  Only one important concept remains:
measurement.  When we work with qubits, we can put them into all sorts of
superpositions and entangled states, but we claimed above that all measurements
would produce either $\ket{0}$ or $\ket{1}$.  The way this works is that quantum
measurements are \emph{probabilistic}.

\begin{definition}[Measurement]\label{def:single-qubit-measurement}
  When we \term{measure} an $n$-qubit state \[
    \ket{\psi} = \psi_{0\cdots0}\ket{0\cdots0}
               + \psi_{0\cdots1}\ket{0\cdots1}
               + \cdots
               + \psi_{1\cdots1}\ket{1\cdots1},
  \] the result is one of the basis vectors; the probability that the result is
  the basis vector $\ket{\vec{B}}$ is $|\psi_{\vec{b}}|^2$ for all $\vec{b} \in
  \Z_2^n$.
\end{definition}

For a single qubit state, $\psi_0\ket{0} + \psi_1\ket{1}$ results in $\ket{0}$
with probability $|\psi_0|^2$ and $\ket{1}$ with probability $|\psi_1|^2$. 
Because $|\psi_0|^2 + |\psi_1|^2 = 1$ by \cref{def:qubit}, the probability
space for this measurement is just $\ket{0}$ or $\ket{1}$.
This also justifies the name
``probability amplitudes'' for $\psi_0$ and $\psi_1$.  It is easy to check that
the tensor product preserves the property that the sum of the squares of the
norms of all the coefficients sum to one, so this generalizes to multi-qubit
states.  Note that there exist distinct qubits which, when measured, look the
same; for instance, $H\ket{0} = (\ket{0} + \ket{1})/\sqrt{2}$ and $H\ket{1} =
(\ket{0} - \ket{1})/\sqrt{2}$.  Since $(1/\sqrt{2})^2 = (-1/\sqrt{2})^2 = 1/2$,
both of these qubits are equally likely to be $\ket{0}$ or $\ket{1}$ when
measured; however, because one has a negative coefficient for $\ket{1}$, they
transform differently in the general case.

\begin{figure}
  \centerline{\Qcircuit{
    \lstick{\ket{0}} &
    \gate{H} &
    \measureD{\metersym} &
    \rstick{\ket{b} \in \{\ket{0},\ket{1}\}} \cw
  }}
  \caption{The quantum circuit diagram for flipping a fair coin; this is the
    circuit in \cref{qcd:hadamard-0} with a measurement at the end.}
  \label{qcd:fair-coin}
\end{figure}

Given this, we can define a circuit to simulate flipping a fair coin by adding a
measurement to the circuit in \cref{qcd:hadamard-0}; the result is the circuit
in \cref{qcd:fair-coin}.  We notate measurement in a quantum circuit diagram
with a \textsf{D}-shaped box containing a stylized meter, ``\metersym''.  What
emerges from a measurement is a \emph{Newtonian} bit, either $\ket{0}$ or
$\ket{1}$; this is notated with a double wire,
``\QcircuitDemo{&\cw}''.

\begin{figure}
  \centerline{\Qcircuit{
    \lstick{\ket{0}} &
      \gate{H} & \ctrl{1} & \multimeasureD{1}{\metersym} &
    \rstick{\ket{b}} \cw
    \\
    \lstick{\ket{0}} &
      \qw & \targ & \ghost{\metersym} &
    \rstick{\ket{b}} \cw
  }}
  \caption{A quantum circuit diagram for simulating agreement on a single coin
    flip; this is the circuit in \cref{qcd:entangle-2-equal} with a two-qubit
    measurement at the end.}
  \label{qcd:fair-coin-agreement-local}
\end{figure}

Entanglement allows us to get something better, though: remote agreement on the
result of a coin flip.  The circuit in \cref{qcd:entangle-2-equal} sets up a
correlated entangled state; adding a measurement allows us to bring that
correlation into the Newtonian world.  The circuit in
\cref{qcd:fair-coin-agreement-local} does this by measuring both qubits after
the original circuit produces the entangled state $(\ket{00} +
\ket{11})/\sqrt{2}$.  Since the probability amplitudes of $\ket{00}$ and
$\ket{11}$ are equal and the probability amplitudes of $\ket{01}$ and $\ket{10}$
are zero, this measurement will either produce $\ket{00}$ or $\ket{11}$, and
will do so with equal probability.

\begin{figure}
  \centerline{\Qcircuit{
    \lstick{\ket{0}} &
      \gate{H} & \ctrl{1} & \measureD{\metersym} &
    \rstick{\ket{b}} \cw
    \\
    \lstick{\ket{0}} &
      \qw & \targ & \qw &
    \rstick{\ket{b}} \qw
  }}
  
  \caption{A quantum circuit diagram for simulating agreement between two
    spacelike separated parties on a single coin flip; this is the circuit in
    \cref{qcd:entangle-2-equal} with a single-qubit measurement at the end.}
  \label{qcd:fair-coin-agreement-separated}
\end{figure}

However, this isn't a particularly powerful demonstration of entanglement.  If
we have both qubits in the same place for measurement purposes, then we can just
as easily simulate a single coin flip and have both participants look at it.
Instead, the scenario we want to model is where Alice and Bob wish to agree on a
single coin flip, but to do so after Alice has blasted off in her rocket and
gotten far away from Bob.  In this situation, they can't communicate, so only
one of them will measure their qubit at a time; things are symmetric, so without
loss of generality, suppose Alice measures her qubit and Bob doesn't.  Then we
get the circuit in \cref{qcd:fair-coin-agreement-separated}, reading the top
wire as Alice and the bottom wire as Bob (and imagining a rocket ship after the
controlled not gate).  This circuit has a couple of interesting properties.
First, we see that Bob's qubit does not get notated with a Newtonian wire.  Bob
cannot know that Alice has measured her qubit, so he doesn't know that his qubit
is in the basis state $\ket{b}$ until he measures it.  Second, measuring Alice's
qubit changes Bob's qubit.  This makes sense: prior to the measurement, the
qubits were in the state $(\ket{00} + \ket{11})/\sqrt{2}$.  If Alice measures
her qubit to be $\ket{0}$, then the only possibility that remains for Bob's
qubit is to be $\ket{0}$ too, and similarly for $\ket{1}$.  However, to make
this notion formal, we must give an account of how measuring \emph{part} of an
entangled state works.

\begin{definition}[Partial measurement]\label{def:partial-measurement}
Any $n$-qubit state can be written as
\[
    \alpha \ket{0} \ket{\phi}
    + \beta \ket{1} \ket{\psi}
\]
where $|\alpha|^2 + |\beta|^2 = 1$ and $\ket{\phi}$ and $\ket{\psi}$ are both $(n-1)$-qubit states.
Then the result of measuring the first qubit is the state
$\ket{0} \ket{\phi}$ with probability $|\alpha|^2$ and
$\ket{1} \ket{\psi}$ with probability $|\beta|^2$.

To measure multiple bits, this processes is repeated recursively.
\end{definition}

Again, the intuitive meaning of this is clear: the probability that qubit $i$ is
$\ket{0}$ is just the sum of the probabilities that we end up in any of the
states where this is true, and measurement discards all the other cases.  
