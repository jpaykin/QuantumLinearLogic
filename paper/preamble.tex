\newif\ifdraft
\drafttrue

\usepackage[T1]{fontenc}
\usepackage[utf8]{inputenc}

\usepackage{lmodern}
\usepackage{microtype}
\usepackage{fullpage}
\usepackage{etoolbox}
\usepackage{amsmath,amsfonts,amsthm}
\usepackage{thmtools}
\usepackage{mathtools}

% The newclude package tries to define this command (but only when thmtools is
% included?)
\makeatletter
\def\restore@saved@for@newclude{%
  \undef\save@for@newclude
  \undef\restore@saved@for@newclude
}
\def\save@for@newclude#1{
  \cslet{save@old@\string#1}#1
  \let#1\relax
  \apptocmd\restore@saved@for@newclude
    {\letcs#1{save@old@\string#1}\csundef{save@old@\string#1}}%
    {}{}
}
\save@for@newclude\g@prependto@macro
\usepackage{newclude}
\restore@saved@for@newclude
\makeatother

\usepackage{natbib}
\setcitestyle{authoryear,round,semicolon,aysep={,},yysep={,}}

\usepackage{newunicodechar}
\newunicodechar{ψ}{\psi}

\input{Qcircuit}
%\let\bra\relax
%\let\ket\relax
\usepackage{braket}

\usepackage[svgnames]{xcolor}

\usepackage{tikz}
\usetikzlibrary{calc,shapes,arrows}
%\makeatletter
\newcommand*{\QCircuitSetup}[1]{%
  % #1 = length of wire.
  \def\wirelength{#1}%
  %
  \def\labeledket##1##2{##1 \quad \mathrlap{\ket{##2}}{\phantom{\ket{ψ}}}}%
  \tikzset{wire label/.style={minimum width=1.5em}}%
  %
  \tikzset{gate/.style={rectangle, minimum width=1em, draw=black, fill=white}}%
  \def\cgateradius{6pt}%
  %
  \def\hadamard at##1;{%
    \node[gate] at ##1 {$H$} ;%
  }%
  %
  \def\control at##1;{%
    \filldraw[black] ##1 circle (\cgateradius) ;%
  }%
  %
  \def\cnot at##1from##2to##3;{%
    \control at  (##1,##2) ;%
    \draw[black] (##1,##3) circle (\cgateradius) ;%
    \draw[black] (##1,##2) -- ($(##1,##3)+(0,-\cgateradius)$) ;%
  }%
  %
  \def\wire at##1for##2from##3to##4;{%
    \draw (0,##1)%
          node[wire label, left]  {$\labeledket{##2}{##3}$}%
          --%
          (\wirelength,##1)%
          node[wire label, right] {$\ket{##4}$} ;%
  }%
  %
  \def\swire at##1from##2;{%
    \draw (0,##1)%
          node[wire label, left] {$\ket{##2}$}%
          --%
          (\wirelength,##1) ;%
  }%
  %
  \tikzset{cwire/.style={double distance=2pt}}%
  %
  \def\finally from##1to##2state##3;{%
    \draw[decorate,decoration={brace, amplitude=1em},xshift=0.2cm]
      ($(\wirelength,##1)+(0,0.2)$)%
      --%
      ($(\wirelength,##2)+(0,-0.2)$)%
      node[midway,right=0.4cm] {$##3$} ;%
  }%
  %
  \def\send cbit at##1from##2to##3;{%
    \draw[cwire] (##1,##2) -- (##1,##3) ;%
    \control at (##1,##2) ;%
  }%
  %
  \def\qstate at##1below by##2from##3to##4state##5;{%
    \draw[loosely dashed, draw=gray]
      ($(##1,##3)+(0,1)$)%
      --%
      ($(##1,##4)+(0,-##2)$)%
      node[below, draw, solid, thin] {$##5$} ;%
  }%
}
\makeatother


\usepackage[section]{placeins} % Floats should stay in their section

\usepackage[ colorlinks
           , pagebackref
           , linktoc = all
           , pdfusetitle ]{hyperref}
\usepackage[all]{hypcap}
\usepackage{fix-natbib-hyperref} % From a TeX.SE answer
\renewcommand*{\backrefalt}[4]{%
  \ifcase #1 %
    No citations.%
  \or
    Cited on page #2.%
  \else
    Cited on pages #2.%
  \fi
}
\hypersetup{ linkcolor = DarkBlue
           , citecolor = DarkGreen
           , urlcolor  = DarkGoldenrod
           , filecolor = Orange
           , menucolor = DarkRed
           , runcolor  = OrangeRed }

% Cleveref (must be loaded late---after hyperref, no less!)
\usepackage[noabbrev,capitalize,nameinlink]{cleveref}
\newcommand{\creflastconjunction}{, and~}

\makeatletter

\declaretheoremstyle[
  headfont=\normalfont\bfseries,
  notefont=\normalfont\bfseries, notebraces={(}{)},
  bodyfont=\normalfont\itshape
]{myplain}

\declaretheorem[style=myplain]{theorem}
\declaretheorem[style=myplain]{lemma}
\declaretheorem[style=myplain]{definition}

\newcommand{\id}{\operatorname{id}}
\newcommand{\unit}{\mathbf{1}}
\newcommand{\zero}{\mathbf{0}}
\newcommand{\basis}{\operatorname{basis}}
\newcommand{\lift}[1]{\ulcorner #1 \urcorner}
\newcommand{\llower}[1]{\llcorner #1 \lrcorner}
\newcommand{\meas}{\operatorname{meas}}
\newcommand{\Qubit}{Q}

\newcommand{\asz}[1]{\ifdraft\textcolor{violet}{[ASZ: #1]}\fi}
\newcommand{\jp}[1]{\ifdraft\textcolor{orange}{[JP: #1]}\fi}

\newcommand*{\R}{\mathbb{R}}
\newcommand*{\C}{\mathbb{C}}
\newcommand*{\T}{\mathsf{T}}
\newcommand*{\bit}{\mathbb{B}}
\newcommand*{\qubit}{\mathbb{Q}}
\newcommand*{\tensor}{\otimes}
\newcommand*{\cross}{\times}

\let\picvector\vector
\let\vector\relax
\def\vector@end{\vector@end}
\def\vector@last{\vector@last}
\def\vector@grab#1:#2 #3\vector@end{%
  \def\vector@temp{#3}%
  \ifx\vector@temp\vector@last
    #2%
  \else
    #2#1\vector@grab#1:#3\vector@end
  \fi
}
  
\newcommand{\vector}[2][\\]{%
  \begin{pmatrix} \vector@grab#1:#2 \vector@last\vector@end \end{pmatrix}%
}

\newcommand*{\term}[1]{\emph{#1}}

\newcommand*{\latin}[1]{\textit{#1}}
\newcommand*{\IE}{\latin{i.e.}}
\newcommand*{\EG}{\latin{e.g.}}

% References
\newcommand\Figure[1]{Figure~\ref{fig:#1}}
\newcommand{\Section}[1]{Section~\ref{sec:#1}}
\newcommand{\Definition}[1]{Definition~\ref{def:#1}}
\newcommand{\Theorem}[1]{Theorem~\ref{thm:#1}}

\newcommand*\crefme[1]{\textcolor{red}{\{!#1!\}}}
\newcommand*\citeme[1]{\textcolor{red}{\{?#1?\}}}

\makeatother

\newcommand{\Hilb}{\text{Hilb}}

