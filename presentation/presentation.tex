\documentclass[mathserif,14pt]{beamer}
\usepackage{presentation-preamble}

% Preamble (fold)
\title{\LARGE Quantum Mechanics, Category Theory, and Linear Logic}
\author{~\\[\smallskipamount]\Large Jennifer~Paykin \and Antal~Spector-Zabusky}
\date{April 30, 2014}
% (end)

% Talk-specific formatting (fold)
\newcommand<>{\fade}[1]{\textcolor#2{black!50}{#1}}

\DeclarePairedDelimiter{\name}{\ulcorner}{\urcorner}
\DeclarePairedDelimiter{\coname}{\llcorner}{\lrcorner}

\usepackage{braket}
\newcommand*{\qubit}{Q}
\makeatletter
\newcommand*{\QCircuitSetup}[1]{%
  % #1 = length of wire.
  \def\wirelength{#1}%
  %
  \def\visibly##1<##2> ##3;{\visible<##2>{##1##3;}}%
  %
  \def\defvisibly##1##2;##3{%
    \expandafter\def\csname\string##1@impl\endcsname##2;{##3}%
    \expandafter\edef\csname\string##1@spec\endcsname{%
      \noexpand\visibly%
      \expandafter\noexpand\csname\string##1@impl\endcsname}%
    \edef##1{\noexpand\@ifnextchar<%
               \expandafter\noexpand\csname\string##1@spec\endcsname
               \expandafter\noexpand\csname\string##1@impl\endcsname}%
  }%
  %
  \def\labeledket##1##2{##1 \quad \mathrlap{\ket{##2}}{\phantom{\ket{ψ}}}}%
  \tikzset{wire label/.style={minimum width=1.5em}}%
  %
  \tikzset{gate/.style={rectangle, minimum width=1em, draw=black, fill=white}}%
  \def\cgateradius{6pt}%
  %
  \defvisibly\hadamard at##1;{%
    \node[gate] at ##1 {$H$} ;%
  }%
  %
  \def\control at##1;{%
    \filldraw[black] ##1 circle (\cgateradius) ;%
  }%
  %
  \defvisibly\cnot at##1from##2to##3;{%
    \control at  (##1,##2) ;%
    \draw[black] (##1,##3) circle (\cgateradius) ;%
    \draw[black] (##1,##2) -- ($(##1,##3)+(0,-\cgateradius)$) ;%
  }%
  %
  \def\wire at##1for##2from##3to##4;{%
    \draw (0,##1)%
          node[wire label, left]  {$\labeledket{##2}{##3}$}%
          --%
          (\wirelength,##1)%
          node[wire label, right] {$\ket{##4}$} ;%
  }%
  %
  \def\swire at##1from##2;{%
    \draw (0,##1)%
          node[wire label, left] {$\ket{##2}$}%
          --%
          (\wirelength,##1) ;%
  }%
  %
  \tikzset{cwire/.style={double distance=2pt}}%
  %
  \def\finally from##1to##2state##3;{%
    \draw[decorate,decoration={brace, amplitude=1em},xshift=0.2cm]
      ($(\wirelength,##1)+(0,0.2)$)%
      --%
      ($(\wirelength,##2)+(0,-0.2)$)%
      node[midway,right=0.4cm] {$##3$} ;%
  }%
  %
  \def\send cbit at##1from##2to##3;{%
    \draw[cwire] (##1,##2) -- (##1,##3) ;%
    \control at (##1,##2) ;%
  }%
  %
  \defvisibly\qstate at##1below by##2from##3to##4state##5;{%
    \draw[loosely dashed, draw=gray]
      ($(##1,##3)+(0,1)$)%
      --%
      ($(##1,##4)+(0,-##2)$)%
      node[below, draw, solid, thin] {$##5$} ;%
  }%
}
\makeatother
% (end)

\begin{document}

\frame{\titlepage}

\begin{frame}{Classical computation}
  We have \term{bits}, that are either $0$ or $1$\par

  \pause\vspace{3\bigskipamount}

  (We're still constructive!  Classical is the opposite of quantum.)
\end{frame}

\begin{frame}{Quantum computation}
  We have \term{qubits}, which are always measured to be either $\ket{0}$ or
  $\ket{1}$.

  \pause\vspace{3\bigskipamount}

  So what's different?
  \begin{enumerate}
    \item Superposition
    \item Entanglement
    \item Measurement
  \end{enumerate}
\end{frame}

\begin{frame}{Introduction to qubits}
A qubit is a vector:

  \[ α\ket{0} + β\ket{1} \]
  \[ α,β ∈ ℂ \]

\medskip

Measured as:
\begin{itemize}
  \item $\ket{0}$ with probability $|α|²$
  \item $\ket{1}$ with probability $|β|²$
\end{itemize}
(So we also require that $|α|^2 + |β|^2 = 1$.)
\medskip

\large\color{Emphasis}
\begin{center}
No cloning!\\
No deleting!
\end{center}
\end{frame}

\begin{frame}{\!Categorical Axiomatization of QM}
\begin{center}
\vspace*{-\bigskipamount}Due to Abramsky and Coecke (2008)\\[1.25\bigskipamount]
\begin{tikzpicture}[every node/.style={draw, circle,
                                       text width=2.3cm,
                                       text centered}]
  \def\sep{4}
  \path (0,0)     node {Linear logic}             ++(\sep,0)
                  node {$⊗ \quad I$ \\ $!$}       ++(\sep,0)
                  node {Vector spaces}            ;
\end{tikzpicture}
\pause
\\~\\
\begin{tikzpicture}[every node/.style={draw, circle,
                                       text width=2.3cm,
                                       text centered}]
  \def\sep{4}
  \path (0,-\sep) node {Quantum logic}            ++(\sep,0)
                  node {$⊗ \quad I$\\$⊕ \quad 0$} ++(\sep,0)
                  node {Hilbert spaces}           ;
\end{tikzpicture}
\end{center}
\end{frame}

\begin{frame}{Tensor product}
  \pause
  $\qubit ≔ (ℂ\timesℂ) / \,{∼}$
  \bigskip
  
  Linear combinations over bases:
%  \vspace{-\baselineskip}%
  \begin{itemize}
    \item for $\qubit$: $\{\ket{0}, \ket{1}\}$ 
    \item for $\qubit ⊗ \qubit$: $\{\ket{00}, \ket{01}, \ket{10}, \ket{11}\}$
    \item …
  \end{itemize}
  \bigskip
  
\end{frame}

%\begin{frame}{Tensor product, categorically}
%  In $\cat{Hilb}$(ert spaces):
%  \begin{itemize}
%    \item $A ⊗ B$ is the regular tensor product (${⊗} : ℂ^m → ℂ^n → ℂ^{mn}$)
%    \item $I = ℂ$
%  \end{itemize}
%\end{frame}

\begin{frame}{\!Entanglement}

  \centerline{\emph{Two-qubit states are not two one-qubit states!}}
  \bigskip

    A pair of qubits in $\qubit ⊗ \qubit$ is \term{separable}
    if it has the form
    \[ (α₀\ket{0} +  α₁\ket{1}) ⊗ (β₀\ket{0} + β₁\ket{1});\] 
    otherwise, it is \term{entangled}.

\end{frame}

\begin{frame}[label=preparation]{\!Preparation of an entangled state}
\begin{center}
\def\Qa{0}
\def\Qb{-2}
\begin{tikzpicture}
  \QCircuitSetup{7.5}
  
  \swire at \Qa from 0 ;
  \swire at \Qb from 0 ;
  
  \hadamard at (2.5,\Qa) ;
  \cnot     at 5 from \Qa to \Qb ;

  \qstate<2-> at 1.25 below by 1 from \Qa to \Qb
              state \ket{0} ⊗ \ket{0} ;
  \qstate<3-> at 3.75 below by 1 from \Qa to \Qb
              state \frac{\ket{0} + \ket{1}}{\sqrt{2}} ⊗ \ket{0} ;
  \qstate<4-> at 6.25 below by 1 from \Qa to \Qb
              state \frac{\ket{00} + \ket{11}}{\sqrt{2}} ;
  
  \finally from \Qa to \Qb state \frac{\ket{00} + \ket{11}}{\sqrt{2}} ;
\end{tikzpicture}
\end{center}
\end{frame}

\begin{frame}{Preparation, categorically}
\begin{columns}
\begin{column}{0.5\textwidth}
\begin{center}
\begin{tikzpicture}[scale=0.155cm]
  \node (a* ox a) at (0,0)  {$A^* ⊗ A$} ;
  \node (a* ox b) at (1,0)  {$A^* ⊗ B$} ;
  \node (1)       at (0,-1) {$I$}       ;

  \draw[->] (1)       -- (a* ox a) node[midway,left]        {$η_A$} ;
  \draw[->] (a* ox a) -- (a* ox b) node[midway,above]       {$\id_{A^*} ⊗ f$} ;
  \draw[->] (1)       -- (a* ox b) node[midway,below right] {$\name{f}$} ;
\end{tikzpicture}
\end{center}
\end{column}

\begin{column}{0.5\textwidth}
  \par\bigskip\bigskip
  $\name{\id_\qubit}({\star}) = \ket{00} + \ket{11}$\par
  \pause\bigskip\bigskip
  But what about the constant factor?
\end{column}
\end{columns}
\end{frame}

\begin{frame}{Scalars, categorically}
\begin{center}
A \term{scalar} is a morphism $I → I$\par

\vspace{3\bigskipamount}

For $f : A → B$ and $s : I → I$:
\[
     s • f : A ≃ A⊗I \xrightarrow{f⊗s}  B⊗I ≃ B
\]
\end{center}
\end{frame}

\againframe<4->{preparation}

\begin{frame}{Measurement, categorically}
\begin{columns}
\begin{column}{0.5\textwidth}
\begin{center}
\begin{tikzpicture}[scale=0.155cm]
  \node (a ox b*) at (1,1) {$A ⊗ B^*$} ;
  \node (b ox b*) at (1,0) {$B ⊗ B^*$} ;
  \node (1)       at (0,0) {$I$}       ;

  \draw[->] (a ox b*) -- (1)       node[midway,above left] {$\coname{f}$} ;
  \draw[->] (a ox b*) -- (b ox b*) node[midway,right]      {$f ⊗ \id_{B^*}$} ;
  \draw[->] (b ox b*) -- (1)       node[midway,below]      {$ε_B$} ;
\end{tikzpicture}
\end{center}
\end{column}

\begin{column}{0.5\textwidth}
  Consider\\$\quad\ β_{i,j} : \qubit ⊗ \qubit^* → I$\\where, for instance,
  \begin{align*}
    β_{0,0}(\ket{00}) &= 1 \\
    β_{0,0}(\ket{01}) &= 0 \\
    β_{0,0}(\ket{10}) &= 0 \\
    β_{0,0}(\ket{11}) &= 0
  \end{align*}
  This represents the case when the input qubits are measured to be $\ket{00}$.
\end{column}
\end{columns}
\end{frame}

\begin{frame}{$⊕$, categorically}
  For two qubits, the four possible measurement outcomes are
  \[ β_{0,0},β_{0,1},β_{1,0},β_{1,1} : Q ⊗ Q^* → I \]

  \bigskip\bigskip
  
  We make $⊕$ be a biproduct, so we have
  \[ ⟨β_{0,0},β_{0,1},β_{1,0},β_{1,1}⟩ : Q ⊗ Q^* → I⊕I⊕I⊕I \]
\end{frame}

\begin{frame}{Quantum teleportation}
\def\Qm{0}
\def\Qa{-2}
\def\Qb{-4}
\pgfmathsetmacro\Qma{((\Qm)+(\Qa))/2}
\pgfmathsetmacro\Qab{((\Qa)+(\Qb))/2}

\begin{center}
\hspace*{-1cm}%
\begin{tikzpicture}[scale=0.7]
  \QCircuitSetup{14}

  \begin{scope}[draw=white]
    \wire at \Qm for m from ψ to x ;
    \wire at \Qa for a from 0 to y ;
  \end{scope}
  \draw (0,\Qm) -- (9,\Qm) ;
  \draw (0,\Qa) -- (9,\Qa) ;
  \wire at \Qb for b from 0 to ψ ;
  
  \node
    [ text height = 1.1cm
    , text depth  = 0.9cm
    , text width  = 1.9cm
    , text centered
    , onslide = <1>{draw = black, fill = white}
    , onslide = <2->{draw = gray, thick} ]
    at (2.25,\Qab)
    {{\visible<1>{\small entangle}}} ;
  
  \node
    [ text height   = 1.1cm
    , text depth    = 0.9cm
    , text width    = 1.9cm
    , text centered
    , onslide = <1-2>{draw = black, fill = white}
    , onslide = <3->{draw = gray, thick} ]
    at (6,\Qma)
    {{\visible<1-2>{\small entangle$^†$}}} ;
  
  \hadamard<2-> at (1.5,\Qa) ;
  \cnot<2-> at 3 from \Qa to \Qb ;

  \cnot<3-> at 5 from \Qm to \Qa ;
  \hadamard<3-> at (6.5,\Qm) ;

  \pgfmathsetmacro\meassize{ 2cm               % gate width
                           + 1.5em             % extra for plain gate size
                           - 1.4pt - 0.35 pt } % line width adjustment
  \pgfmathsetmacro\meassizehalf{\meassize/2}
  \pgfmathsetmacro\measwidthflat{2.5cm - \meassizehalf pt}

  \draw[cwire, line cap=butt]
    (9,\Qm) -- (12.1,\Qm)   % (\wirelength,\Qm)
    (9,\Qa) -- (12.9,\Qa) ; % (\wirelength,\Qa) ;
  
  \path[fill=white,draw=black]
    ($(8,\Qma) + (0,\meassizehalf pt)$)
    -- ++(\measwidthflat pt,0)
    arc (90:-90:\meassizehalf pt)
    -- ++(-\measwidthflat pt,0)
    -- cycle ;

  \coordinate (meas meter flat mid)
    at ($(8,\Qma) + (\meassizehalf pt, -0.4em) + (-0.25em,0)$) ;
  \draw (meas meter flat mid) ++(-2em,0) .. controls ++(2em,1em) .. ++(4em,0)
        (meas meter flat mid) ++(-0.375em,0) -- ++(0.75em,1.5em) ;

  \send cbit at 12.1 from \Qm to \Qb ;
  \send cbit at 12.9 from \Qa to \Qb ;
  \node[gate] at (12.5,\Qb) {$F_{x,y}$} ;
  
  \qstate<4-> at 0.35  below by 2 from \Qm to \Qb
              state \ket{ψ00} ;
  \qstate<5-> at 4.1   below by 2 from \Qm to \Qb
              state \ket{ψ} ⊗ \frac{\ket{00} + \ket{11}}{\sqrt{2}} ;
  \qstate<6-> at 7.8   below by 2 from \Qm to \Qb
              state \ket{ω} ;
  \qstate<7-> at 11    below by 2 from \Qm to \Qb
              state \ket{xy\mathord{⊗}ψ'} ;
  \qstate<8-> at 13.75 below by 2 from \Qm to \Qb
              state \ket{xyψ} ;
\end{tikzpicture}
\end{center}
\end{frame}

\end{document}

% ASZ: I use this to compile my LaTeX document; you can ignore it
(defun my-compile (output)
  (interactive "P")
  (save-buffer)
  (shell-command (concat
    "echo Compiling..."
    " && "
    "pdflatex -halt-on-error presentation.tex" (if output "" " >/dev/null")
    " && "
    "osascript ~/misc/Skim-open-revert.scpt presentation.pdf"
    " && "
    "echo Done. || echo Error."
    " &")))
(local-set-key (kbd "C-c C-l") #'my-compile)

% ASZ: Make the Agda input method more useful
(defun my-agda-ignore (str)
  (setq agda-input-user-translations
        (cons (list str (concat "\\" str)) agda-input-user-translations)))
(loop for str in '("begin" "end" "item" "usepackage" "emph" "def" "newcommand"
                   "par" "centerline" "text" "bigskip" "left" "right" "frac"
                   "sqrt" "relax" "expandafter" "string" "vskip" "vspace"
                   ; TiKZ
                   "node" "draw" "fill" "filldraw" "path" "text"
                   ; Beamer
                   "visible" "invisible" "pause"
                   ; Application specific
                   "ket" "bra" "cat" "Qa" "Qb" "Qm" "visibly" "id" "term")
      do (my-agda-ignore str))
(agda-input-setup)
(set-input-method "Agda")
